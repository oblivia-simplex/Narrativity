\frontmatter

\mainmatter

\hypertarget{the-killers}{%
\section{The Killers}\label{the-killers}}

Kathy Acker

I wrote a piece today and it's really just the beginning of something on
narration. I've got time trouble because of overwork. You know: you're
broke, and you take on too much work, and then you fuck up on
everything. It's never ending.

So understand this is simple and it's the beginning of something. You
know, it's totally open rather than anything else.

And I thought we were going to touch on two issues talking about
narration. One is identity. That's an obvious issue, the business about
who am I---you know, this is all postmodernist shit. What, therefore, is
real? Is there any real? And the other issue is raised by the quote from
this little card that advertises this panel about writing for century's
end. Constantly there's this trouble hovering in the air. The immediacy
of all the danger we're living in, of the possible end of both humanism
and of humanity, that we might all be going the way of the dinosaurs
very fast. And that culture is going the way of? of whatever---dinosaur
culture. And when we talk about writing these two things, these issues
hover there.

But let me start, since we're talking about narration, by telling you a
story. Very simple. My story begins with my friend Bob Glück who one
day, once upon a time, as story structure goes, told me that he has a
certain habit. A habit of circumventing his own habits by asking his
friends to give him a reading lesson. What should he, Bob Glück, read?
Perhaps in order to improve himself, or perhaps in order to do the
opposite. So Bob Glück goes to Kevin Killian, another friend, "Kevin,
what do you want me to read?"

Kevin replies, "'The Killers,' by Ernest Hemingway."

Upon hearing this story, for this is a story, I replied, "Oh Bob, how
weird, Hemingway?"

"Not weird" replies Bob. "All the English students at San Francisco
State love Hemingway."

Two nights after this happened I had the following dream. First, I have
to tell you a few details about my childhood and about before my
childhood in order to elucidate this dream for you. I never met my
father. Though he was married to my mother, he left her when she was
three months pregnant with me.

When I was 26 years old, through an accident, I traced my father's
family. I wrote to them and they wrote back that they would accept me
into their family and we arranged to meet. I thought that I was going to
meet my real father, but I only met the first cousin. He told me (and I
think he was a little crazy---well not crazy but eccentric, because rich
people are never crazy), that perhaps I should not meet my father. Why?
Because my father had murdered someone who was trespassing on his yacht.
After remaining six months in a lunatic asylum, the state had excused
him of any murder charge. My father then disappeared. No one now knew
where he was, said the first cousin. And so I abandoned my search for my
father, for my life at that time was hard enough, and this new trouble
was simply not worth the trouble.

When I was 30 years old, my mother suicided.

Enough of my childhood.

I had the following dream: I dreamt that I was looking for my real
father. In my dream, I knew that it was dumb for me to do this because
my father was dead. Since I'm not dumb, I or the dreamer thought, I must
be trying to find my father so that I can escape from this house which
is being run by a woman. I go to a private detective. He calls me a
dame.

I say, "I'm looking for my father."

The private detective, who might be a friend, replies that my case is an
easy one. I like that I'm easy. We begin our search. According to his
instructions, I tell this private eye everything that I know about my
mystery. It takes me several days for me to recount all the details. It
was summertime in Dallas. Everything was yellow. I didn't remember
anything about this first period of my life, about my childhood. After
this not remembering, I remembered jewels. As soon as my mother passed
away, a jewel case was opened. The case, consisting of one tray, had
insides of red velvet. Perhaps I'm dreaming my mother's cunt. I'm given
a jewel which is green. I don't know where that jewel is now. I have no
idea what happened to it. This is the mystery of which I'm speaking.

The private eye pursued my matter. A couple of days later he came up
with my father's name: Olen. This name means nothing to me.

"Olen. Your father's name is Olen. Furthermore, your father killed your
mother."

I thought, in order to dismiss thought, that's possible. The detective
continued to give me details about my father. He's from Iowa, and of
Danish blood. All this could be true, because how can I know anything?

Now when I woke up out of this dream, I remembered details about my
mother's suicide. She had killed herself eight days before Christmas. A
note in her handwriting, lying beside her dead body, said that her white
poodle was staying with such and such veterinarian. Nothing else. But
despite this note the cops were convinced that my mother was murdered by
a man whose name wasn't known. Nevertheless, it was Christmas and there
wasn't going to be any police investigation because the cops wanted to
return to their homes, Christmas warmth, and holiday.

For the first time ever in my life I had the following thought: My
father could have murdered my mother. After all, what if my real father
is crazy? At that moment I became very scared. If my father did murder
my mother, he could now be planning to murder me.

Now last week I was touring around and I found myself on a gig in
Roanoke, Virginia, where I met this absolutely wonderful writer, Richard
Dillard, and Richard told me that when he had been a boy he had
encountered his first proposition by a man during a showing of "The
Killers" at a local Roanoke theater.

That's the end of the story.

Also I should mention that in "The Killers" the name of the Swede who's
about to be slaughtered by those killers is Olen.

Now, is this a story, what I've just told you? Certainly it's narrative,
right? And each incident actually did happen. That is, it was real. But
a story? Stories are about something. And what I've just told you---it's
not about anything. It's not even about me. A story, you see, a story
has something to do with realism, and what I've just told you, though
each little bit was real or had happened, has nothing to do with
realism.

Listen to another voice as we proceed on this treasure hunt, or
narration. The voice of Julio Cortasar.

"Almost all the stories I have written belong to the genre called
fantastic, for lack of a better word, and are opposed to that false
realism that consists of believing that everything can be described and
explained, as was assumed by the optimism of 19th century philosophy and
science. That is, as part of a world governed more or less harmoniously
by a system of laws, principles, cause and effect relations,
well-defined psychologies, etc." That lovely world order---that
governing harmoniously by all these principles---we all know is now
over. Though Bill Clinton has a hype.

Okay, a little more on realism. Richard Dillard rightly calls that
narrative structuring named realism "reductive."

"But isn't that the central story of 19th century realistic fiction?" he
exclaims in his review of Alester Grey's marvelous novel \textbf{Poor
Things}. "The central story of realistic fiction: we are born, we are
misshaped by biological inheritance, economic forces beyond our control,
and cultural biases beyond our recognition, and finally we die with our
failed dreams on our dry lips."

Realism: reductive and dehumanizing.

"There's another order," says Cortesar, "more secret and less
communicable. That the true study of reality lay not in laws but in the
exceptions to those laws has been the guiding principle in my personal
search for literature beyond all naive realism."

Now return to the narration that I told you. I can't even say what it's
about: Bob Glück's conversation, my dream about my real father, Richard
Dillard's memory. Where in this narration lies the real? It lies in the
connections between the three sections; in the connections between the
"real" events and the holes, the silences. In the slippages. Slippages
into what?

In Moby Dick, Melville speaks of reality as the interstices through
which all of us fall. There's another, a further way of putting this, of
proceeding on our treasure hunt. These interstices can be named chaos,
or places where language cannot be, or death. When storytelling, humans
attempt to cling to meaning. I think of narration, of narrative, or that
narrative which tries to encounter the real, as that which is
negotiating between two orders of time: clock time and chaos. The writer
is playing---when structuring narrative or when narrative is structuring
itself---with life and death. He or she is maneuvering between order and
disorder, between meaning and meaninglessness and so is making
literature.

However, these movements between clock time and chaos in written
narrative are more structurally complex. To begin, consider one aspect
of time in the novel: time. What? The time it takes to write a novel. A
novel's a big thing. I'm talking about novel narration at the moment
because that's what I deal with. It usually takes at least a year, often
many years. During that time the writer's life changes. So there's the
time of all the actual changes the writer is going through---the time it
takes to write the novel.

Two: the time it takes to read a novel. A novel is very rarely something
you read in one sitting. So that time it takes to read a novel
incorporates all the readers memories, all the interstices of time, the
time lapses between readings, all the returns to earlier parts of the
novel, etc. Three: the fictive time. The time within the story or the
narrations. Even Hopscotch---there's fictive time there. So in this
sense a novel, structurally, is a time triad.

Rather than continue with this---and you can start analyzing novelistic
structure---I want to just quickly skip to a final point. Consider again
realism. And I must say the more I think about it, I think writing is
more and more about time. Consider again this thing of realism (I'd like
to abandon the whole thing). Within the realm of realism lies the
assumption that language mirrors all that isn't language, right? A
table. That what a narrative is about: telling what is or should be
true. That a narrative mirrors reality.

Now do I need to say---how simplistic. But what I want to ask is, why
bother being so simplistic? Why bother with the lie of realism? Why
bother being so miserable, reductive, when one could play? If I'm going
to tell you what the real is by mirroring it, by telling you a story
which expresses reality, I'm attempting to tell you how things are. By
letting you see through my own eyes, I give you my viewpoints, moral and
political. In other words, realism is simply a control method. Realism
doesn't want to negotiate, open into, even know chaos or the body or
death. Because those who practice realism want to limit their readers'
perceptions, want to limit perceptions to a centric---which in this
society is always a phallocentric---reality. "I am the one," says the
realistic writer, "I'm telling you reality." And we can talk. I have the
same quarrel about narrowing anything to single identity.

In other words, behind every literary or cultural issue lies the
political, the realm of political power. And whenever we talk about
narration, narrative structure, we're talking about political power.
There are no ivory towers. The desire to play, to make literary
structures which play into and in unknown or unknowable realms, those of
chance and death and the lack of language, is the desire to live in a
world that is open and dangerous, that is limitless. To play, then, both
in structure and in content, is to desire to live in wonder.

\hypertarget{note}{%
\subsection{Note}\label{note}}

Kathy Acker gave this talk on a panel called "In Extremis, Writing at
the Century's End." The panel was presented by Small Press Traffic in
the New College Theater in San Francisco on April 29, 1993. Fortunately,
the event was videotaped by S.S. Kush for his Cloud House Poetry
Archives (1557 Franklin St.~San Francisco CA 94109, 415-292-5554). We
are grateful to Kush and also to the Acker estate. The tape was
transcribed by Quintan Wikswo and edited by Robert Glück. The panel also
included Dodie Bellamy, Earl Jackson Jr., and Benjamin Weissman.

\hypertarget{last-exit-to-victoria}{%
\section{Last Exit to Victoria}\label{last-exit-to-victoria}}

Lawrence Ytzhak Braithwaite

(5mins. of Murder Alone with \textbf{Wigger} and ``Bloodland'')

~

~

~

\hypertarget{section}{%
\subsection{1}\label{section}}

...as a child I was told that not knowing the alphabet will cause
illiteracy. It'll send you into a drugged out gangland life of white
trash nightmares and corner boy peddling to homosexuals, who are
professional players, obsessed with age and are willing to drag it and
you into emptiness. That in knowing the letters, Ill know that they
assemble to construct various images that become words. Words are the
narrative transformation of the images. Printing a page of unbroken
words is like a fresh tattoo. It captures a moment/place, sentiment and
period. It orchestrates the body in motion as it flexes to move a
pen/strike at a key/form a fist/lift a drink or move to a rhythm. The
words become the unspoken intertextuality of ethnic, racial and cultural
metaphoric speech. The metre of casual dialogue = a
rhythm/noise/visual/bass, a soundtrack to a post-literate train of
thought.

I dedicated \textbf{Wigger} to the sound of Slayer and Sonic
Youth---"Bloodland" to Cannibal Corpse (George "Corpsegrinder" Fisher) =
I dedicated them to sound. Slayer is for the fury and speed and violence
that the book has. Deathmetal is the living desire of the neo-redneck
burnout. It's all going after the sport of brutality---the art of
hurting someone. The walking jokes, with targets on their backs, placed
there by lilbitches, taking them for Mn Ms.~The only violence is, the
way the words appear on the page, marked by the slashes that connote
rhythm of speech and interrupted thought. They are like semi-colons = /
the // are colons and so are the = signs. Sometimes the - move out to
separate speech---someone takes lead//does a solo. Sonic Youth's sounds
appear in the form of the book dissolving. Deathmetal is just dealing
with the situation. The book and the story, like the characters, are
trying to hold themselves together. Brian and Jerehmia symbolize this
the most. They talk. No one listens.

...a printer ejects swatches of stolen tissue that collects the sound
and images of what is considered low brow art and skill//

Hardcore (H.C.)/Country/Rockabilly

Thrash and Speedmetal/Rap/Deathmetal/Blackmetal

Cinema/Television and;

Comics Books

Brian appears to be rambling but he makes the most sense because he
speaks through metaphor and his heart, like Black speech and song---the
negro spiritual, seeming to sing about a river,

when it is a code,

for an escape,

at a certain time,

from the plantation.

Gangsta rap seems to boast

about an evening

or event

but is just shouting

validity,

existence and

enterprise.

Brians grasp on reality becomes inconsequential.

Jerehmias story is dumbshow.

Brian surrenders his ideals and his soul, by killing Spook (the Black
hustler friend).

Jerehmia is just having his taken and he's telling you about it---but I
knew people in Atlanta that didn't want to hear about any of that or
them. I knew a boy from Tennessee hurting worst than what was going as
oppression under those 100 dollar Afro centric hats.

...Wordcore, Jamaican dub poetry, Rap and Rock Steady are disposed to
the Homeric boasts and catalogues of post-modern thugs/hoodlums and
desperadoes w/ the hope of kindness and compassion.

In donning the Black persona, symbolized through the silver jacket,
Brian finally does what everyone has been attempting to do throughout
the book. Brian is killed---his soul is killed, through the burden of
the weight of the Black youth---the Black persona, the persona of
deglamoured oppression. He has achieved the goal of being Black but he
is unprepared to handle something that the Blacks are raised to deal
with through centuries of struggle---youd suppose.

I met a boy with crystal smack grey eyes, who was offering to suck the
fender (for \$5) off an urban assault vehicle for a painkiller.

If you listen and see the page, you'll see a tattoo and note well, that
some words are//

-Rhythm

-Lead and

-Bass

Jerehmia, he has no desire to be Black. He knows what the shit is and
he's got his own. He just watches the seemingly liberal Perry go catch
back the Black boy trying to bail on the whole scene. He just knows that
nothing that's happening, is about him or the niglet. It's about the
little men with big shirts and the chicks in bed or sitting off from the
Bar b q..

...strange how all the youngmen holding up walls waiting on passers by
had deep drawls, no smiles and trailers in their hearts.

The book and the story are collapsing. The spaces and sparsness of the
narrative is there to get you to listen carefully and read again and
close read and not take things at face value. People say more,
especially, when they've had a lifetime of hurt, through very few words
and the words look in ways that they feel---speeching in accapella. The
words sound the way they feel. The book shows patterns of speech.

Mcluhan would argue the global village: technology is good---the book
people are lacking in the rich grammar of the TV = This is referred to
today as the post-literate generation---John Cages bastard children turn
to violence and the acoustic space of video and computer games and the
delicacy of words typed over television screens.

Mcluhan would argue...

...the acoustic

~

~

~

\hypertarget{section-1}{%
\subsection{2}\label{section-1}}

~

The Comic book uses frames and juxtaposition to connect the
disconnectedness of thought and words, be it through the gruff noireness
of Frank Miller, Dave Sim or Martin Wales "Kinder Nacht," or the smooth
lines, colours and ruff justice of Todd McFarlane. They collect the
street sounds and activities the way Rap culture manipulates and
emulates, via spin art and samples, the ready made terror of interurban
life. To frame is to make perfect the moment of the fingers striking the
keys, which can be the repetition of words and phrases = outlets ripping
and shredding, w/ the knowledge that framing is certainly an attempt to
make perfect, to make the words as enticing and elegant as the bovine
euphony of a fascist's goose step.

Brian is condemned to sit, numbed out on pills, in a room, filled with
drag queens, watching them transform. Just like Brian's desire, and the
other characters desires, to be Black, the drag queens attempt to be
women, without knowing, that beyond the glamour, is a lot of hardship
and struggle. Drag queens die, as well, when they finally achieve their
dream of being a woman. It's not as much fun or better than they thought
it would be. They are unprepared for the daily struggle and threats. You
can see this on the page. They are boxed in, allowing themselves to be,
finally, separated from the freeness and flow of the book. They must
stay within the parameters. The glamour is gone. They have to bear the
actual pain. Brian has got the auditory recollection of rape in his
head. Brian is numb. Sos Jerehmia, but he's numb from the beginning, any
way. Hes just recollecting, now---no need to figure. He cant hold it
together, nor does he want to. His flag is upside down, Oswalds
dead---nothings right. You separate the body from the mind and its all
good.

...a word can be the hums or bops in the background as when a funk
musician beats the strings w/ his thumb---the buzz and scratch of a tat
gun on a steadied arm or back---the stilled breath and firm muscle of a
word or image engraved on a belly. It gets tied into the paragraph. Some
words take on the repetitive ecstatic riff of ska or reggae or the decay
of the crash/noise of H.C. or metal. They recline on the page or in the
air as if they were all going to amount to a junkies last sigh. Instead,
they collapse into the lost, disorientated and somewhat satisfied image
of Orsen Wells after thrashing a room at Xanadu.

...and the book and story crumble into a final sustained note, done
through a note to Andrew from Jerry. "Bloodland" has a crack of gun shot
and shit and a heart beat. As it began on a note---it ends. His (Jerrys)
speech is proper and distant from the language of the book. He has
become a literary character of the Modernist sense.

Jerking the Modernist approach to distance, into "groovy times," is the
dumbshow to the contemporary adoration of the absurdist tragic comedy =
anti-romantic. It presents itself, today, through the jaded situationism
of talkshow culture. The sort that "Blast" and "Counter Blast" utilized
through the headlines and print of newspapers. The newspapers place
horror and glory side by side in a folio that has a calculated
randomness. It will always assemble to state 1 thing = these events
happen. It is only through such genres of media
(music/comic/print/television), that we can see that each has a thing in
common/coupons and advertising = rhetoric and metaphor. If the passage
of a text takes on the sound and image of the disjointedneness of casual
speech and media, then the results are not only the coded tales being
woven, but the presentation of the brutality that leads to the violent
outcome.

Jerehmia, he just figures...hes just feedback from a bass, right now.

\textbf{Wigger} and "Bloodland" are what the printer jetted out---a
conceptual pastiche alphabet from the corner or gutter. It assembles the
images in sequence which persuades the narrative to be pulled out from
under a character. Its what got them blocked off---formatted, layed out,
then pile-drived with carefully selected skin grafts of onomatopoeias.
The words takes Audens diver, diving w/ his brilliant bow and takes it
beyond the enthralled spectator. They suppose that he has forgotten to
test the water and has come out a dumb struck paraplegic = what just
happened/where are our people/who are our people/do we have any people:
NO. However, they still keep tabs on the moment, place sound and period
that left the object devastated, hung over, and w/ an image permanently
scratched and coloured onto his flesh or hanging in a window.-\/-

Oi Cheers,

Laz

\hypertarget{flange-or-premises}{%
\section{Flange or Premises}\label{flange-or-premises}}

Mary Burger

Can I say it went on for a week. Can I say I woke up each day, and went
through each day,

"You were there, and you were there, and you, and you!"

Word went out, a consciousness with a consistency, looked at itself
looking at itself, halted the infinite regress, the mirror facing the
mirror, a language made possible a belief in location.

I wanted to speak as someone who had experiences. I wanted to speak as
someone who knew others. To have things happen, and to speak of them,

someone told a story to which I was unable to respond. The story was the
story itself.

~

I want the other thing. Hinges and holes and string. The impossible
appearing to be,

an engine, a large machine, an apparatus, a stage for a scene.

~

Can it be a narrator who doubts its premises?

The crumbling, if something will, whom it will happen to.

\begin{quote}
"Lyric, whose operation is display, and narrative, whose operation is
seduction.

Who will speak for the beside-the-point?" (Aaron Shurin, Narrativity)

~
\end{quote}

As what happens then, if what happens is always something more.

~

~

\hypertarget{despite}{%
\section{Despite}\label{despite}}

David Buuck

\hypertarget{section-2}{%
\subsection{1}\label{section-2}}

One narrative theory is that Agents are social beings. Equidistant from
Los Alamos and Roswell, the artist Walter de Maria places poles in the
landscape in order to construct a grid. The land is not the setting for
the work but a part of the work. The poles are angled at eighty-eight
degrees else lightening always strikes twice. The primary experience
takes place within the grid. Weather is total.

One narrative theory is that social beings are agents. The grid is the
imaginary frame by which it becomes possible to sense the spatial
landscape. The grid dwarfs Agents, the grid in turn being dwarfed by the
landscape, provoking Agents to realize their own relation to the spatial
landscape. In that the grid itself cannot be comprehended in its
totality, neither can the spatial landscape. Castles are sand that can't
be shelter to kings. In that the spatial landscape cannot be
comprehended in totality, neither can there be kings. Rabbits nibbling
on apple cores are nervously aware that there can't be shelter from
weather. Humility is nervous awareness of Agents not being rabbits.

There are three agents in the landscape who are not kings. The Agents'
relation to the grid becomes paramount. Agent A's scale becomes
paramount. Agent B's silence becomes paramount. Agent C's isolation
becomes paramount. Humility is nervous awareness of Agents not being
kings. The invisible is real.

The spatial relation of sky to the landscape is such that the clouds are
pressed flat against an imaginary plane of glass. The imaginary plane
rests upon the grid. One narrative theory is that Agents' minds function
as the imaginary plane. The action is the air and wind and rain.
So-called natural elements are grounding such that Agents' social being,
in isolation, in being dwarfed, is grounded. That this is not separate
from Agents' social being, but rather central to Agents' realization as
social beings, is one effect. Rabbits nibbling on apple cores are
nervously aware that lightening always strikes twice. Humility is
nervous awareness of Agents not being total. The primary experience
takes place within the social grid. Distance is total.

\hypertarget{section-3}{%
\subsection{2}\label{section-3}}

One narrative theory is that one never dies in photographs. Equidistant
from Hiroshima and Nagasaki, the artist Kenji Yanobe builds nuclear-safe
capsules within which he is photographed while being in Chernobyl. That
the artist Kenji Yanobe would be in Chernobyl inhabiting sculptural
objects for the ostensible sake of safety is troubling to one. That the
photographs of the artist Kenji Yanobe inside nuclear-safe capsules
while being in Chernobyl are aesthetically pleasing is troubling to one.

One narrative theory is that in desiring the apocalypse it is
continually deferred. The artist Kenji Yanobe builds nuclear-safe
capsules for sexual acts. Such acts are expressed outward as exhalations
of perfumed music, compressed air modulated through anthropomorphic
trumpets, so-called safe sex ironized into cartoon farts. That
nuclear-safe capsules for sexual acts are aesthetically pleasing is
troubling to one.

One narrative theory is that one's apocalypse is inside. Such apocalypse
is not nuclear-safe, such sex is disharmonious, such art is a cartoon
that is actual. Agents inside nuclear-safe capsules are not safe from
themselves. Agents cannot be insulated in machines that are cartoon, in
art that farts its own radioactivity.

There are three Agents in the capsules who are not cartoons. Agent
Yellow breathes its own air. Agent Red eats its own waste. Agent Orange
is the machine that desires its own apocalypse. That the technology
necessary to provide safety for Agents is the same technology that
produces the effects from which Agents desire safety is troubling,
three, two, one. Castles are cartoon machines that can't be shelter to
kings. There are no kings, only agents.

\hypertarget{section-4}{%
\subsection{3}\label{section-4}}

The sky is always in relation to Agents' landscape. Even underground,
though that's a dream. In dreams Agents' own minds are the relation of
the imaginary plane to Agents' own underground landscape. The grid's
only function is to provoke Agents' realization of Agents' own landscape
in actuality. The grid exists so as to become invisible. The grid exists
so that it doesn't have to. The invisible is real. Dreams are real.
Shelter is context.

One narrative theory is that the sun never sets. Agent Yellow, from a
distance, disappears into the landscape. Agent Red is still
disappearing, though Agent Orange is still here. Perspective is
constitutive of Agents' social being. Agents have different perspectives
that are relational. Total perspective is ideology that can't be shelter
to kings. Perspective is constitutive of Agents' social bearing. Context
is shelter.

Being dwarfed is perspective that reduces distances between social
beings. Relations between social beings provoke social bearings within
the grid. Roswell and Los Alamos are two landscapes that provoke such
social bearings. That the historical landscape dwarfs Agents, provoking
such social bearings, is troubling to one. Hiroshima and Nagasaki are
two landscapes that provoke such social bearings. Social bearings
realize social grids that produce the social landscape. That Agent A
might become grounded, in its isolation, is one effect. That Agent B
might become grounded, in its social being, is one effect. That Agent C
might become grounded, in its social bearings, is one effect. The social
grid is constructed by the many social beings. Its construction is
relational to the many social bearings. Perspective dwarfs the
historical landscape. The sun never sets.

\hypertarget{section-5}{%
\subsection{4}\label{section-5}}

Dystopia is this topia, the cynically willed non-place of the
apocalyptic imaginary. Its filmy visuals are projected onto the
imaginary plane of glass that rests upon the social grid. The sky is the
image of the impending apocalypse. Its spectral traces appear
transparent when viewed from the ground up. The sky is always falling,
else the bottom falls out. Social beings labor at constructing the grid
to support the imaginary plane. Agent-sweat is radioactive. Agents drink
it as refreshment while awaiting the coming attractions.

One narrative theory is that one never dies in film. At the end of a
long day of grid-building it is nice to lie on the ground and watch the
bottom of the sky fall onto the imaginary plane. The ground is the
imaginary plane of consciousness upon which social beings construct the
social grid. Ideology is the nightmare made material by certain social
bearings. The ideological grid exists so that it might become invisible.
The ideological grid exists so that it doesn't have to. The invisible is
reel-to-reel.

One narrative theory is that Agents are never awake, except in history.
So-called nightmares are in actuality the history that Agents desire to
have been merely dreamt. Underground is the collective repressed. At
night it dreams itself through rabbit-holes into the social grid. One
narrative theory is that one never dies in dreams. Rabbits nibbling at
nuclear cores are nervously aware that history always strikes twice.
Humility is nervous awareness of Agents awaking from dreams and being in
history.

\hypertarget{section-6}{%
\subsection{5}\label{section-6}}

The artist Kenji Yanobe's sculptural objects, in that they provide
habitation, in the form of objects of safety and art, while at the same
time being distancing, in that Agents are further distanced from the
actual world, while still being in it, are troubling to one. That the
artist Kenji Yanobe would be in Chernobyl inhabiting sculptural objects
for the ostensible sake of art is troubling to one. That the art
necessary to provide safety for Agents is the same art that produces the
effects from which Agents desire safety, is troubling, four, three,
agents, two, one. Agents are never safe from themselves.

The artist Kenji Yanobe's photographs, in that they document art
providing safety, while at the same time being distancing, in that they
are further distanced from Chernobyl, while still being in it, are
troubling to one. The photographs are viewed outside of Chernobyl, in
place of Chernobyl, as art and as safety. They are already happening,
though in actuality they are not Chernobyl. Distance is safety. There's
no distance in actuality. There's no safety.

The artist Kenji Yanobe has not yet built nuclear-safe capsules for
rabbits nibbling apple cores. One narrative theory is that nervous
awareness is not enough to prevent an apple core breach. That the wind
from Chernobyl might kill the rabbits is troubling, four, them, three,
to one. That the repeated vocalization of "Chernobyl" might become
aesthetically pleasing by virtue of its repetition is troubling to one.
There are sixty-six dead rabbits in the grid. Their bottoms have fallen
out.

\hypertarget{section-7}{%
\subsection{6}\label{section-7}}

Sentry posts guard the perimeter. Their dogs are in nuclear-safe
capsules, making it difficult to chase the rabbits. Acid rain leaks into
the underground reservoir. Social beings labor into a future that is
still undecided. Agent-sweat is radioactive. Agents drink it as
refreshment while awaiting the sky to fall. One narrative theory is that
the sentry posts face in rather than out. Each pole waves a flag
advertising the coming attractions. One narrative theory is that there
is no perimeter.

Social beings that are also rabbits that live in underground bomb
shelters. Castles made of pamphlet bomb shreddings can't be shelter for
rabbit-kings. Rabbits have infrared night vision that always strikes
twice. They drink from the radioactive reservoir that runs from Roswell
to Los Alamos. That the wind from Los Alamos might kill the rabbits is
troubling to one. The sentries put the sixty-six dead rabbits into piles
to climb up and up. Else the bottom would fall out. That the repeated
vocalization of "Los Alamos" might become aesthetically pleasing by
virtue of its repetition is troubling five, four, three, to one.

One narrative theory is that dreams are underground nuclear tests. They
mushroom into consciousness through rabbit-holes and anthropomorphic
trumpets. Cartoon farts are air-raid sirens that go up and up. The music
is the bomb. It makes the bottom fall out.

\hypertarget{section-8}{%
\subsection{7}\label{section-8}}

The underground compartment containing radioactive warheads is called
"The Pit". The warheads are angled at eighty-eight degrees, else the
bottom falls out. Agent-sweat is radioactive. Agents drink it as
refreshment while awaiting the bottom to fall out. That the bottom would
fall out onto Agents' heads is troubling, six, five, four, three,
troubling to one.

One narrative theory is that there is an underground nightclub called
"The Pit," where rabbits push up and up. The player piano is programmed
to perform eighty-eight degrees of sound, else the bottom falls out.
Music is the distance of time within sound's detonation. The speed of
sound is measured by the distance between lightening and its thunder.
Agent Yellow never strikes twice. Agent Red is not a rocket scientist.
Agent Orange is the bomb.

The speed of sound is measured by the distance between the bomber and
the detonation of its payload. The warhead is programmed to perform
eighty-eight degrees of separation. Its coin slides into its slot. Its
music is the bomb, whose bottom falls out. The speed of light is
measured by the detonation of the bomb within Agents' own bottoms
falling out.

\hypertarget{section-9}{%
\subsection{8}\label{section-9}}

One narrative theory is that one never dies in sculpture. Kenji Yanobe's
sculptural objects evoke a nostalgia for an apocalypse that never
happened. It is a nostalgia for the future of a past that is continually
deferred to a present that hasn't happened yet. It's already happening.
The so-called-and-ever-wished-for last instance never arrives, and yet
the present is-and-always-is this very last instance. The apocalypse is
always happening.

One narrative theory is that one consumes or is consumed. The artist
Kenji Yanobe has produced a never-ending simulation of a survival cart
race, which will protect Agents from radioactivity as long as Agents
keep inserting coins into their nuclear-safe capsules. Agents must exit
the capsule in order to insert the coins into its exterior
anthropomorphic coin slots. There is no future here, only the present
which enables Agents to survive as long as Agents keep consuming. The
rabbits chew through the string that is used for the coin-slot coin
trick. The only safety is to insert coins into another's coin slots in
hopes that the other will do the same. Distance is the opposite of
safety.

The artist Kenji Yanobe stands at the sentry post facing out, toward the
ruins of the future. Rabbits have infrared vision that can see through
the nuclear-safe capsule. The inside is lined with rabbit fur. It's a
never-ending simulation of eighty-eight degrees inside. Agent-sweat is
radioactive. Agents drink it as refreshment while awaiting the survival
cart race.

\hypertarget{section-10}{%
\subsection{9}\label{section-10}}

The social grid exists so as to become invisible. The social grid exists
so that it doesn't have to. That the invisibility of the social grid
might provoke the realization of the ideological landscape is troubling
to one. Even underground, though that's a dream. The so-called social
elements are the imaginary plane of social beings. The imaginary plane
stretches out into the distance, held up by the many social beings.
History never disappears.

The imaginary plane reaches from one pole to the next, grounded by the
many social bearings. The present disappears into the historical
landscape, though it's still here. That such disappearance into the
historical landscape might be the forgetting of social bearings is
troubling to one. Even underground, though that's a dream. History is
the imaginary plane of social bearings. Here it is now, stretching out
into the distance, held up by the many social bearings.

The social grid extends indefinitely into the future. Social beings
labor at building it outward. The apocalypse of totality is continually
deferred. Humility is nervous awareness that Agents are grid-workers.
That the social bearings necessary to destroy the grid are the same
social bearings that build the grid is troubling, eight, seven, six,
five, four, three, agents, troubling, to, one.

\hypertarget{section-11}{%
\subsection{10}\label{section-11}}

The Agents are in their survival carts. The carts fart perfumed
radioactivity through anthropomorphic trumpets. Agent Orange breathes
its own air. The music is air-raid sirens that go up and up. The
rabbits' infrared vision burns through the string that is used for the
coin-slot coin trick. Agent Red eats its own waste. It's a never-ending
simulation of eighty-eight degrees inside. Agent Yellow drinks its own
sweat. Agent-sweat is radioactive. Agents drink it as refreshment while
awaiting the coins to run out.

Agent A puts its coin into Agent B's coin slot. Agent B puts its coin
into Agent C's coin slot. Agent C puts its coin into Agent A's coin
slot. There are no kings to make more coins, so the grid-poles are
dismantled and melted into currency. Without the poles, the imaginary
plane's bottom falls out. The grid collapses into itself. The perimeters
collapse into their centers. Agents trade in their narrative theories
for the few remaining coins. The sun is setting. The social bearings are
detonated. In the last instance coins fall to the ground in piles that
go up and up. The sky is falling.

Agent Yellow's survival cart is red. Agent Red's survival cart is
orange. Agent Orange's survival cart is yellow. As they collapse into
one another, the detonation provokes the sunset. From a distance the
shutters click shut. That the sunset is a film that is aesthetically
pleasing is troubling, ten, nine, eighty-eight degrees, seven, sixty-six
rabbits, five, four, whom, three, agents, troubling, to, one. Action.

\hypertarget{postscript}{%
\subsection{Postscript}\label{postscript}}

One narrative theory is that there is always another narrative theory.
Action is choosing. The film reel is stuck in the machine. The so-called
sunset is merely the melting celluloid of one possible future. The
machine is the apocalyptic imaginary that desires its own foreclosure.
Agent Orange has killed off Agents A and B, Agents Red and Yellow. But
Agent C has managed to elude such foreclosure. New grids are being
constructed by social beings with different social bearings. Action is
choosing made material by certain social bearings. Humility is nervous
awareness that action is total. It's already happening.

One narrative theory is that one never dies in writing. The writing
cannot in actuality provide safety. It is a machine desiring its own
apocalypse. In actuality it is its own social being, desiring its own
narrative theory. Here it is now, stretching out into the distance, held
up by its social bearings. Agent C is writing itself onto the imaginary
plane. The script has yet to be finished. It's already happening.

~

\hypertarget{narrative-is-boring}{%
\section{Narrative is boring}\label{narrative-is-boring}}

Jacques Debrot

Narrative is boring because it precludes the direct actualization of the
world through our perception. But there is no alternative; narrative is
both a disabling and a necessary condition of perception. For this
reason, boredom is the affective condition that writing predominantly
negotiates and exploits. However, it has always struck me that boredom,
despite its stigmatization, is, in reality, a complexly ambivalent
emotion. Somewhere Roland Barthes--in \textbf{Roland Barthes} I
believe---describes boredom, suggestively, as a form hysteria (or is it
stupidity that he calls hysterical?). Whatever the case, I try to make
my writing both as boring and as stupid as I possibly can---in Barthes'
words, "I agree to pluralize myself, to permit free cantons of stupidity
to live inside me." Error, play, bad taste, incomprehension, artifice,
and a lack of truth or reasonableness are, thus, the narrative
potentialities I attempt to fulfill.

Paradoxically, with the loss of narrative's legitimizing and explanatory
functions, in much of the most interesting contemporary writing there
has been an increasing recourse to, rather than a retreat from,
narrative forms. My own work is transversed by multiple, parataxic
narratives and entangled discourses. Ideally (and impossibly) I would
like to produce a narrative that would need to be read in the same way
one "reads" a painting; that is, by orienting oneself gradually, by
means of various intensities and expressive deformations of
surface---the textual equivalents, if they exist, of the mechanisms of
collaging, dribbling, scraping, scratching, scumbling, attaching,
interposing, reversing, permutting, and so on---to the entire
environment and atmosphere of the writing; a method aligned not so much
with an aesthetics of fragmentation, but rather, one that proceeds
through the \emph{accretion} and flux of many ephemeral ruptures and
stases.

This, obviously, entails a fundamental reconsideration of what is
traditionally classified as narrative form. However, for me, the central
question is not a taxanomical, or a typological one. More important than
the question of whether writing such as mine is narrative, is the
problem of how, in what sense, this writing can be read as narrative.

The situation of communication---the roles of author and reader, the
relations of subject positions within the narrative itself, as well as
the historical contexts in which the meanings of the narrative
circulate---cannot be excluded from the kind of reading my work attempts
to invite. But even more specifically, I am interested by the aesthetic
situation in which my writing occurs. By this I mean I am concerned most
of all with my feelings, and with my tastes and my preferences. Like
everyone's, my experience of art, as Thierry du Duve argues, "begins and
ends with aesthetic judgement." It is impossible---at least on aesthetic
grounds---to be insincere to one's tastes. However our preferences are
not objective, and our feelings, although they are our own, are also
social constructions. Inasmuch as literature and narrative are no longer
given domains but generative practices, what I want to experience in the
act of writing, or reading is to feel, as du Duve puts it, the conflict
of values both within myself, and within and for the cultural field as a
conflict of feelings, "to give in to the reflexive feeling of dissent. .
. . To give my consent to the felt absence of concensus."

It is this ongoing dissent-\/-that is to say, my resistance to any
single unitary or consummating construction---that \emph{is} the self
from which, as much as the goal toward which, I write.

~

~

\hypertarget{text-and-the-site-of-writing}{%
\section{"Text" and the Site of
Writing}\label{text-and-the-site-of-writing}}

Jeff Derkson

\begin{quote}
"That the widespread textualization of the outside world in contemporary
thought (the body as a text, the state as a text, consumption as a text)
should itself be seen as a fundamental form of postmodern
spatialization..." (158)

-\/-Jameson, \textbf{Postmodernism: Or the Cultural Logic Of Late
Capitalism}

"For its part the poststructuralist version of postmodernism worked to
exceed both formal aesthetic categories (the disciplinary order of
painting, sculpture, and so on) and traditional cultural distinctions
(high versus mass culture, autonomous versus utilitarian art) with a new
model of art as text." (72).

-\/-Hal Foster, \textbf{The Return of the Real}

"A provisional conclusion might be that in advanced art practices of the
past thirty years the operative definition of the site has been
transformed from a physical location---grounded, fixed, actual---to a
discursive vector---ungrounded, fluid, virtual."

Miwon Kwon, \textbf{One Place After Another: Notes on Site Specificity},
October 80 (Spring 97): 95.

~
\end{quote}

I've started with this top-heavy loading of quotations to highlight a
general movement from textualization of (generally) the "outside world"
and place/site alongside a similar movement within (generally) art and
site-specific art in order to establish a frame to address how the
"place" of writing and cultural production could be rearticulated. Aside
from a poststructuralist tendency to read structures as texts, how does
this "textual turn" implicate writing as a constructive act? Has the
modernist literary project of writing an imagined world been deflected
through poststructuralism so now the world is imagined as not only
written, but as text?

Parallel to this "textual turn," which leads intriguingly to Mowing
Kwon's vague "discursive vectors," there has been both a related mapping
impulse and a constructive intent directed at "place." The mapping
impulse is both ontological and geographic (writing as a mapping of a
mind or of subjectivity, writing as part of the process of realizing
"place"). In popular media, texts, particularly novels, are given a
primarily ontological role, of telling us something about the places we
live in, and by extension to tell us something about "ourselves" or to
illuminate the author as subject. Geographically, the constructive
intent is perhaps clearest in a national literature's assembling of
images and icons to create the imagined place of a nation and to the
related levels of regional literatures and urbanist texts.

In contemporary art, a sort of sociological turn and mapping has
emerged. Hal Foster cites Dan Graham"s ``Homes for America'' which
"mapped" typologies of American suburban houses as well as the taste
that they constructed and were based on, and ``Twenty-Six Gas Stations''
by Ed Ruscha as examples of the "sociological mapping...implicit in some
conceptual art" (185). More recently, and more ironically and more
internationally, there is Komar \& Melamid's "The People's Choice"
({http://www.diacenter.org/km/}) which uses official polling agencies to
survey a nation's preferences in visual art based on approximately ten
preferences (ranging from "Favorite color" to "Prefer indoor or outdoor
scenes") and then realizes the "most wanted painting" and "most unwanted
painting" utilizing the information. While this project is lightly
politicized around issues of taste and in its adaptation of the current
political tendency to base policies on poll results, it is linked to
current site-specific art practices which, as Kwon notes, "routinely
engage the collaborative participation of audience groups for the
conceptualization and production of the work.\ldots{}" From this
collaboration, these site-oriented works "are seen as a means to
strengthen art's capacity to penetrate the sociopolitical organization
of contemporary life with greater impact and meaning. In this sense the
possibilities to conceive of the site as something more than place---as
repressed ethnic history, a political cause, a disenfranchised social
group---is a crucial conceptual step in redefining the public role of
art and artists"(96). A recent example---and there are many---of this
laying bare of the historical determinants of place is realized in Stan
Douglas' Nu*tka* which presents a "Canadian Gothic" of late-nineteenth
century Nootka Sound on the Northwest Coast of B.C. through interlacing
video images of the area and disembodied voices of the Spanish and
English colonizers.

This chronotopic imagining of place as the site of a repressed racial
(ethnic, class and gendered) history has been an obvious project of
literature. In Canadian literature, this project was determined both
discursively and historically. Discursively, the embarrassingly narrow
yet dominant critical trope assigned to the national literature the role
of providing a history to a country strategically defined as having
none. This necessitated a "return of the repressed" in literature to
counter the dominant literary (national) historical projects.
Historically, Canada has imagined itself as bicultural and this
framework worked to suppress the histories of groups other than the
French and English. Within this very generalized framework, official and
aestheticized responses to this historical repression have emerged.
Small-town history chronicles that celebrate a town such as
\textbf{Morden in Manitoba}, which can be bought at City Hall (here in
Austria these projects are very similar and are called Heimatbuch
{[}roughly, "homeland book"{]}) or a book such as Andreas Schroeder's
\textbf{The Mennonites} which provides, in a coffeetable book format, a
history of the Mennonites in Canada. More well known, and with a larger
cultural impact, Joy Kogawa's novel \textbf{Obasan} brought forward the
history of the Japanese-Canadian internment. Yet the moment that
novelists are taken as historians is as problematic as when artists are
believed to be sociologists or social workers. For instance, Michael
Ondaatje's \textbf{In the Skin of a Lion} aestheticizes the history of
working-class people in Toronto, as if the workers appreciated the
modernist aesthetics of the worksite and the solidarity of shared labor
rather than complained about the relations of production and wages.

That these textual and visual models present place and site as being a
manifestation of history can lead to a deflecting of the present. For
instance, Roy Miki speculates that \textbf{Obasan} could, from one
vantage point, "become an object of knowledge as a Canadianized text
that teaches us about racism in our past" (\textbf{Broken Entries} 143).
The implication is that racism is relegated, chronotopically, to the
past. Douglas' \textbf{Nu*tka*}, in its return of repressed history,
could mimic the trope of First Nations' culture as a relic of the past,
as being only determined by European actions, as the passive site where
the history of dominant culture is acted out.

Here I want to speculate on (and politicize in a different way), Kwon's
idea of place as a "discursive vector" as a means to situate oneself
within a present site. I write "politicize" because it is possible to
propose a move from text to discourse as a movement from a static
structure open to analysis (whether virtuoso or standard) to a
constructing determinant of place and subjectivity. Here place/site
would not solely be determined by histories (dominant or repressed:
emergent or residual), but by competing constitutive discourses that
both affirm and erode the local/national and the everyday in the name of
the global. Without solely reducing globalization to the effect of a
discourse, it is possible to perhaps clarify the effects of
globalization by understanding them as being discursively enacted at one
level, as having a constitutive effect. Discursive is not synonymous
with "fluid," "ungrounded," as Kwon proposes, but is precisely
constitutive and grounding. In this formulation, the particularisms of a
place/site and its histories are not just the oppositional force to
globalization (or the corrective to dominant historical narratives), but
an aspect of place that can be utilized by globalism; that is, a
dialectic of local and global, or site and nonsite (if a place is
imagined as siteless in its loss of particularities due to
globalization).

My own turn toward the discursive effects of globalism in relation to
place/site arose because of the contradictions I saw in the city in
which I was living. The city of Calgary in Alberta imagines itself as a
regionally-based "open-for-business" kind of city, free of, but also
wary of, the ills of larger cities, yet a city of international
standards. This discourse of regionalism serves to cloak the existing
relations of production which Calgary---as an "oil town"---is linked
into, and determined by. Global capital through the multinational oil
and gas industry whose corporate logos hover above Calgary's gridded
streets is a key determinant of both the social relations of the city
and how the city imagines itself. When the oil industry is profitable
(due to high crude prices or a manufactured crisis), the city thrives
and new homes are built, rents rise, corporate headquarters shuffle,
cigar bars and fusion-cooking restaurants open---in short, a lot of
money is made by a few people but the scramble for profit is on. This
imaginary masking of real relations, a kind of cognitive masking, is
what Arjun Appadurai cites as fetishistic: "The locality (both in the
sense of the local factory or site of production and then extended sense
of the nation-state), becomes a fetish that disguises the dispersed
forces that actually drive the production process" (\textbf{Modernity At
Large}, 42). In Calgary, the global forces then have a direct effect on
the planning and layout of the city, as well as the architecture; the
social space of the city (itself a constituting aspect), is determined
by the management of the effects of globalization and localization.
Seeing how the effective discourses of place/region and nation could
serve as a mystifying factor, blocking the real relations of the city, I
moved away from an investigation of place and toward the discursive
determination of a place/site.

As a writer, the emphasis on the constitutive discourses of site/place
can supplement site as repressed histories. These discourses must also
be seen, alongside repressed histories, as historical developments, as
constitutive elements of the repressed history and emergent history. The
site of writing then becomes imagined as a place of intersecting
discourses: not groundless and fluid but both determined and
determining.

\hypertarget{long-note-on-new-narrative}{%
\section{Long Note on New Narrative}\label{long-note-on-new-narrative}}

Robert Glück

To talk about the beginnings of New Narrative, I have to talk about my
friendship with Bruce Boone. We met in the early seventies through the
San Francisco Art Institute's bulletin board: Ed and I wanted to move
and Bruce and Burton wanted to move---would we all be happy living
together? For some reason both couples dropped the idea and remained in
our respective flats for many years. But Bruce and I were poets and our
obsession with Frank O'Hara forged a bond.

I was twenty-three or twenty-four. Bruce was seven years older. He was a
wonderful teacher. He read to transform himself and to attain a correct
understanding. Such understanding was urgently political.

Bruce had his eye on the catastrophic future, an upheaval he predicted
with a certain grandeur, but it was my own present he helped me find. I
read and wrote to invoke what seemed impossible-\/-relation itself-\/-in
order to take part in a world that ceaselessly makes itself up, to "wake
up" to the world, to recognize the world, to be convinced that the world
exists, to take revenge on the world for not existing.

To talk about New Narrative, I also have to talk about Language Poetry,
which was in its heroic period in the seventies. I treat diverse poets
as one unit, a sort of flying wedge, because that's how we experienced
them. It would be hard to overestimate the drama they brought to a Bay
Area scene that limped through the seventies, with the powerful
exception of feminist poets like Judy Grahn, and the excitement
generated by movement poetry. Language Poetry's Puritan rigor, delight
in technical vocabularies, and professionalism were new to a generation
of Bay Area poets whose influences included the Beats, Robert Duncan and
Jack Spicer, the New York School (Bolinas was its western outpost),
surrealism and psychedelic surrealism.

Suddenly people took sides, though at times these confrontations
resembled a pastiche of the embattled positions of earlier avant-guards.
Language Poetry seemed very "straight male"---though what didn't?
Barrett Watten's \textbf{Total Syntax}, for example, brilliantly
established (as it dispatched) a lineage of fathers: Olson, Zukofsky,
Pound, etc.

If I could have become a Language poet I would have; I craved the
formalist fireworks, a purity that invented its own tenets. On the snowy
mountain-top of progressive formalism, from the highest high road of
modernist achievement, there was plenty of contempt heaped on less
rigorous endeavor. I had come to a dead end in the mid-seventies like
the poetry scene itself. The problem was not theoretical---or it was: I
could not go on until I figured out some way to understand where I was.
I also craved the community the Language Poets made for themselves.

The questions vexing Bruce and me and the kind of rigor we needed were
only partly addressed by Language Poetry which, in the most general
sense, we saw as an aesthetics built on an examination (by subtraction:
of voice, of continuity) of the ways language generates meaning. The
same could be said of other experimental work, especially the
minimalisms, but Language Poetry was our proximate example.

Warring camps and battle lines drawn between representation and
non-representation---retrospection makes the argument seem as arbitrary
as Fancy vs.~Imagination. But certainly the "logic of history" at that
moment supported the idea of this division, along with the struggle to
find a third position that would encompass the whole argument.

I experienced the poetry of disjunction as a luxurious idealism in which
the speaking subject rejects the confines of representation and
disappears in the largest freedom, that of language itself. My
attraction to this freedom and to the professionalism with which it was
purveyed made for a kind of class struggle within myself. Whole areas of
my experience, especially gay experience, were not admitted to this
utopia, partly because the mainstream reflected a resoundingly coherent
image of myself back to me---an image so unjust that it amounted to a
tyranny that I could not turn my back on. We had been disastrously
described by the mainstream---a naming whose most extreme (though not
uncommon) expression was physical violence. Political agency involved at
least a provisionally stable identity.

Meanwhile, gay identity was also in its heroic period---it had not yet
settled into just another nationalism and it was new enough to know its
own constructedness. In the urban mix, some great experiment was
actually taking place, a genuine community where strangers and different
classes and ethnicities rubbed more than shoulders. This community was
not destroyed by commodity culture, which was destroying so many other
communities; instead, it was founded in commodity culture. We had to
talk about it. Bruce and I turned to each other to see if we could come
up with a better representation---not in order to satisfy movement
pieties or to be political, but in order to be. We (eventually we were
gay, lesbian, and working class writers) could not let narration go.

Since I'm confined to hindsight, I write as though Bruce and I were
following a plan instead of stumbling and groping toward a writing that
could join other literatures of the present. We could have found
narrative models in, say, Clark Coolidge's prose, so perhaps narrative
practice relates outward to the actual community whose story is being
told. We could have located self-reference and awareness of artifice in,
say, the novels of Ronald Firbank, but we didn't. So again, our use of
language that knows itself relates outward to a community speaking to
itself dissonantly.

We were fellow travelers of Language Poetry and the innovative feminist
poetry of that time, but our lives and reading lead us toward a hybrid
aesthetic, something impure. We (say, Bruce Boone, Camille Roy, Kevin
Killian, Dodie Bellamy, Mike Amnasan, myself and, to include the dead,
Steve Abbott and Sam D'Allesandro) are still fellow travelers of the
poetries that evolved since the early eighties, when writers talked
about "nonnarrative." One could untangle that knot forever, or build an
aesthetics on the ways language conveys silence, chaos, undifferentiated
existence, and erects countless horizons of meaning.

How to be a theory-based writer?---one question. How to represent my
experience as a gay man?---another question just as pressing. These
questions lead to readers and communities almost completely ignorant of
each other. Too fragmented for a gay audience? Too much sex and "voice"
for a literary audience? I embodied these incommensurates so I had to
ask this question: How can I convey urgent social meanings while opening
or subverting the possibilities of meaning itself? That question has
deviled and vexed Bay Area writing for twenty-five years. What kind of
representation least deforms its subject? Can language be aware of
itself (as object, as system, as commodity, as abstraction) yet take
part in the forces that generate the present? Where in writing does
engagement become authentic? One response, the politics of form,
apparently does not answer the question completely.

One afternoon in 1976, Bruce remarked on the questions to the reader I'd
been throwing into poems and stories. They were self-consciously
theatrical and they seemed to him to pressure and even sometimes to
reverse the positions of reader and writer. Reader/writer dynamics
seemed like a way into the problems that preoccupied us, a toe in the
water!

From our poems and stories, Bruce abstracted text-metatext: a story
keeps a running commentary on itself from the present. The commentary,
taking the form of a meditation or a second story, supplies a succession
of frames. That is, the more you fragment a story, the more it becomes
an example of narration itself-\/-narration displaying its
devices-\/-while at the same time (as I wrote in 1981) the metatext
"asks questions, asks for critical response, makes claims on the reader,
elicits comments. In any case, text-metatext takes its form from the
dialectical cleft between real life and life as it wants to be."
{[}\^{}1{]}

We did not want to break the back of representation or to "punish" it
for lying, but to elaborate narration on as many different planes as we
could, which seemed consistent with the lives we lead. Writing can't
will away power relations and commodity life; instead, writing must
accept its relation to power and recognize that at present group
practice resides inside the commodity. Bruce wrote, "When evaluating
image in American culture, isn't it a commodity whether anyone likes it
or not? You make your additions and subtractions from that point
on."{[}\^{}2{]}

In 1978, Bruce and I launched the Black Star Series and published my
\textbf{Family Poems} and his \textbf{My Walk With Bob}, a lovely
book.({[}\^{}3{]}) In "Remarks on Narrative"---the afterword of
\textbf{Family Poems}---Bruce wrote, "As has now been apparent for some
time, the poetry of the '70s seems generally to have reached a point of
stagnation, increasing a kind of refinement of technique and available
forms, without yet being able to profit greatly from the vigor, energy
and accessibility that mark so much of the new Movement writing of gays,
women and Third World writers, among others. Ultimately this impasse of
poetry reflects conditions in society itself."( {[}\^{}4{]})

We appreciated the comedy of mounting an offensive ("A critique of the
new trends toward conceptualization, linguistic abstraction and process
poetry") with those slenderest volumes. My poems and stories were set
"in the family," not so antipsychological as they might have been given
that we assumed any blow to interiority was a step forward for mankind.

We contended with the Language Poets while seeking their attention in
the forums they erected for themselves. We published articles in
\textbf{Poetics Journal} and \textbf{L=A=N=G=U=A=G=E}, and spoke in talk
series and forums---a mere trickle in the torrent of their critical
work. If Language Poetry was a dead end, what a fertile dead end it was!

New Narrative was in place by the time \textbf{Elements of a Coffee
Service} was published by Donald Allen's Four Season's Foundation in
1982, and Hoddypole published Bruce's novel, \textbf{Century of Clouds}
in 1980. We were thinking about autobiography; by autobiography we meant
daydreams, nightdreams, the act of writing, the relationship to the
reader, the meeting of flesh and culture, the self as collaboration, the
self as disintegration, the gaps, inconsistences and distortions, the
enjambments of power, family, history and language.

Bruce and I brought high and low together between the covers of a book,
mingling essay, lyric, and story. Our publishing reflected those
different modes: stories from \textbf{Elements} appeared in gay
anthologies, porn magazines, \textbf{Social Text}, and \textbf{Soup}.
Bruce wrote about Georges Bataille for \textbf{The Advocate}\emph{. } I
wanted to write with a total continuity and total disjunction since I
experienced the world (and myself) as continuous and infinity divided.
That was my ambition for writing. Why should a work of literature be
organized by one pattern of engagement? Why should a "position" be
maintained regarding the size of the gaps between units of meaning? To
describe how the world is organized may be the same as organizing the
world. I wanted the pleasures and politics of the fragment and the
pleasures and politics of story, gossip, fable and case history; the
randomness of chance and a sense of inevitability; sincerity while using
appropriation and pastiche. When Barrett Watten said about \textbf{Jack
the Modernist}, "You have your cake and eat it too," I took it as a
great compliment, as if my intention spoke through the book.

During the seventies, Bruce was working on his doctorate at UC Berkeley.
His dissertation was a structuralist and gay reading of O'Hara, that is,
O'Hara and community, a version of which was published in the first
issue of \textbf{Social Text} in 1979. He joined the Marxism and Theory
Group at St.~Cloud which gave birth to that journal. Bruce also wrote
critical articles (especially tracking the "gay band" of the Berkeley
Renaissance).({[}\^{}4{]}) Bruce introduced me to most of the critics
who would make a foundation for New Narrative writing.

Here are a few of them:

Georg Lukacs: In \textbf{The Theory of the Novel}, Lukacs maintains that
the novel contains---that is, holds together---incommensurates. The epic
and novel are the community telling itself its story, a story whose
integration becomes increasingly hard to achieve. \textbf{The Theory of
the Novel} leads to ideas of collaboration and community that are not
naive, that is, to narrative that questions itself. It redistributes
relations of power and springs the writer from the box of psychology,
since he becomes the community speaking to itself. I wrote "Caricature,"
a talk given at 80 Langton in 1983, mostly using Lukacs' book, locating
instances of conservative and progressive communities speaking to
themselves: "If the community is a given, so are its types."

Louis Althusser: His essay, "Ideological State Apparatuses," refigures
the concept of base/superstructure. Terry Eagleton rang the following
change on Althusser's bulky formula: Ideology is the imaginary
resolution of real contradictions. By 1980, literary naturalism was
easily deprived of its transparency, but this formula also deprives all
fantasy of transparency, including the fantasy of personality. If making
a personality is not different from making a book, in both cases one
could favor the "real contradictions" side of the formula. If
personality is a fiction (a political fiction!) then it is a story in
common with other stories---it occurs on the same plane of experience.
This "formula" sets those opacities---a novel, a personality---as equals
on the stage of history, and supports a new version of autobiography in
which "fact" and "fiction" inter-penetrate.

Althusser comes with a lot of baggage. For example, he divided science
from ideology, and ideology from theory. Frankly, Bruce and I pillaged
critical theory for concepts that gave us access to our experience. In
retrospect, it might be better simply to "go with" cultural studies. To
the endless chain of equal cultural manifestations (a song by REM, the
Diet of Worms, Rousseau's \textbf{Confessions}), we add another equal
sign, attaching the self as yet another thing the culture "dreamed up."

Georges Bataille: Bataille was central to our project. He finds a
counter-economy of rupture and excess that includes art, sex, war,
religious sacrifice, sports events, ruptured subjectivity, the
dissolution of bodily integuments---"expenditure" of all kinds. Bataille
showed us how a bath house and a church could fulfill the same function
in their respective communities.

In writing about sex, desire and the body, New Narrative approached
performance art, where self is put at risk by naming names, becoming
naked, making the irreversible happen---the book becomes social practice
that is lived. The theme of obsessive romance did double duty,
de-stabling the self and asserting gay experience. Steve Abbott wrote,
"Gay writers Bruce Boone and Robert Glück (like Acker, Dennis Cooper or
the subway graffitists again) up the ante on this factuality by weaving
their own names, and those of friends and lovers, into their work. The
writer/artist becomes exposed and vulnerable: you risk being foolish,
mean-spirited, wrong. But if the writer's life is more open to judgement
and speculation, so is the reader's." ({[}\^{}6{]})

Did we believe in the "truth and freedom" of sex? Certainly we were
attracted to scandal and shame, where there is so much information. I
wanted to write close to the body---the place language goes reluctantly.
We used porn, where information saturates narrative, to expose and
manipulate genre's formulas and dramatis personae, to arrive at ecstacy
and loss of narration as the self sheds its social identities. We wanted
to speak about subject/master and object/slave. Bataille showed us that
loss of self and attainment of nothingness is a group activity. He
supplied the essential negative, a zero planted in the midst of
community. His concept of transgression gave us lots of fuel, as did his
novels of philosophic pornography.

Now I'd add that transgressive writing is not necessarily about sex or
the body---or about anything one can predict. There's no manual;
transgressive writing shocks by articulating the present, the one thing
impossible to put into words, because a language does not yet exist to
describe the present. Bruce translated Bataille's \textbf{Guilty} for
Lapis Press when I worked as an editor there. We hammered out the
manuscript together, absorbing Bataille gesturally.

Five more critics. Walter Benjamin: for lyrical melancholy (which reads
as autobiography) and for permission to mix high and low. V.N.
Voloshinov: for discovering that meaning resides within its social
situation, and that contending powers struggle within language itself.
Rolland Barthes: for a style that goes back to autobiography, for the
fragment, and for displaying the constructed nature of story---"baring
the device." Michel Foucault: for the constructed nature of sexuality,
the self as collaboration, and the not-to-be-underestimated example of
an out gay critic. (Once at 18th and Castro, Michel pierced Bruce with
his eagle gaze and Bruce was overcome!---he says.) Julia Kristeva: for
elaborating the meaning of abjection in \textbf{Powers of Horror}.

Our interest in Dennis Cooper and Kathy Acker produced allegiances and
friendships with those writers. Kathy moved to San Francisco in the fall
of 1981; while getting settled she stayed with Denise Kastan, who lived
downstairs from me. Denise and I co-directed Small Press Traffic. Kathy
was at work on \textbf{Great Expectations}. In fact, Denise and I appear
in it; we are the whores Danella and Barbraella. Kathy's writing gave
Bruce, Steve Abbott and myself another model, evolved far beyond our own
efforts, for the interrogation of autobiography as "text" perpetually
subverted by another text. Appropriation puts in question the place of
the writer---in fact, it turns the writer into a reader.

Meanwhile, Bruce and I were thinking about the painters who were
rediscovering the figure, like Eric Fischl and Julian Schnabel. They
found a figuration that had passed through the flame of abstract
expressionism and the subsequent isms, operating through them. It made
us feel we were part of a crosscultural impulse rather than a local
subset. Bruce wrote, "With much gay writing and some punk notoriously
(Acker the big example), the sexual roots of aggression come into
question. There's a scream of connection, the figure that emerges
ghostly: life attributed to those who have gone beyond. So in Dennis
Cooper's \textbf{Safe} there's a feeling-tone like a Schnabel
painting:the ground's these fragments of some past, the stag, the Roman
column, whatever---on them a figure that doesn't quite exist but would
maybe like to. The person/persona/thing the writer's trying to construct
from images---"({[}\^{}7{]})

In 1976, I started volunteering in the non-profit bookstore Small Press
Traffic and became co-director not long after. From 1977 to 1985, I ran
a reading series and held free walk-in writing workshops at the store.
The workshops became a kind of New Narrative laboratory attended by
Michael Amnasan, Steve Abbott, Sam D'Allesandro, Kevin Killian, Dodie
Bellamy, Camille Roy, and many other writers whose works extend my own
horizon. I would start by reading some piece of writing that interested
me: Chaucer, Robert Smithson, Lydia Davis, Ivan Bunin, Jim Thompson, a
book of London street games, Thomas Wyatt, Sei Shonagon. We were
aspiring to an ideal of learning derived as much from Spicer and Duncan
as from our contemporaries.

Most writers we knew were reading theory. Later, guided by Bruce, we
started a left reading group at Small Press Traffic, attended by Steve
Benson, Ron Silliman, Denise Kastan, Steve Abbott, Bruce, myself and
others. The personal demolished the political, and after a few months we
disbanded. From that era I recall Ron's epithet (which Bruce and I
thought delicious) \emph{The Small Press Traffic School of
Dissimulation}.

More successful was the Left/Write Conference we mounted in 1981 at the
Noe Valley Ministry. The idea for a conference was conceived in the
spring of 1998 by Bruce and Steve Abbott, who sent letters to thirty
writers of various ethnicities and aesthetic positions. Steve was a
tireless community builder, and Left/Write was an expression of New
Narrative's desire to bring communities together, a desire which
informed the reading series at Small Press Traffic, Steve Abbott's
\textbf{Soup} (where the term New Narrative first appeared), Michael
Amnasan's \textbf{Ottotole}, Camille Roy and Nayland Blake's
\textbf{Dear World}, Kevin Killian and Brian Monte's \textbf{No
Apologies}, and later Kevin and Dodie Bellamy's \textbf{Mirage}. We felt
urgent about it, perhaps because we each belonged to such disparate
groups. To our astonishment, three hundred people attended Left/Write,
so we accomplished on a civic stage what we were attempting in our
writing, editing and curating: to mix groups and modes of discourse.
Writers famous inside their own group and hardly known outside, like
Judy Grahn and Erica Hunt, spoke and read together for the first time.

Out of that conference the Left Writers Union emerged; soon it was
commandeered by its most unreconstructed faction which prioritized gay
and feminist issues out of existence. At one meeting, we were instructed
to hold readings in storefronts on ground level so the "masses of San
Francisco" could walk in!

During this decade-\/-1975-1985-\/-Bruce and I carried on what amounted
to one long gabby phone conversation. We brought gossip and anecdote to
our writing because they contain speaker and audience, establish the
parameters of community and trumpet their"unfair" points of view. I
hardly ever "made things up," a plot still seems exotic, but as a
collagist I had an infinite field. I could use the lives we endlessly
described to each other as "found material" which complicates
storytelling because the material also exists on the same plane as the
reader's life. Found materials have a kind of radiance, the truth of the
already-known.

In 1981 we published \textbf{La Fontaine} ** as a valentine to our
friendship. In one poem, Bruce (and Montaigne!) wrote, "In the
friendship whereof I speak...our souls mingle and blend in a fusion so
complete that the seam that joins them disappears and is found no more.
If pressed to say why I loved him I'd reply, because it was him, because
it was me."({[}\^{}9{]})

In using the tag New Narrative, I concede there is such a thing. In the
past I was reluctant to promote a literary school that endured even ten
minutes, much less a few years. Bruce and I took the notion of a
"school" half seriously, and once New Narrative began to resemble a
program, we abandoned it, declining to recognize ourselves in the
tyrants and functionaries that make a literary school. Or was it just a
failure of nerve? Now I am glad to see the term used by younger writers
in San Francisco, writers in other cities, like Gail Scott in Montreal,
and critics like Earl Jackson, Jr., Antony Easthope, Carolyn Dinshaw,
and Dianne Chisholm. Bruce and I may have been kidding about founding a
school, but we were serious about wanting to bring emotion and subject
matter into the field of innovative writing. I hope that these thoughts
on our project---call it what you will---are useful to those looking for
ways to extending the possibilities of poem and story without
backtracking into the mainstream, or into 19th-century transparency.

\hypertarget{notes}{%
\subsection{Notes}\label{notes}}

{[}\^{}1{]}. Robert Gluck, "Caricature," \textbf{Soup: New Critical
Perspectives} \#4, ed.~Bruce Boone (San Francisco, 1985), 28.
{[}\^{}2{]}. Bruce Boone, "A Narrative Like a Punk Picture: Shocking
Pinks, Lavenders, Magentas, Sickly Greens," \textbf{Poetics Journal} \#5
ed. Barrett Watten and Lyn Hejinian, (Berkeley, May 1985) 92.
{[}\^{}3{]}. Black Star published \textbf{He Cried} by Dennis Cooper and
\textbf{Lives of the Poets} by Steve Abbott. The Black Star Series still
publishes, most recently Camille Roy's \textbf{Swarm}, and soon John
Norton's \textbf{Re:marriage}. {[}\^{}4{]}: Bruce Boone, "Remarks on
Narrative," afterward of \textbf{Family Poems} by Robert Gluck, Black
Star Series, San Francisco, 1979) 29. {[}\^{}5{]}. "Spicer's Writing in
Context," \textbf{Ironwood} 28, Tuscon, 1986. "Robert Duncan \& Gay
Community," \textbf{Ironwood} 22, (Robert Duncan Special Issue, Tuscon
1983. Bruce's studies have led him to eastern religion---now he's a
non-denominational minister specializing in caring for people who are
terminally ill. {[}\^{}6{]}: Gluck, "Caricature," 19. {[}\^{}7{]}: Steve
Abbott, "Notes on Boundaries, New Narrative," \textbf{Soup: New Critical
Perspectives} \#4) 81. {[}\^{}8{]}: Boone, "A Narrative Like a Punk
Picture," \textbf{Poetics Journal} \#5, 92. {[}\^{}9{]}: La Fontaine,
"Perukes" (Black Star Series, 1981) 63.

~

~

\hypertarget{notes-toward-essaying-i-narratively}{%
\section{Notes Toward (Es)saying "I",
Narratively}\label{notes-toward-essaying-i-narratively}}

Rob Halpern

*I was already being prepared for {[}the world's{]} tournaments by a\\
training\\
which taught me to have a horror of faulty grammar instead of\\
teaching me,\\
when I committed these faults, not to envy others who avoided them.*\\
~\\
\hspace*{0.333em} ~~~~~~~~~~~--Saint Augustine, Bishop of Hippo

\textbf{Once upon a time} is always here. Crisis, my origin: this fault.
Someone is called where I am compelled to explain. To make narrative
critically: this response to crisis. Otherwise it's narrative by default
simply because there is no other way actively to be here.

~

\textbf{From the Sanskrit root} \emph{gna}, or know, our word narrative
derives bound to practice. In Latin, the root meaning appears to have
doubled, manifesting an otherwise latent critical difference. There is
at once \emph{(g)narus}: knowing, expert, skillful; and \emph{narro}: to
relate or tell. ({[}\^{}1{]}) A fault---mine, and ours---runs thru the
space between. \emph{Fault}: deficiency, lack, scarcity; as well as
slip, error, mistake. Also, more critically: \emph{fault} as dislocation
or a break in continuity of the strata or vein. \emph{Here}, knowing is
held in suspense and knowledge is always potential. The process of
telling/relating needn't connive with already determined knowledge;
narrative might, for example, radically deviate, turning away from, or
perverting, the very practice of authorization, questioning rather than
securing the position of the one who tells. Here, between the
overdetermined and the indeterminate, \emph{narrative determines}, that
is, it constrains and enables, just as it struggles and responds,
testing the tension between knowing and composing. \emph{Here, we make
our narratives ourselves, but, in the first place, under very definite
assumptions and conditions.} ({[}\^{}2{]})

::: \{align=``center''\}\\
\textbf{I. Poetically}\\
:::

\textbf{At once wayless and wayward}, I positions a self \emph{here} at
the site of narrative beginnings: place of mutability, place of many
turns. The crossroads of legend where every way is lost and every way is
open. Site of impossible possibility and paradox, where extremes
coincide. At once \emph{here}: where aversion keeps one in
place---\emph{I: this paralyzed fixture}---disinclined to turn another
way; and \emph{here}: away from which every turn is perversion, an
aberrant way away from the proper. \emph{Here} is forever shifting, the
site of all potential, site of suspense: where \emph{aporia} (knowing no
way, having no word, possessing no resource) and \emph{euporia} (knowing
so many ways, having everything to say, possessing it all in abundance)
converge. Here is where I intervenes in its own story, the permeable
interface of averse states---for example: impoverishment and
abundance---as if with \emph{two faces averse, and conjoined}. In the
space between states I spreads, this narrative surface: at once this
promise of disruption, this promise of continuity.

\textbf{Crisis} is the point at which old orders no longer cohere and
\emph{here} looms suddenly bereft of foundation. Here, for example, is
where Socrates steers his interlocutor: toward \emph{aporia}-place of
lost ways. But what if this crisis could be experienced
\emph{euporically}, as a moment of possibility precisely here where a
way appears impossible. Here occasions the paradoxical coincidence of
antithetical terms. Not the paralyzing absence of a way but the
accessibility of always other ways. Not naively, not as if everything
and anything were possible, but realistically, as if the site of
narrative were the only place where the impossible might open onto
possibility.

\emph{Here}: wayless, I is exposed; and \emph{here}, wayward, I resists.
It seems so familiar: experiencing the simultaneity of political
possibility and impossibility. The fullness of the imagination
paradoxically constrained and enabled by a politically invested
environment which threatens to render that fullness innocuous if not
impotent. How to breathe this air that animates such cynicism, how to
live in this element that triumphantly promotes irony as the "task of
the day". Somewhere between the pessimistic and the utopian, where these
extremes coincide, where pessimism reveals its utopian bent: here, one
might ask how to think the moment when "nothing is possible" becomes
"only the impossible is possible". And here another realism takes the
world for its measure and referent: here, where the limits of history,
the bleak horizon that pens us in, meet the pressures of the
imagination.

\textbf{Discrete identities and sovereign subjects}---like the kings and
saints of history---are what they are by virtue of narrative strength.
But I am more interested in narrative frailty, narrative faults: proxies
and prostheses, artificial extensions, these interventions in the world.
Rather than narratives saturated with ends known in advance of every
beginning, I am interested in narratives that destabilize the coherence
of the I which, like a promissory note or stand-in for some future
fulfillment, suffers this contractual obligation to self-identify over
time. The self-same I, untroubled by difference, might fill the space
between disruption and continuity, but it cannot fulfil it. The
self-same I: at once the coherent \emph{effect} of narrative and the
grammatical position to which one recursively returns to ensure that
coherence and whereby narrative possibility is filtered or censored to
preserve an ideal of intelligibility. Coherence, here, is but the
accomplishment of an optimal performance, a labored production of social
and linguistic grammars the faults of which are seamlessly concealed
\emph{so as no appearance of any wound or hurt outwardly might be at
once perceived}. But there are other possible effects, other ways,
veering toward away or versing waywardly. What if, for example, we were
able to avow I's fictitiousness without and affirm our necessary, if
only provisional, belief in it. Here, rather than persuading, narrative
might perform its agnostic potential: the promise of possible
impossibilities that holds sway in the ambiguous space between. This
refusal of certainty, this averse turn away from knowledge, this
perverse turn toward knowing: this is not a declaration of incoherence.
Between impoverishment and abundance (where impoverishment becomes
itself a figure for abundance) the I traverses narrative
orders---disrupting and maintaining---while it is simultaneously
traversed by them.

\textbf{I goes between: a love story.} In Plato's \emph{Symposium},
Diotima locates Love in the space between ignorance and wisdom.
Offspring of Poros (way, resource, expedient) and Penia (wayless,
poverty, lack.) Love paradoxically embodies this antinomy and cannot
give a reason for itself. According to Diotima, Love judges
\emph{without being able to give proper reasons}. "It is judging things
correctly without being able to give a reason," says Diotima, "surely,
you see that this is not the same as knowledge, for how can knowledge be
devoid of reason?" Jacques Lacan derives from this---in however faulty a
way---one of his characteristic figures for love: giving that which one
does not have to give. I am thinking of narrative like Love: wayless and
wayward, determined to do precisely what it cannot do. Its reasons are
but ruses. Between \emph{aporia} and \emph{euporia} Love is risked
agnostically, like a narrative that risks its own foundation. Sentenced
and committed, here, I goes between, exposed to every possibility of
rejection and loss, abjection and shame.

\textbf{Augustine and Stein:} Someone is called where I is compelled to
explain. "Well well is he. Explain my doubts, well well is he explain my
doubts.": this is how Gertrude Stein begins "Regularly Regularly In
Narrative" in \textbf{How to Write}. But the work narrative is called
upon to perform ("Explain my doubts") is alien to the work narrative
will ultimately do. Explain my doubts: at once the intended imperative
and the impossible end. My doubts will never be properly explained, my
uncertainty never properly reasoned away. Explanation fills the space
between ignorance and knowledge, but it cannot fulfil it. Recalling
another essay of Stein's, explanation is composition while composition
itself is figured as this "beginning again and again and again
explaining composition and time." And taking the perverse turn toward
the threshold of early modernity, there is Saint Augustine determined to
found the authority necessary to begin a book whose very goal it is to
authorize its subject. He, too, is concerned with beginning, explanation
and time about which he writes in the \textbf{Confessions}: "If no one
asks me, I know; if I want to explain it to someone who asks me, I don't
know." Between faith and doubt, past and future, knowing and
not-knowing, between 'he' (the partial referent to which past experience
is attached; it interferes with and informs writing's present) and 'you'
(a total referent---my audience, my God, the other I love---to which
"something more" that escapes representation is attached and toward
which narrative tends) between all these faulty binarisms Augustine
navigates a way to forge an 'I': the vehicle and the effect of an other
faith. Explain my doubts: the self-imposed imperative to narrate, to
compose a self. A crisis in knowing runs thru composing just as
narrative traces the fault along which I becomes to articulate or align
itself with the world \emph{in ways}. In the long afternoon shadows
thrown by the likes of Augustine and Stein, I am drawn to this, the
faultiness of narrative, its promise.

\textbf{To I this plot}

\hypertarget{section-12}{%
\subsection{:::}\label{section-12}}

\emph{Grant me Lord to know and understand whether a man is first to
pray to you for help or to praise you, and whether he must know you
before he can call you to his aid. If he does not know you, how can he
pray to you? For he may call for some other help, mistaking it for
yours.}

~~~~~~~~~~ --Saint Augustine

\begin{longtable}[]{@{}l@{}}
\toprule
\endhead
\begin{minipage}[t]{0.97\columnwidth}\raggedright
:::\strut
\end{minipage}\tabularnewline
\begin{minipage}[t]{0.97\columnwidth}\raggedright
The problem of beginning, the order of calling and knowing, is a
perennial preoccupation of both narrative and history. In our postmodern
era in which history has illusorily evanesced, narrative promises at
once to rescue the I from an indifferent individualism and to turn the
present away from indifference. Augustine's double bind is here: in
order to gain knowledge and understanding, one must call or praise; but
without knowledge and understanding one cannot know who or what it is
one calls. Mistaken audience is tantamount to mistaken identity. Who is
listening. Who is out there. Here, along the fault that runs between
audience and identity, narrative and history continue to produce each
other. Here, an I is "emplotted." Here, Augustine's address, his "how am
I able to speak to you?." the fragility of its beginning, awakens him to
the danger of being dispossessed of precisely that to which his I
professes to cling: communicative possibility. But who is the subject of
that possibility if the I who depends on past experience for recognition
is displaced, transfigured by the new experience of narrating a
conversion: the advent of an other I? I is this essay, this attempt,
this fragment, this experiment---like composition is explanation---this
reach transversing the pronominal distance between "Explain my doubts,
well, well is he explain my doubts" and "how am I to speak to you?." In
reaching out, I, discomfitted, doubles over the hinge of this plot and
risks a world. Between belief and doubt, this communication without
guarantees: the very condition of narrative possibility.\strut
\end{minipage}\tabularnewline
\begin{minipage}[t]{0.97\columnwidth}\raggedright
\textbf{A crisis of audience}\strut
\end{minipage}\tabularnewline
\begin{minipage}[t]{0.97\columnwidth}\raggedright
::: \{align=``center''\}\strut
\end{minipage}\tabularnewline
\begin{minipage}[t]{0.97\columnwidth}\raggedright
\textbar{} \emph{How shall I call upon my God for aid, when the call I
make is for \textbar{} my \textbar{} Lord and my God to come into
myself?... Does this then mean, O Lord \textbar{} my \textbar{} God,
that there is in me something fit to contain you? Can even \textbar{}
heaven \textbar{} and earth, which you made and in which you made me,
contain you? Or \textbar{} since nothing that exists could exist without
you, does this mean \textbar{} that \textbar{} whatever exists does, in
this sense, contain you? If this is so, \textbar{} since \textbar{} I
too exist, why do I ask you to come into me? For I should not be
\textbar{} there at all unless, in this way, you were already present
within \textbar{} me... \textbar{} But if I exist in you, how can I call
upon you to come to me? And \textbar{} where \textbar{} would you come
from? For you, my God, have said that you fill heaven \textbar{} and
earth, but I cannot go beyond heaven and earth so that you may
\textbar{} leave \textbar{} them to come to me.} \textbar{} \textbar{}
~~~~~~~~~~ --Saint Augustine \textbar{} \textbar{} \textbar{} \emph{So
then beside as any one can come to be certain of then if it is as
\textbar{} it \textbar{} is that is an audience is what it is what is it
if an audience is \textbar{} this, \textbar{} pretty soon then can feel
again that an audience is this, and then \textbar{} introspection can go
on but the habit of this thing makes it cease to \textbar{} be
\textbar{} this, because the audience and is it this keeps going
on...That is \textbar{} to \textbar{} say can does any one separate
themselves from the land so they can \textbar{} see \textbar{} it and if
they see it are they the audience of it or to it. If you \textbar{} see
\textbar{} anything are you its audience and if you tell anything are
you its \textbar{} audience, and is there any audience for it but the
audience that sees \textbar{} or \textbar{} hears it...And all this has
so much to do with writing a narrative \textbar{} of \textbar{} anything
that I can almost cry about it.} \textbar{} \textbar{} ~~~~~~~~~~
~--Gertrude Stein\strut
\end{minipage}\tabularnewline
\begin{minipage}[t]{0.97\columnwidth}\raggedright
:::\strut
\end{minipage}\tabularnewline
\begin{minipage}[t]{0.97\columnwidth}\raggedright
Bewildered by the question of containment, uncertain exactly which way
to turn, Augustine's move in relation to his audience is paradoxical.
Turning toward what "there is in me" he intimates, indeed creates, the
depth of a private self. Simultaneously he turns toward "heaven and
earth" revealing the self's dependence on exteriority. But is there an
essential difference between the two dimensions? Must one
\emph{aversion}---one turn away---achieve priority over and above the
another turn? Or might a mutable perversion---a turning aside from truth
or right, a diversion to improper use, a recognition of necessary
distortion, error, fault---intervene thus making coextensive interior
and exterior? Privative and excessive, I becomes this permeable surface,
this interface of in and out: this narrative tension between self and
other. Paradoxically, for Augustine, only a turn away from the
destabilizing question of audience, in other words, only silence or
not-writing, can ensure his possession of a self untroubled by spatial
location; only a turn away from either way will ensure comfort and allow
him to remain confident in the knowledge of such a self \emph{without
knowing it at all}. Self-knowledge as stasis annuls self-knowing as
narrative process. The very fact of Augustine's writing is testament to
his recognition that aversion can offer no way; it cannot ensure a
truthful I, only a proper self, that is, a property which must risk
itself in the act of narrating a story of origins, a return to the
fault. As soon as one begins to narrate oneself, one is adrift, exposed
to the hazard of contingency and doubt, relatively unmoored but bound to
the world nevertheless. In the face of political danger, I mobilizes
narratively: hazard becomes promise as I perverts a way away from
established norms. Looking back I still obeys, only paradoxically.
\emph{Here} crisis troubles the ground on which Augustine is authorized
to speak as "himself." Every word of address threatens to expose the
subject to the vertigo that looms when the secure ground of
"self-knowledge" gives way to groundlessness. And \emph{here} it is
precisely the move to explain my doubts, to provide for them a
foundation, that makes foundation impossible. Truthful I's are improper
selves, selves held in suspense, in potential and \emph{in situ} in the
space between. In the \textbf{Confessions}, this abyss, as it opens onto
narrative possibility, becomes paradoxically the very ground of love and
faith.\strut
\end{minipage}\tabularnewline
\begin{minipage}[t]{0.97\columnwidth}\raggedright
\textbf{Intentional fallacies}\strut
\end{minipage}\tabularnewline
\begin{minipage}[t]{0.97\columnwidth}\raggedright
::: \{align=``center''\}\strut
\end{minipage}\tabularnewline
\begin{minipage}[t]{0.97\columnwidth}\raggedright
\textbar{} \emph{They who write narrative and history do not do what
they say they \textbar{} will \textbar{} do when they start out to do
what they are about to do. \textbar{} } \textbar{} ~~~~~~~~~~
~--Gertrude Stein \textbar{} \textbar{} \emph{What the storyteller
narrates must necessarily be hidden from the \textbar{} actor \textbar{}
himself, at least as long as he is in the act or caught in its
\textbar{} consequences, because to him the meaningfulness of his act is
not in \textbar{} the \textbar{} story that follows.} \textbar{}
\textbar{} ~~~~~~~~~~ ~--Hannah Arendt\strut
\end{minipage}\tabularnewline
\begin{minipage}[t]{0.97\columnwidth}\raggedright
:::\strut
\end{minipage}\tabularnewline
\begin{minipage}[t]{0.97\columnwidth}\raggedright
Around every one of I's articulations, a story begins to congeal, a plot
thickens. Once upon a time, I stands in for this memory of the present,
a lapidary effect of all the stories I've ever told. Here is where I
unbecomes along this fault in the world, this cleavage or interruption
that opens onto narrative in the discomfiting space between where I
never fits. Here is where I encounters oblivion and possibility
simultaneously on the threshold of another sentence, another scene,
another world. "It was at the threshold of a world such as this that I
stood in peril as a boy" {[}Augustine{]}. But it is not, as in
Augustine, "my past foulness and carnal corruption of my soul" that
defines the narrative fault; rather the I itself---the risk and the
promise.\strut
\end{minipage}\tabularnewline
\begin{minipage}[t]{0.97\columnwidth}\raggedright
\textbf{"Something more"}\strut
\end{minipage}\tabularnewline
\begin{minipage}[t]{0.97\columnwidth}\raggedright
::: \{align=``center''\}\strut
\end{minipage}\tabularnewline
\begin{minipage}[t]{0.97\columnwidth}\raggedright
\textbar{} \emph{Do heaven and earth, then, contain the whole of you,
since you fill \textbar{} them? Or, when once you have filled them, is
some part of you left \textbar{} over because they are too small to hold
you? If this is so, when you \textbar{} have filled heaven and earth,
does that part of you which remains \textbar{} flow over into some other
place?} \textbar{} \textbar{} ~~~~~~~~~~ ~--Augustine\strut
\end{minipage}\tabularnewline
\begin{minipage}[t]{0.97\columnwidth}\raggedright
:::\strut
\end{minipage}\tabularnewline
\begin{minipage}[t]{0.97\columnwidth}\raggedright
\emph{That part of you which remains} cannot be represented, cannot be
assimilated to any way. You---audience, condition and limit of my
self---an unrepresentable horizon coincidently inside and out of me.
"For without you, what am I to myself but the leader of my own
destruction?" {[}Augustine{]}. Or, as Sandra Bernhardt titles her film:
\textbf{Without You I'm Nothing}. Her self composed thru an appeal to
the other. Turning toward 'you.' toward the invitation to explain, I
becomes, thru narration, \emph{in relation to}. Relation alone saves one
from destruction while paradoxically establishing the bond between 'you'
and 'I' which erotically threatens I's preservation as a discrete and
coherent self. Narrative's prosthetic origin: this other, who both is
and is not I. \emph{You may} be I's determining "last instance"---an
absolute referent modulated thru innumerable social circumstances that
determines 'I' in the final analysis. You, an Archimedean point doubled
and reflecting two horizons, in and out, creating this field of apparent
depth in the space between. You is equidistant from and coincident with
I's every utterance. In relation to you, I's meanings are measured. Thus
is I realistic without being real. Risking nothing short of love, the
voice aims at "a particular absolute... it aims beyond particular
objects to that 'something more' that exceeds them." ({[}\^{}3{]}) That
part of you that always remains.\strut
\end{minipage}\tabularnewline
\begin{minipage}[t]{0.97\columnwidth}\raggedright
::: \{align=``center''\}\strut
\end{minipage}\tabularnewline
\begin{minipage}[t]{0.97\columnwidth}\raggedright
\textbar{} ::: \{align=``center''\} \textbar{} \textbf{II. Politically}
\textbar{} :::\strut
\end{minipage}\tabularnewline
\begin{minipage}[t]{0.97\columnwidth}\raggedright
:::\strut
\end{minipage}\tabularnewline
\begin{minipage}[t]{0.97\columnwidth}\raggedright
\textbf{Narrative as counter-statement}. What is the difference between
being called to state (called to call a thing by its proper name) and
being called to \emph{the} state (called to recognize one's place within
the established order of things)? I ask this with the belief that
narrative has everything to offer the effort that would resist these
imperatives, enabling responses to always other calls. Called to state
and called to \emph{the} state: names, like bodies, are assimilated to
established narrative orders. But narrative needn't accede to the status
of statement, or doxa: rather, it is narrative as counter-statement, as
paradox, that enables one to refuse stately calls.\strut
\end{minipage}\tabularnewline
\begin{minipage}[t]{0.97\columnwidth}\raggedright
\textbf{Edward II, a digression.} The story of the English king Edward
the Second. In the sixteenth century, Holinshed, the popular chronicler
or historical narrator to whom both Marlowe and Shakespeare turned for
material, tells us that King Edward "began to hold the nobles in no
regard, to set nothing by their instructions, and to take small heed
unto the good government of the commonwealth, so that within a while, he
gave himself to wantonness, passing time in voluptuous pleasure, and
riotess excess: and to help them forward that kind of life, the foresaid
Peers who (as it may be thought, he had sworne to make the king \emph{to
forget himself, and the state to which he was called} {[}my italics{]})
furnished the court with jesters, ruffians, flattering parasites,
musicians and other vile and naughtie ribalds, that the king might spend
both days and nights in jesting, playing, blanketing and in such other
filthy and dishonourable excercises." {[}Chronicles of England, Scotland
and Ireland, 1587{]}. "The foresaid Peers" is none other than Gaveston,
the king's minion to whom Christopher Marlowe has the Edward of his
drama implore, "Knowest thou not who I am? Thy friend, thy self, another
Gaveston." To forget oneself and to forget the state to which one is
called are virtually synonymous offenses that render Edward a 'bad
subject' rather than a good sovereign. They are capital crimes serious
enough to land Edward in the tower, where he is sodomized to death with
a red hot iron spit. As Holinshed reports, the method was chosen "so as
no appearance of any wound or hurt outwardlie might be at once
perceived." Death appears woundlessly so that Edward can appear in
state. Were this wound to surface it would assume the figural status of
a counter-statement. Here, between life and death, continuity and
discontinuity; here, around this sign of struggle, a different narrative
would take shape paradoxically, this intervention in the orders of
state. Thus would the wound's articulation make it all "connectedly
different."\strut
\end{minipage}\tabularnewline
\begin{minipage}[t]{0.97\columnwidth}\raggedright
\textbf{Narrative difference}\strut
\end{minipage}\tabularnewline
\begin{minipage}[t]{0.97\columnwidth}\raggedright
::: \{align=``center''\}\strut
\end{minipage}\tabularnewline
\begin{minipage}[t]{0.97\columnwidth}\raggedright
\textbar{} \emph{When you consider the very long history of how everyone
ever acts or \textbar{} has \textbar{} felt, it is interesting that
nothing inside them in all of them makes \textbar{} it \textbar{}
connectedly different.} \textbar{} \textbar{} ~~~~~~~~~~ ~--Gertrude
Stein, "Composition as Explanation"\strut
\end{minipage}\tabularnewline
\begin{minipage}[t]{0.97\columnwidth}\raggedright
:::\strut
\end{minipage}\tabularnewline
\begin{minipage}[t]{0.97\columnwidth}\raggedright
From Augustine to Stein, narration has provided the way for
re-articulating the world in the space between, a space inassimilable to
already established orders, a space where discontinuity and continuity
together define a fault "outwardly." But what about politics? What about
poetry? And what about the division of intellectual labor that
traditionally separates the two? Marx's \textbf{Eighteenth Brumaire} is
often cited in this regard: "The social revolution of the nineteenth
century cannot draw its poetry from the past, but only from the future."
But what if the inquiry into the relation between narrative and "social
revolution" were to focus on neither past nor future but rather on this
space between? Right now, I am reading Stein's \textbf{Narration} beside
Louis Althusser's essay on "Ideology and Ideological State Apparatuses".
"The tenacious obviousness of the point of view of production alone, or
even that of mere productive practice are so integrated into our
everyday 'consciousness' that it is extremely hard, not to say almost
impossible, to raise oneself to the point of view of reproduction."
{[}Althusser{]}. In Stein's text, the question appears as follows: "Can
anyone separate themselves from the land so they can see it?" And, as if
in response to both questions, she writes "I am so certain so more than
certain that it ought to be done. I know so well all the causes why it
cannot be done and yet if it cannot be done cannot it be done it would
be so very much more interesting than anything if it could be done even
if it cannot be done." What is to be done is precisely what cannot be
done. Nevertheless, the imperative is sustained: to recognize the
unrecognizable, to do the undoable. Here, the narrative imperative, like
a Luddite wrench in the machinery of state, would prevent the apparatus
from reproducing itself. "To make a thing a thing that they recognize
while they are writing make it something that had no existing before
that writing gave it that recognition, they tried to do this by changing
something" {[}Stein{]}. This must be done. The point of view of
reproduction is the point of view of an antinomy defined by the apparent
impossibility of doing mutually exclusive things, like continuing and
discontinuing, localizing and sublimating. "Revolution" is a narrative
not at all unlike the narrative of conversion and it too is defined by
paradox according to which continuity with past experience must be
maintained in order to compose a discontinuous way \emph{here}. Stein
might be discoursing on ideology and Althusser, at great pains to
demonstrate how Marx succeeded precisely "by changing something." might
be writing about narration. And while it is dangerous to reduce politics
to literary theory---and this is not at all my intent---it is critical
to ask whether any such success at making it "connectedly different" is
conceivable without narrative? To make it all otherwise: a deceptively
simple imperative. To find one's way maplessly---aporically and
euporically---along the fault. To essay the unthinkable, to attempt to
do it all by other means: that is to practice narrative "as a socially
symbolic act." Fredric Jameson articulates this as a dialectical
imperative: "To reckon one's own position as an observer into the
critical thinking in process...one {[}then{]} no longer has to posit an
end to history in order for historical thought to take place."
{[}\textbf{Marxism and Form}{]}.\strut
\end{minipage}\tabularnewline
\begin{minipage}[t]{0.97\columnwidth}\raggedright
\textbf{Ideology \emph{is} a narrative practice}. Narrative situates us
historically in the world. Once upon a time opens onto this critical
scene of telling and distributes an historical \emph{here} equally thru
the space of narration. The scene is primal in the sense that it is
domestic; it frames a world within the world while promising a way to
encounter the frame. Narrative is also the vehicle of ideology: it
represents (in Althusser's formulation) one's imagined relations to real
conditions of existence; it is charged with the demands of history, the
production and reproduction of a world. And while narrative can also
intervene in the space between, narrators cannot separate themselves
from the land in order to see it. In other words, there is no outside in
relation to ideology, just as there is no outside of narrative. To deny
narrative, to resist its claim upon us, is not so much resisting
ideology as it is ideologically blind to the fact that without narrative
we are bereft of the means of counter-strategy in the face of dominant
and oppressive ideological orders. This is telling, this is our common
practice whereby the limits and truths of the world are negotiated. In
the making of history, every composition is the site of as many
decompositions. Rather than the solution to crisis, narrative is the
performance of it (this fault to which I return and from which I emerge
again and again); not a repetition of prefigured limitation, but a
renegotiation, an approach, a reaching out that defines the world. Here,
'I' appears as a provisional place-holder at the site where a subject is
called and assumed. The 'call' is a critical trope in Althusser's theory
of ideology where 'interpellation' is said to hail the subject into
social being. Accordingly, the ahistorical individual is called to
assume the historical position of a socially intelligible 'I'. But I
like also to think that, appearing as a third term between the
determined and the indeterminate, 'I' is equally the effect of
misrecognitions, incommensurable both with the name by which it is
called and the name I call myself. Mistaken audience, mistaken identity.
'I' holds the promise of some non-identical content: 'you' my audience
and 'he' my self. Narrative denaturalizes whatever 'I' and naturalizes
only the need to narrate. The space of narrative, then, is one of
intervention, conversion, transfiguration: where one renders oneself
other than oneself in a paradoxical quest for a more truthful 'I'.
Selves are called and recalled in turn. Augustine's ethic of conversion
and Edward II's ethic of friendship both articulate ways of potential
resistance to a prior call by way of a new response. Ethics are
ideological, but they allow one to narrate one's imagined relations to
real conditions differently, that is, to imagine it differently.\strut
\end{minipage}\tabularnewline
\begin{minipage}[t]{0.97\columnwidth}\raggedright
\textbf{Critical faith}\strut
\end{minipage}\tabularnewline
\begin{minipage}[t]{0.97\columnwidth}\raggedright
::: \{align=``center''\}\strut
\end{minipage}\tabularnewline
\begin{minipage}[t]{0.97\columnwidth}\raggedright
\textbar{} \emph{Only, how are they to call upon the lord until they
have learned to \textbar{} believe in him?} \textbar{} \textbar{}
~~~~~~~~~~ ~--Saint Augustine\strut
\end{minipage}\tabularnewline
\begin{minipage}[t]{0.97\columnwidth}\raggedright
:::\strut
\end{minipage}\tabularnewline
\begin{minipage}[t]{0.97\columnwidth}\raggedright
But how do we come to believe in the I that is called? Stein writes of
"...names being not existing because anybody can know what any body else
is talking about without any name being mentioning, without any belief
in the any name being existing." Without risking belief, communication
lies in state. And belief, situated on the other side of doubt and
uncertainty, must be born in the space between, the space of the call
itself. This recalls Althusser recalling Pascal writing (more or less)
that one gets down on one's knees and belief follows. But with a
critical difference: belief needn't refer to the uncritical reproduction
of a dominant call and response; it might rather refer to one
\emph{abandoning} oneself to a space between incommensurable calls.
Perhaps this would be the space of neither belief nor doubt but of a
more critical faith whose subject, whose 'I', is attentive to the
paradox: for what (or who) is the subject of \emph{abandon} if one's
self is the object? While belief is necessarily ideological, ideology
isn't unequivocally "bad": in fact, it is necessary. Like narrative,
ideology enables. Here again, the crisis, the plot: narrative's origin,
this doubling along the fault which I essays. Augustine abandons himself
to narrative and in doing so the stakes of communication are raised
without offering the security of a predictable outcome. If coherent
selfhood is only an effect of narration, and if narration is always a
potentially destabilizing force, then security is always already
preempted when one risks a beginning. This is both the promise of
narrative and its betrayal. And this is what makes Augustine's
\textbf{Confessions} an exquisite act of love, as political as it is
poetic. "If poetry is the calling upon a name until that name comes to
be anything if one goes on calling on that name more and more calling
upon that name as poetry does then poetry does make that calling upon a
name a narrative it is a narrative of calling upon that name."
{[}Stein{]} The poetic tradition, however, itself ideologically
informed, has connived with the market of names---authorizing it in
ways---a devaluing trade in which names are used and exchanged
calculably as guaranteed referents, their faults concealed. \emph{So as
no appearance of any wound or hurt might outwardly be at once
perceived.} Thus has the very need to call been obviated and the space
between, from which another way may be reached, occluded. I believe,
however, that in relation to crisis, it is the response-ability of
narrative to restore our awareness of the need, and intensify our desire
for another way.\strut
\end{minipage}\tabularnewline
\begin{minipage}[t]{0.97\columnwidth}\raggedright
And only by way of narrative might we unriddle "this knot of imaginary
servitude that love must always undo again." ({[}\^{}4{]})\strut
\end{minipage}\tabularnewline
\begin{minipage}[t]{0.97\columnwidth}\raggedright
\emph{Note:} Many thanks are due to Earl Jackson Jr.'s generous reading
of this piece and all the accompanying conversation.\strut
\end{minipage}\tabularnewline
\begin{minipage}[t]{0.97\columnwidth}\raggedright
\textbf{Notes}\strut
\end{minipage}\tabularnewline
\begin{minipage}[t]{0.97\columnwidth}\raggedright
{[}\^{}1{]}: Hayden White notes this etymology in \textbf{The Content of
the Form}\emph{.}\strut
\end{minipage}\tabularnewline
\begin{minipage}[t]{0.97\columnwidth}\raggedright
{[}\^{}2{]}: Frederich Engels writes in his letter to Joseph Block,
September 21, 1890: "We make our history ourselves, but, in the first
place under very definite assumptions and conditions." Glossing this
line in \textbf{Marxism and Literature}, Raymond Williams writes, "What
this restores, as against the alternative development, is the idea of
direct agency: 'we make our history ourselves.' The 'definite' or
'objective' assumptions and conditions are then the qualifying terms of
this agency: in fact 'determination' as 'the setting of limits'".\strut
\end{minipage}\tabularnewline
\begin{minipage}[t]{0.97\columnwidth}\raggedright
{[}\^{}3{]}: Copjec, Joan, \textbf{Read My Desire}, p.~148.\strut
\end{minipage}\tabularnewline
\begin{minipage}[t]{0.97\columnwidth}\raggedright
{[}\^{}4{]}: Lacan, "The Mirror Stage.\emph{"}\strut
\end{minipage}\tabularnewline
\begin{minipage}[t]{0.97\columnwidth}\raggedright
\#\# Bibliography\strut
\end{minipage}\tabularnewline
\begin{minipage}[t]{0.97\columnwidth}\raggedright
Althusser, Louis, \textbf{Lenin and Philosophy and Other Essays}.
London: New left Books, 1971.\strut
\end{minipage}\tabularnewline
\begin{minipage}[t]{0.97\columnwidth}\raggedright
Arendt, Hannah, \textbf{The Human Condition}. Chicago: University of
Chicago Press, 1958.\strut
\end{minipage}\tabularnewline
\begin{minipage}[t]{0.97\columnwidth}\raggedright
Augustine, \textbf{Confessions}. translated by R.S. Pine-Coffin, London:
Penguin Books, 1961.\strut
\end{minipage}\tabularnewline
\begin{minipage}[t]{0.97\columnwidth}\raggedright
Augustine, \textbf{The Confessions of Saint Augustine}. translated by
John K. Ryan. New York: Doubleday, 1960.\strut
\end{minipage}\tabularnewline
\begin{minipage}[t]{0.97\columnwidth}\raggedright
Copjec, Joan, \textbf{Read My Desire}. Boston: Massachusetts Institute
of Technology, 1995.\strut
\end{minipage}\tabularnewline
\begin{minipage}[t]{0.97\columnwidth}\raggedright
James, Fredric, \textbf{Marxism and Form}. Princeton, New Jersey:
Princeton University Press, 1971.\strut
\end{minipage}\tabularnewline
\begin{minipage}[t]{0.97\columnwidth}\raggedright
Stein, Gertrude, \textbf{Narration}. Chicago: The University of Chicago
Press, 1935.\strut
\end{minipage}\tabularnewline
\begin{minipage}[t]{0.97\columnwidth}\raggedright
White, Hayden, \textbf{The Content of the Form}. Baltimore: Johns
Hopkins University Press, 1987.\strut
\end{minipage}\tabularnewline
\begin{minipage}[t]{0.97\columnwidth}\raggedright
Williams, Raymond, \textbf{Marxism and Literature}, Oxford University
Press, 1977.\strut
\end{minipage}\tabularnewline
\begin{minipage}[t]{0.97\columnwidth}\raggedright
\textgreater{} Clark Coolidge, \textbf{The Crystal Text} (I
think)\emph{.} It seems likely I \textgreater{} might be alone in seeing
in these eight words, words ripped right out \textgreater{} of their
context, a possible model for narrative. \textgreater{} \textgreater{} ~
\textgreater{} \textgreater{} \href{hunt.html}{Back to The Avenue}\strut
\end{minipage}\tabularnewline
\begin{minipage}[t]{0.97\columnwidth}\raggedright
\textgreater{} For some years now Brenda Coultas, in her narratives, has
been \textgreater{} conjuring worlds to which multiple qualities of the
preceding \textgreater{} description might apply. Whether in her book
\textbf{Early Films} (Rodent \textgreater{} Press 1996) or in her recent
chapbook \textbf{A Summer Newsreel} (Second \textgreater{} Story Books,
1999) Coultas is engaged in constructing usefully \textgreater{}
disjunctive, lyric-enhanced investigations into landscapes that have
\textgreater{} come apart. For me, part of the great thing about
Coultas' writing is \textgreater{} that she brings an indomitable hunger
for beauty to bear on her \textgreater{} cracked and spilling subjects.
\textgreater{} \textgreater{} \href{hunt.html}{Back to The Avenue}\strut
\end{minipage}\tabularnewline
\begin{minipage}[t]{0.97\columnwidth}\raggedright
\textgreater{} E.g., Pamela Lu's terrific \textbf{Pamela: A Novel}
(Atelos, 1998). A \textgreater{} bildungsroman of sorts, an intricate
chronicle of the coming awake of \textgreater{} a highly-syntaxed
consciousness, in which the greater part of the \textgreater{} narrative
seems to occur in between commas. \textgreater{} \textgreater{} ~
\textgreater{} \textgreater{} \href{hunt.html}{Back to The Avenue}\strut
\end{minipage}\tabularnewline
\begin{minipage}[t]{0.97\columnwidth}\raggedright
\textgreater{} "The expression that there is nothing to express, nothing
with which \textgreater{} to express, nothing from which to express, no
power to express, no \textgreater{} desire to express, together with the
obligation to express." Samuel \textgreater{} Beckett (in 1949
conversation with Georges Duthuit). \textgreater{} \textgreater{}
\href{hunt.html}{Back to The Avenue}\strut
\end{minipage}\tabularnewline
\begin{minipage}[t]{0.97\columnwidth}\raggedright
\textgreater{} The tendered premise in a recent episode of the X-Files:
since World \textgreater{} War Two Americans have had an unquenchable
appetite for 'bogus \textgreater{} revelation'; having determined this,
the United States Government has \textgreater{} built, presumably as a
cover for its more radical weapons of mass \textgreater{} destruction
programs, an elaborate web of alien abduction and sighting
\textgreater{} hoaxes. This hoax, we are led to believe, may or may not
be one. At \textgreater{} the end, intrigued to a ghastly degree, we are
left both believing \textgreater{} (the elements of documentary inherent
in the X-Files aiding this \textgreater{} process) that there are aliens
around and that the Government has \textgreater{} cooked up (and
deployed) weapons even more diabolical than the ones it \textgreater{}
has copped to. This tv 'fiction' seems to me only slightly less
\textgreater{} credible than what the evening news, reporting live from
the White \textgreater{} House! serves up, or than the gamut of what can
be found gathered up \textgreater{} daily in \textbf{The New York
Times}. (The day after watching the X-Files \textgreater{} episode I
read allegations by Puerto Rican petitioners at the United
\textgreater{} Nations that the United States Government was testing
uranium-capped \textgreater{} bullets at its facility on Vieques Island;
during the debate an Iraqi \textgreater{} delegate stated that similar
weapons had been used during the Gulf \textgreater{} War.)
\textgreater{} \textgreater{} ~ \textgreater{} \textgreater{}
\href{hunt.html}{Back to The Avenue}\strut
\end{minipage}\tabularnewline
\begin{minipage}[t]{0.97\columnwidth}\raggedright
\textgreater{} Harry Mathews, who, in part of his work, has
vertiginously built upon \textgreater{} (see \textbf{The Conversions and
Tlooth}, recently reissued by Dalkey \textgreater{} Archive), the
wonderful, strange narratives of Raymond Roussel, had \textgreater{}
this to say in a recent interview in Rain Taxi: "The great discovery
\textgreater{} I made through him {[}Roussel{]} was that you didn't have
to write \textgreater{} about what happened the day you fell in the
swimming pool, but that \textgreater{} you could invent your own
swimming pool that had never existed before, \textgreater{} full of
quicksilver, lighted by imaginary lamps." Which, abstracted,
\textgreater{} might lead to a formula something like, The writer is in
the world; \textgreater{} the writing is in the writer. Or, looking at
it from a reader-oriented \textgreater{} perspective, we might arrive
at: The writing is in the reader; the \textgreater{} reader is in the
world; Either way we are dealing with a considerable \textgreater{}
degree of mediation. One I am interested in. \textgreater{}
\textgreater{} \href{hunt.html}{Back to The Avenue}\strut
\end{minipage}\tabularnewline
\begin{minipage}[t]{0.97\columnwidth}\raggedright
\textgreater{} "The more strictly the novel adheres to realism in
external things, \textgreater{} to the gesture that says 'this is how it
was,' the more every word \textgreater{} becomes a mere 'as if,' and the
greater becomes the contradiction \textgreater{} between this claim and
the fact that it was not so."\\
\textgreater{} \textgreater{} \textgreater{} --Adorno, "The Position of
the Narrator in the Contemporary Novel" \textgreater{} \textgreater{}
\href{hunt.html}{Back to The Avenue}\strut
\end{minipage}\tabularnewline
\begin{minipage}[t]{0.97\columnwidth}\raggedright
\textgreater{} Among contemporary works, the narratives of W.G. Sebald
(\textbf{The \textgreater{} Emigrants, The Rings of Saturn}) are
exemplary in this regard, \textgreater{} pointing up, as they do,
against the backdrop of this bloody century \textgreater{} (and with an
apercu of others), the enormous difficulty of fixing \textgreater{}
experience, both personal and collective. Keith Waldrop's harrowing
\textgreater{} memoir \textbf{Light While There Is Light} (which calls
itself fiction) is \textgreater{} another example. As is David Markson's
intriguing \textbf{Wittgenstein's \textgreater{} Ladder}. \textgreater{}
\textgreater{} ~ \textgreater{} \textgreater{} \href{hunt.html}{Back to
The Avenue}\strut
\end{minipage}\tabularnewline
\begin{minipage}[t]{0.97\columnwidth}\raggedright
\textgreater{} Ryonusuke Akutagawa brilliantly treats aspects of this in
a pair of \textgreater{} stories which were later combined and made into
the much more famous \textgreater{} film \textbf{Rashomon} by Akira
Kurosawa. Aaron Shurin writes, in passing, \textgreater{} in his 1990
essay "Narrativity," of the 'irreversible solidity of \textgreater{} the
past tense.' I've lately begun to marvel, because of this \textgreater{}
generally ascribed quality, that History continues to be written in
\textgreater{} it. It was Herodotus, Father of History, who kicked
things \textgreater{} off---neatly blending directly observed
incident/object with reported \textgreater{} incident/object with
rumored incident/object with imagined \textgreater{} incident/object. At
least a good part of what gets built into the past \textgreater{} tense,
then, is hardly irreversible, hardly solid. Pretty slippery, in
\textgreater{} fact. \textgreater{} \textgreater{} ~ \textgreater{}
\textgreater{} \href{hunt.html}{Back to The Avenue}\strut
\end{minipage}\tabularnewline
\begin{minipage}[t]{0.97\columnwidth}\raggedright
\textgreater{} Walter Benjamin writes beautifully on this in the opening
pages of his \textgreater{} essay, ``The Image of Proust''
(\textbf{Illuminations}): ``For the important \textgreater{} thing for
the remembering author is not what he experienced, but the
\textgreater{} weaving of his memory, the Penelope work of recollection.
Or should \textgreater{} one call it, rather, a Penelope work of
forgetting? Is not the \textgreater{} involuntary recollection, Proust's
memoire involuntaire, much closer \textgreater{} to forgetting than what
is usually called memory? And is not this work \textgreater{} of
spontaneous recollection, in which remembering is the woof and
\textgreater{} forgetting the warf, a counterpart to Penelope's work
rather than its \textgreater{} likeness? For here the day unravels what
the night has woven. When we \textgreater{} awake each morning, we hold
in our hands, usually weakly and loosely, \textgreater{} but a few
fringes of the tapestry of lived life, as loomed for us by
\textgreater{} forgetting. However, with our purposeful activity and,
even more, our \textgreater{} purposive remembering each day unravels
the web and the ornaments of \textgreater{} forgetting. This is why
Proust finally turned his days into nights, \textgreater{} devoting all
his hours to undisturbed work in his darkened room with \textgreater{}
artificial illumination, so that none of those intricate arabesques
\textgreater{} might escape him.'' \textgreater{} \textgreater{} ~
\textgreater{} \textgreater{} \href{hunt.html}{Back to The Avenue}\strut
\end{minipage}\tabularnewline
\begin{minipage}[t]{0.97\columnwidth}\raggedright
\# The Avenue\strut
\end{minipage}\tabularnewline
\begin{minipage}[t]{0.97\columnwidth}\raggedright
Laird Hunt\strut
\end{minipage}\tabularnewline
\begin{minipage}[t]{0.97\columnwidth}\raggedright
\emph{In my skull is an avenue I stroke}{[}\^{}huntA{]}\strut
\end{minipage}\tabularnewline
\begin{minipage}[t]{0.97\columnwidth}\raggedright
\emph{Clark Coolidge}\strut
\end{minipage}\tabularnewline
\begin{minipage}[t]{0.97\columnwidth}\raggedright
~\strut
\end{minipage}\tabularnewline
\begin{minipage}[t]{0.97\columnwidth}\raggedright
There is an avenue in my skull too---albeit one that is poorly
maintained, cavernously pitted, strewn with rubble, whole segments
blasted away; one that is curved, possibly circular, that, like the
backgrounds in cartoons, maddeningly repeats itself; one that is
ill-marked, with many a false turn-off and many a false vista; one that
is skewed out of proportion, that is frequently unsafe, almost always
unsavory; one that is troubled by converging lanes, of which there are
hundreds, that even resembles a parking lot in places and is probably
haunted -\/- \emph{hell of a place}{[}\^{}huntB{]}. And yet, {[}I,
narrator{]}{[}\^{}huntC{]}, stroke it, speak through its mess, speak of
its mess, multiply it. It's an interesting dilemma---aspects of which
are taken up in Adorno's seminal essay on the place of the narrator in
the contemporary novel -\/- the narrator, with no story to narrate,
\emph{narrates anyway}{[}\^{}huntD{]}, a story that has been blown to
bits. Of course not everyone sees it that way. We live in {[}an age of
errata, of misinformation, of disinformation, of hoax{]}{[}\^{}huntE{]};
perhaps it is little wonder that there continues to be such a hunger for
narratives that, as Adorno describes them, largely by way of 19th
century techniques \emph{mimic the real}{[}\^{}huntF{]}; that say to us,
with disarming earnestness, this, my friends, is \emph{how it
was}{[}\^{}huntG{]}. In the domain of the fictive narrative, I tend to
have little patience for such works. I am much more interested in (and
seem only capable of constructing) narratives that are to some degree
aware of \emph{the provisional nature of their own
authority}{[}\^{}huntH{]}, in which \emph{the fictive quality of
recollection}{[}\^{}huntI{]} is acknowledged, in which
\emph{forgetting}{[}\^{}huntJ{]} is considered the key constituent of
memory, and in which, finally, getting it right shares center stage with
getting it wrong.\strut
\end{minipage}\tabularnewline
\begin{minipage}[t]{0.97\columnwidth}\raggedright
~\strut
\end{minipage}\tabularnewline
\begin{minipage}[t]{0.97\columnwidth}\raggedright
~\strut
\end{minipage}\tabularnewline
\begin{minipage}[t]{0.97\columnwidth}\raggedright
{[}\^{}huntA{]}: Clark Coolidge, \textbf{The Crystal Text} (I think). It
seems likely I might be alone in seeing in these eight words, words
ripped right out of their context, a possible model for narrative.\strut
\end{minipage}\tabularnewline
\begin{minipage}[t]{0.97\columnwidth}\raggedright
{[}\^{}huntB{]}: For some years now Brenda Coultas, in her narratives,
has been conjuring worlds to which multiple qualities of the preceding
description might apply. Whether in her book \textbf{Early Films}
(Rodent Press 1996) or in her recent chapbook \textbf{A Summer Newsreel}
(Second Story Books, 1999) Coultas is engaged in constructing usefully
disjunctive, lyric-enhanced investigations into landscapes that have
come apart. For me, part of the great thing about Coultas' writing is
that she brings an indomitable hunger for beauty to bear on her cracked
and spilling subjects.\strut
\end{minipage}\tabularnewline
\begin{minipage}[t]{0.97\columnwidth}\raggedright
{[}\^{}huntC{]}: E.g., Pamela Lu's terrific \textbf{Pamela: A Novel}
(Atelos, 1998). A bildungsroman of sorts, an intricate chronicle of the
coming awake of a highly-syntaxed consciousness, in which the greater
part of the narrative seems to occur in between commas.\strut
\end{minipage}\tabularnewline
\begin{minipage}[t]{0.97\columnwidth}\raggedright
{[}\^{}huntD{]}: "The expression that there is nothing to express,
nothing with which to express, nothing from which to express, no power
to express, no desire to express, together with the obligation to
express." Samuel Beckett (in 1949 conversation with Georges
Duthuit).\strut
\end{minipage}\tabularnewline
\begin{minipage}[t]{0.97\columnwidth}\raggedright
{[}\^{}huntE{]}: The tendered premise in a recent episode of the
X-Files: since World War Two Americans have had an unquenchable appetite
for 'bogus revelation'; having determined this, the United States
Government has built, presumably as a cover for its more radical weapons
of mass destruction programs, an elaborate web of alien abduction and
sighting hoaxes. This hoax, we are led to believe, may or may not be
one. At the end, intrigued to a ghastly degree, we are left both
believing (the elements of documentary inherent in the X-Files aiding
this process) that there are aliens around and that the Government has
cooked up (and deployed) weapons even more diabolical than the ones it
has copped to. This tv 'fiction' seems to me only slightly less credible
than what the evening news, reporting live from the White House! serves
up, or than the gamut of what can be found gathered up daily in
\textbf{The New York Times}. (The day after watching the X-Files episode
I read allegations by Puerto Rican petitioners at the United Nations
that the United States Government was testing uranium-capped bullets at
its facility on Vieques Island; during the debate an Iraqi delegate
stated that similar weapons had been used during the Gulf War.)\strut
\end{minipage}\tabularnewline
\begin{minipage}[t]{0.97\columnwidth}\raggedright
{[}\^{}huntF{]}: Harry Mathews, who, in part of his work, has
vertiginously built upon (see \textbf{The Conversions and Tlooth},
recently reissued by Dalkey Archive), the wonderful, strange narratives
of Raymond Roussel, had this to say in a recent interview in Rain Taxi:
"The great discovery I made through him {[}Roussel{]} was that you
didn't have to write about what happened the day you fell in the
swimming pool, but that you could invent your own swimming pool that had
never existed before, full of quicksilver, lighted by imaginary lamps."
Which, abstracted, might lead to a formula something like, The writer is
in the world; the writing is in the writer. Or, looking at it from a
reader-oriented perspective, we might arrive at: The writing is in the
reader; the reader is in the world; Either way we are dealing with a
considerable degree of mediation. One I am interested in.\strut
\end{minipage}\tabularnewline
\begin{minipage}[t]{0.97\columnwidth}\raggedright
{[}\^{}huntG{]}: "The more strictly the novel adheres to realism in
external things, to the gesture that says 'this is how it was,' the more
every word becomes a mere 'as if,' and the greater becomes the
contradiction between this claim and the fact that it was not
so."~(Adorno, "The Position of the Narrator in the Contemporary
Novel")\strut
\end{minipage}\tabularnewline
\begin{minipage}[t]{0.97\columnwidth}\raggedright
{[}\^{}huntH{]}: Among contemporary works, the narratives of W.G. Sebald
(\textbf{The Emigrants, The Rings of Saturn}) are exemplary in this
regard, pointing up, as they do, against the backdrop of this bloody
century (and with an apercu of others), the enormous difficulty of
fixing experience, both personal and collective. Keith Waldrop's
harrowing memoir \textbf{Light While There Is Light} (which calls itself
fiction) is another example. As is David Markson's intriguing
\textbf{Wittgenstein's Ladder}.\strut
\end{minipage}\tabularnewline
\begin{minipage}[t]{0.97\columnwidth}\raggedright
{[}\^{}huntI{]}: Ryonusuke Akutagawa brilliantly treats aspects of this
in a pair of stories which were later combined and made into the much
more famous film \textbf{Rashomon} by Akira Kurosawa. Aaron Shurin
writes, in passing, in his 1990 essay "Narrativity," of the
'irreversible solidity of the past tense.' I've lately begun to marvel,
because of this generally ascribed quality, that History continues to be
written in it. It was Herodotus, Father of History, who kicked things
off---neatly blending directly observed incident/object with reported
incident/object with rumored incident/object with imagined
incident/object. At least a good part of what gets built into the past
tense, then, is hardly irreversible, hardly solid. Pretty slippery, in
fact.\strut
\end{minipage}\tabularnewline
\begin{minipage}[t]{0.97\columnwidth}\raggedright
{[}\^{}huntJ{]}: Walter Benjamin writes beautifully on this in the
opening pages of his essay, ``The Image of Proust''
(\textbf{Illuminations}): ``For the important thing for the remembering
author is not what he experienced, but the weaving of his memory, the
Penelope work of recollection. Or should one call it, rather, a Penelope
work of forgetting? Is not the involuntary recollection, Proust's
memoire involuntaire, much closer to forgetting than what is usually
called memory? And is not this work of spontaneous recollection, in
which remembering is the woof and forgetting the warf, a counterpart to
Penelope's work rather than its likeness? For here the day unravels what
the night has woven. When we awake each morning, we hold in our hands,
usually weakly and loosely, but a few fringes of the tapestry of lived
life, as loomed for us by forgetting. However, with our purposeful
activity and, even more, our purposive remembering each day unravels the
web and the ornaments of forgetting. This is why Proust finally turned
his days into nights, devoting all his hours to undisturbed work in his
darkened room with artificial illumination, so that none of those
intricate arabesques might escape him.''\strut
\end{minipage}\tabularnewline
\begin{minipage}[t]{0.97\columnwidth}\raggedright
\# Why I Write Narrative\strut
\end{minipage}\tabularnewline
\begin{minipage}[t]{0.97\columnwidth}\raggedright
Trevor Joyce\strut
\end{minipage}\tabularnewline
\begin{minipage}[t]{0.97\columnwidth}\raggedright
But, first: why not? A chain of telling that requires discrete and
distinct figures, already in some sense familiar, acting across a
background, largely unexamined and abandoned merely to suggestion,
towards the achievement of, or the failure to achieve, certain goals,
whose importance holds the earlier periods in suspense until in some
final resolution all that is significant crystallizes in a perfection of
plot and motivation, and all the rest, wanting any real brush with
language, retreats once more to ground: that's narrative as I knew it
thirty years ago, and it seemed then, as it does now, inadequate to the
world of my experience.\strut
\end{minipage}\tabularnewline
\begin{minipage}[t]{0.97\columnwidth}\raggedright
To avoid the fate of perpetrating such stuff, I instead wrote a poetry
which found itself increasingly characterized by argument, deploration,
pleading, threats, until I realized that I didn't like how that worked
either, and stopped writing altogether for about twenty years.\strut
\end{minipage}\tabularnewline
\begin{minipage}[t]{0.97\columnwidth}\raggedright
But I went on thinking and planning and testing and learning, though
largely just in imagination. I took a shine to the detective genre in
which, in theory, any object or event can be a clue and, as such, be
exalted into meaning as the everyday world sifts through the riddle of
observation, inference and deduction. But, even there, the world existed
only towards an end in which all might be revealed, and the head of
Holmes had a strictly limited inventory capacity.\strut
\end{minipage}\tabularnewline
\begin{minipage}[t]{0.97\columnwidth}\raggedright
Eventually, it was only in the parallel forensics of Gombrowicz's
"Cosmos" that I found some satisfaction with detection, since nothing is
revealed there, and the patterns and clues are rubbish and trivial
chatter remains fraught, even on rereading.\strut
\end{minipage}\tabularnewline
\begin{minipage}[t]{0.97\columnwidth}\raggedright
I had by this time resumed writing poetry, though with conventional
narrative either expunged or twisted, and temporal change borne instead
by repetition, either incremental or with calculated variation. My
models for this were in the refrain structures of folk-song, often
mediated through the likes of Yeats or Lorca, and in the interplay of
stasis and movement in Chinese parallel verse. I was forced to
recognize, however, that the 'lyric' mode which I practiced was quite as
prone to exclude the incoherent world as was the mannered narrative I so
distrusted. I had also encountered Cage for a second time, and with more
understanding of how the play of ambient noise across the receptivity of
his spaces might circumvent those exclusions and admit what might
otherwise not be acknowledged.\strut
\end{minipage}\tabularnewline
\begin{minipage}[t]{0.97\columnwidth}\raggedright
I wrote a longish 'bicameral' piece called \textbf{Syzygy}, consisting
of two halves, ``The Drift'' and ``The Net'' respectively. ``The Drift''
consists of twelve compact, elliptical, but distinctly 'lyric' poems.
``The Net'' is a single poem of 72 long lines, comprising 24 three-line
stanzas. Significantly these two sections are made up of exactly the
same phrases reordered, rigorously and exhaustively mapped through a
one-to-one matrix, the exact structuring of which is not directly
relevant here. There is a brief set of notes added, which ground some of
the detailed references of ``The Drift'' within the empirical
world.\strut
\end{minipage}\tabularnewline
\begin{minipage}[t]{0.97\columnwidth}\raggedright
For me, much of the significance of the poem is in the way in which
sequences of phrases in ``The Net,'' arrived at through the blind
deploymen of predetermined procedures, carry a force both of lyricism
and of narrative, intense though severely fragmented in both cases,
which revealed a meaning different from and additional to anything I had
deliberately written into the work. Having lived with it for over two
years now, I have come to understand that "with the first dream of fire
they hunt the cold" and how "devastation fell attending headbone the
high" while "outside the foundries the clumsy the deadlocked
disintegrates" though "not a tremor manifests the rare the quickening
across these settlements" . So, I had confirmed for myself that a
densely overdetermined language, functioning in its most intensely
personal mode of the lyric, could survive radical disruption and return
from that alienation a yield which the reader might gather. It gave the
world an in.\strut
\end{minipage}\tabularnewline
\begin{minipage}[t]{0.97\columnwidth}\raggedright
I had meanwhile read all the Fu Manchu novels of Sax Rohmer, intrigued
the Chinese Doctor's ceaseless attempts to appropriate the world by
abducting experts in all fields, whose task was to analyse and replicate
in enhanced form all aspects of the workaday world. It matched my sense
as a child that what I saw in mirrors was a world diligently assembled
by unseen agents to match that which I inhabited, and I watched
carefully for small discrepancies to justify that sense, but never found
them, and the specialists of Fu Manchu's underworld empire attended only
to the great realms of science, technology, politics, always were
disbanded by resurgent law before their analysis could address the
classified advertisments in the newspapers, the torn betting slips
outside a bookie's shop, the inconsequential gabbling of drunks in a
pub. The workings of what I came to know, through Marx, Adorno and
Benjamin, as phantasmagoria fascinated and appalled me. And how is
responsibility to be assigned across mock-worlds if not through the
causal chains which are narrative's stock-in-trade?\strut
\end{minipage}\tabularnewline
\begin{minipage}[t]{0.97\columnwidth}\raggedright
Therefore:\strut
\end{minipage}\tabularnewline
\begin{minipage}[t]{0.97\columnwidth}\raggedright
\textbar{} while detailed depositions state \textbar{} how further on
\textbar{} within the wood \textbar{} . . . \textbar{} \textbar{}
\textbar{} the bright axe \textbar{} blossom suddenly \textbar{}
\textbar{} \textbar{} the long bones lever \textbar{} up from it like
anthers \textbar{} and beyond the startling \textbar{} calyx of teeth
\textbar{} \textbar{} \textbar{} an avid buzzing perishable \textbar{}
fruit set thicken \textbar{} and disintegrate \textbar{} to load with
sweet \textbar{} \textbar{} \textbar{} secure deposits \textbar{} of
afflicting gold \textbar{} their remote cells . . .\strut
\end{minipage}\tabularnewline
\begin{minipage}[t]{0.97\columnwidth}\raggedright
Here is not just one narrative, but two. Firstly, the sequence of
blossoming, the detail of anther and calyx expanding, the fruiting
adumbrated in the gathering of bees about the flower, and their
dispersal to the hive where they load the cells with honey. This
apparently natural and value-free sequence is overlaid on another causal
chain, which starts with the felling of a tree by a logger with an axe
(commercial or strategic deforestation has a long and significant
history in Ireland as in much of the developing world today). From this
action, the causal sequence is run backwards, seeking earlier sources
where responsibility may be assigned: the long bones of the arm, the
gasping jaw of the labourer, give way to the investors depositing their
profits in banks. This single instance is simple, but the structure of
reversed causality running back from an act of violence against the
person or ecological ruin, masked meanwhile by a natural sequence of a
bird fledging or a mineral cystallizing, is repeated twice more in the
poem, broken by cases for and against the possibility of asylum amid
such wilderness. Here, I began again to write narrative because the
forensic process it allows seemed to me necessary to any possibility of
living ethically, of recognizing and fulfilling due
responsibility.\strut
\end{minipage}\tabularnewline
\begin{minipage}[t]{0.97\columnwidth}\raggedright
Another narrative genre which interested me is that of scientific
experiment, where the researcher actively intervenes in the course of
nature, attempting to limit the causal influences at work, so that one
element may be manipulated, and the change in another, dependent
element, observed and measured. It is intended that the record of
correct prediction and accurate calculation of effect may then grant
understanding of what was previously obscure.\strut
\end{minipage}\tabularnewline
\begin{minipage}[t]{0.97\columnwidth}\raggedright
"A body thrown vertically down from the top of a tower moves through a
distance of 88 feet during the third second of its flight. Calculate,
then, the speed of projection, and determine the speed at which the
sleeve begins to move upwards."\strut
\end{minipage}\tabularnewline
\begin{minipage}[t]{0.97\columnwidth}\raggedright
And yet, it is in a field complex with uncertainty, that we attempt to
understand, to categorize, to measure, and the experimenter must attempt
to exclude all forces not considered relevant to the investigation, and
accurately account for all that ensues.\strut
\end{minipage}\tabularnewline
\begin{minipage}[t]{0.97\columnwidth}\raggedright
"When he attempted to speak to her, the patient jumped off the bridge
falling some 30 feet into about 20 feet of water. There is always a
chance therefore that the critical act or change may take place when the
observer's eyes are withdrawn."\strut
\end{minipage}\tabularnewline
\begin{minipage}[t]{0.97\columnwidth}\raggedright
Yet we must dispassionately observe, measure and record.\strut
\end{minipage}\tabularnewline
\begin{minipage}[t]{0.97\columnwidth}\raggedright
"Mild plethora of the face ensued, it being divided into three parts,
namely: the forehead, fair complected, one; the nose, another, sand
present in abundance there admixed with small crustacean shells; and
from the nose to chin, exhibiting extensive tooth loss though with roots
intact, another. Notice the blood tinged fluid coming from mouth. Red is
warm and radiates across the ground."\strut
\end{minipage}\tabularnewline
\begin{minipage}[t]{0.97\columnwidth}\raggedright
In such a manner, sometimes we come to face what can scarcely be
countenanced.\strut
\end{minipage}\tabularnewline
\begin{minipage}[t]{0.97\columnwidth}\raggedright
"It may happen that we are not aware of all the conditions under which
our researches are made. Some substance may be present or some power may
be in action, which escapes the most vigilant examination. Not being
aware of its existence, we are unable to take proper measures to exclude
it, and thus determine the share which it has in the results of our
experiments."\strut
\end{minipage}\tabularnewline
\begin{minipage}[t]{0.97\columnwidth}\raggedright
Though the outcome be uncertain, may we still presume to have advanced
knowledge in certain quarters? Lacking the machinery of suspense, to
what end may such a narrative aspire?\strut
\end{minipage}\tabularnewline
\begin{minipage}[t]{0.97\columnwidth}\raggedright
"A man has several bones in all, and beauty is lost when meaning and
form are split asunder. The handsome man must be swarthy, and the woman
fair, etc., the genitalia, both internal and external, without injury.
Provisional diagnosis: probable drowning. And had we exhausted all the
known phenomena of a mechanical problem, how can we tell that hidden
phenomena, as yet undetected, do not intervene in the commonest actions?
I will not tell you about the irrational animals, because you will never
discover any system of proportion in them."\strut
\end{minipage}\tabularnewline
\begin{minipage}[t]{0.97\columnwidth}\raggedright
The plausibility of narrative is increasingly an issue for me, and not
without reason, perhaps, given the dominance of contending
master-narratives in the interpretation of the Irish past and,
consequently, in my present world. From such master-narratives it seems
worth trying to retrieve as much as may still have value. I have tried
this in a recent long work called ``Trem Neul''---part prose, part
'verse'---which I see as, in part, an attempt to recoup part of the
history of my world from what Beckett terms 'the uniform memory of
intelligence'. "Genealogies. The elementary tables. Dictionaries,
assembled in blind frosts. Grammar and chronology. Libraries. Index: the
Encyclopaedia, damascened with ice. So is the perfect body of knowledge
dislimned."\strut
\end{minipage}\tabularnewline
\begin{minipage}[t]{0.97\columnwidth}\raggedright
How may one conduct a narrative of change, of loss and recovery, of
breakage and continuity, without presuming the existence of distinct
agents, freeing them incredibly from their ground, and committing a
plotwork of events, utterly plausible because familiar and foreknown?
Scholarship has shown us how the integrity and closure of the human
agent was arrived at in early Greek poetry, must we take that achieved
unity as more than provisional? Can we not tell without it?\strut
\end{minipage}\tabularnewline
\begin{minipage}[t]{0.97\columnwidth}\raggedright
"We build ourselves through the world and each through other, and this
proceeds to death as the world alters with experience." The bodies in
this plot are not distinct, either from one another or from their
ground, they emerge to make themselves, enjoy a transitory closure, and
then resolve again into a ground which offers further figures. "When the
biology of your body breaks down, the skin has to be cut so as to give
access to the inside. Later it has to be sewn back like memory, when it
may house all knowledge. Memory is our comfort and our attire. Fashioned
with our hands it is the accomplishment of our dreams and lapses; always
a meaningful pattern though never an abiding one; a shifting harmony of
sub-patterns. Pretend I'm lost and try to find me."\strut
\end{minipage}\tabularnewline
\begin{minipage}[t]{0.97\columnwidth}\raggedright
Because I wish to work comprehensively with the world which I inhabit,
however, and because that world is current with named identities, I have
admitted one such, and one specific narrating voice to speak of him. I
have permitted myself also two specific occasions in time, one exactly
situated starting at about 11 a.m. on February 15th. 1838), one not.
Between the two is a connection, and my essay is to account for that, to
recount it. "Do not think it coincidental that memory should begin to
fail just as taxonomies become a prominent tool for thought."\strut
\end{minipage}\tabularnewline
\begin{minipage}[t]{0.97\columnwidth}\raggedright
As, I believe Braudel and the Annales school refocused history on the
wholeness of ordinary lives, their habits, orientations and crises, by
attending patiently to things left, I see my present course, of
rendering the experience of connectedness, as being sustainable only
through exercising a similar patience with language. No longer expecting
to find there an exact mirror-image of the world I know, but rather to
have it deliver me one I don't.\strut
\end{minipage}\tabularnewline
\begin{minipage}[t]{0.97\columnwidth}\raggedright
\textbar{} We will read \textbar{} every day \textbar{} in the afternoon
\textbar{} When shall we learn \textbar{} to write? \textbar{} We shall
soon \textbar{} learn \textbar{} I once went \textbar{} to Europe
\textbar{} but I do not now \textbar{} remember \textbar{} what I saw
\textbar{} there \textbar{}\strut
\end{minipage}\tabularnewline
\begin{minipage}[t]{0.97\columnwidth}\raggedright
It is in preferring to concentrate on the unpredictable ground rather
than to people it with puppets of my own making that I have elected in
these more recent pieces to work increasingly through collage. Each
fragment of language I adopt is already tale-bearing, a vector: carrier
from a prior host, director of action also across the space of my world.
And the point of such a narrative? Interim figures on an interim ground;
preservation of the complex weave of actions, not denouement; attachment
understood embraced abandoned; wanting executive or summary.\strut
\end{minipage}\tabularnewline
\begin{minipage}[t]{0.97\columnwidth}\raggedright
\#\# Notes\strut
\end{minipage}\tabularnewline
\begin{minipage}[t]{0.97\columnwidth}\raggedright
All quotations are from \textbf{Syzygy} (Wild Honey Press, Dublin,
1998), available on the web at the Sound Eye site
http://indigo.ie/\textasciitilde{}tjac/Poets/Trevor\_Joyce/Syzygy/syzygy.htm\strut
\end{minipage}\tabularnewline
\begin{minipage}[t]{0.97\columnwidth}\raggedright
\textless{}!--\strut
\end{minipage}\tabularnewline
\begin{minipage}[t]{0.97\columnwidth}\raggedright
I'll need to restore the footnotes manually, I think. They seem to have
disappeared.\strut
\end{minipage}\tabularnewline
\begin{minipage}[t]{0.97\columnwidth}\raggedright
2 \textbf{Without Asylum} (Wild Honey Press, Dublin, 1998) 3 ``Damaged,
we bleed time.'' This is the central one of a sequence of three
prose-poems called \textbf{Hopeful Monsters} (Wild Honey Press, Dublin,
1999, forthcoming). This section is available on the web at the Alsop
Review site http://www.alsopreview.com/tjdamaged.html 4 Ibid. 5 Ibid. 6
Ibid. 7 Ibid. 8 Samuel Beckett: \textbf{Proust} (Calder and Boyars,
London) p.32 9 ``Trem Neul,'' section XLII. The complete text will be
included in my forthcoming collected poems, \textbf{with the first dream
of fire they hunt the cold} (New Writers' Press, Dublin, 1999). The
title is from a phrase in the Irish language, meaning "through my
dreams". 10 I have in mind the first chapters of Bruno Snell's
\textbf{The Discovery of the Mind} (Harper and Row, New York), and of
E.R.Dodds \textbf{The Greeks and the Irrational} (University of
California Press, California). 11 ``Trem Neul,'' section XI. 12 Ibid,
section XV. 13 Ibid, section XXXI 14 Ibid, section XXVII\strut
\end{minipage}\tabularnewline
\begin{minipage}[t]{0.97\columnwidth}\raggedright
--\textgreater{}\strut
\end{minipage}\tabularnewline
\begin{minipage}[t]{0.97\columnwidth}\raggedright
\# Poison\strut
\end{minipage}\tabularnewline
\begin{minipage}[t]{0.97\columnwidth}\raggedright
Kevin Killian\strut
\end{minipage}\tabularnewline
\begin{minipage}[t]{0.97\columnwidth}\raggedright
I'm standing on a flat plain, and then, or so it seems, a little hole
appears in the sand ahead of me (like that movie \textbf{Tremors})? The
hole grows larger in diameter, this is my sanity, and all the little
pieces of my sanity are breaking up and slipping down into the hole.
That's what it feels like. I'm trying to write this piece, "Poison,"
about the ways in which the writer's personality dissolves as it weaves
in and out of the sentences he or she so painfully struggles to produce.
While writing it I notice a host of familiar symptoms. Nobody calls me
on the phone. I feel so isolated. I can't hear very well and wonder if
I'm going deaf like Beethoven, like Brian Wilson. When people do speak
it's with loud, ultra-charged voices, as though they're annoyed with me.
I feel like I'm losing my mind and with my mind, the meaning of life I
once held onto. I used to think that people are basically good at heart.
I used to think I would be happy someday, but now I feel differently,
that there's no chance for me, since that hole before me is opening up
and soon everything I ever clung to will be sucked down into it and I'll
be homeless and curled up on the gutter outside my former apartment on
Minna Street, a crack-ridden block South of Market in San Francisco. I
just bought a car and there's not really any place to park it.\strut
\end{minipage}\tabularnewline
\begin{minipage}[t]{0.97\columnwidth}\raggedright
How did I write a whole book? I'm trying to remember. In particular how
I wrote \textbf{Bedrooms have Windows}, which I loved so much too, I
wrote it in part as a shipwreck victim sends out a message in a bottle;
in particular to my dear friend Terry Black, with whom I'd lost touch a
few years before. The book has many appeals to him to get in touch with
me. (Before the Internet, through which, apparently, everyone is
available to or traceable by everyone else, this seemed my only
recourse.) I longed to see it in print, feeling that he would pick it up
and call me. But after it was published a mutual friend sent me Terry's
obituary, he had died in Richmond, Virginia, of AIDS, the same month the
memoir came out. This was not the answer I had hoped for. Part of me
felt that \textbf{Bedrooms have Windows} killed Terry Black---detailing
as it did our sex life and its creation, the way we had made up sex to
answer certain suburban needs for the authentic, the "real," the
colorful. Naming names, his. Implicating others, him. What portion of
one's personality is a fiction? It wasn't going to do any good to
realize this was a sentimental fantasy, part of the mind's response to
the inexplicable horror of AIDS, part of my own need to find myself on
centerstage always. I went back to "Poison" as I delivered it
originally, as a talk in Bob Gluck's series, \emph{In Context}, a series
of talks delivered week after week at Intersection, once an important
writing venue in San Francisco. This was a happier time for me---April,
1987. When New Narrative writing seemed wide open, a place where
something entirely new under the sun could be created. And that we were
doing it, doing so. But "Poison" I found out was imperfect, it didn't
help me. "Last week Dodie Bellamy's talk stressed the paradox of writing
as a two-way street-the importation of the world into the self, and the
generous export of the self back into the world. Bob's talk may have
been allied to my own idea of writing as an expression of the death
drive from \textbf{Beyond the Pleasure Principle}---what he calls
'Freudian pleasure based on an instinct to return to the
inanimate.'"\strut
\end{minipage}\tabularnewline
\begin{minipage}[t]{0.97\columnwidth}\raggedright
Every writing act is an act of dying, or killing, or mortification.
Every time I write it's to expose to the air of the page a false part of
my personality. ---I guess this goes directly against Bob's theory of
writing, and links mine closer to Dodie's, though not in any way she'd
like or approve of. My talk, I thought, would be a patchwork of
quotations, writing about me written by others; some that I wrote about
other people, lots I wrote about myself, and also work by others that
didn't have me in it at all---to give a wide scale against which I could
test my propositions. And the first was from Alan Davies' book
\textbf{Name}, which I found useful in its treatment of the tie between
language and self; is language a function of the word or of the self? Or
vice versa?\strut
\end{minipage}\tabularnewline
\begin{minipage}[t]{0.97\columnwidth}\raggedright
\textbar{} \textbar{} The cryptic tongue. \textbar{} We are getting
ourselves \textbar{} in the mood to have been \textbar{} done with
having been done \textbar{} with this again. \textbar{} It's all very
irreversible \textbar{} which is what \textbar{} makes its guts open up.
\textbar{} I wonder how long it will \textbar{} be until this writes
\textbar{} itself, in \textbar{} my direction. And we \textbar{} haven't
proved \textbar{} that it isn't true yet. \textbar{} When I think of you
\textbar{} the sentences come, \textbar{} but I don't. \textbar{}\strut
\end{minipage}\tabularnewline
\begin{minipage}[t]{0.97\columnwidth}\raggedright
Then I read a passage from Dodie's book \textbf{The Letters of Mina
Harker}, in which Mina, a fictional character, reflects on her love
life. In this passage I felt myself inextricably named and described, my
human body a vessel for a flood of narrative concerns.\strut
\end{minipage}\tabularnewline
\begin{minipage}[t]{0.97\columnwidth}\raggedright
Flaccid, KK's penis is endearing, so velvety and shy---but the trouble
with babies (as my mother always said) is that they grow up---your bed
inflates to the breaking point with thirty-three-year-old male desire
panting and prodding \emph{the thing inside burst through her belly,
horror props, sausage links and ketchup} around my neck KK fastens a
locket filled with a snip of his hair \emph{to protect me from evil} I
cross the street with my eyes closed, cars screech then cease to exist,
the atom remains unsplit forever, cells multiply at a reasonable rate
every death is from a natural cause LOVE LOVE LOVE LOVE LOVE LOVE LOVE
LOVE LOVE remember when his nails were half an inch long, thick, hard,
yellowed---he clipped them off for me parting my capillary pink flesh
without a scratch \emph{all it took was one "ouch"} claws retract,
breathing softens. He extends his palm from the bathtub and says, "Sit
on it," human form follows function, in Cocteau's \textbf{Beauty and the
Beast} gloved arms poke out of the walls holding candles, their flames
trembling as Beauty recedes down the endless corridor \emph{I wish I
could walk through mirrors} our entwined bodies tighten into a circle, a
champagne bubble about to be swallowed by Marilyn Monroe \emph{pushing
the metaphor to the breaking point, in a word: orgasmic} when we fuck we
are two great hands shaking \emph{his cock a thumb} in an explosion of
light the bearded creator in Blake's watercolor points from the
heavens---mortal heads bow or stare up in awe and terror the way I do
whenever I'm naked \emph{a woman's hair is never thick enough to hide
her thoughts} KK reaches for a condom, fumbles with its little blue
capsule PRESS FIRMLY ON DOT AND PULL APART \emph{town fathers pack data
in time capsules burying them underground, schoolchildren dig them up,
crack them open a hundred years later. Things.}\strut
\end{minipage}\tabularnewline
\begin{minipage}[t]{0.97\columnwidth}\raggedright
I used to ask Dodie, "Shouldn't it be, 'When we fuck we are two great
hands shaking his cock a \emph{giant} thumb?' Wouldn't that make the
passage clearer?" She said, "No." That's all. Just "no."\strut
\end{minipage}\tabularnewline
\begin{minipage}[t]{0.97\columnwidth}\raggedright
At first I was embarrassed by this passage and many others like it in
her writing, for if at any time my penis is flaccid I don't want to know
about it, nor the world to suspect it. At the same time, I felt
flattered, singled out by her language as I felt singled out by her
love. "Beauty recedes down the endless corridor \emph{I wish I could
walk through mirrors}." The sense of the syntax issues a seductive
invitation into a mystery world---and I am that mystery. Then the cold
water hit me; with a start I came to and asked the difficult question,
is it I who is being described? \emph{The Letters of Mina Harker} seem
to describe the sex lives, the love lives of two actual people, but of
course they don't, they don't even especially want to: their veiled and
mediating nature hints at this:\strut
\end{minipage}\tabularnewline
\begin{minipage}[t]{0.97\columnwidth}\raggedright
---extends his palm from the bathtub and says, "Sit on it," human form
follows function, in Cocteau's \textbf{Beauty and the Beast} gloved arms
poke out of the walls holding candles, their flames trembling as Beauty
recedes down the endless corridor\strut
\end{minipage}\tabularnewline
\begin{minipage}[t]{0.97\columnwidth}\raggedright
---and so forth, then the phrase "pushing the metaphor to the breaking
point" nails it. Do you know the funny feeling you get when a stranger
waves on the street, you wave back, and then you realize the stranger's
waving not at you but at the little weirdo behind you in the brown
fedora? One's identification with the words that seem to conjure one up
like "I Dream of Jeannie" out of black letters on a white page is
like---just like-the way your heart then sinks and a blush colors your
face; except it's even more mortifying---that stranger on the street's
waving all right, and you wave back, then you turn around and it isn't
even a person that stranger is really waving at, but an atmosphere
perhaps, a draft of air. Only convention associates the wave (or the
sentence) to a corporeal body like one's own.\strut
\end{minipage}\tabularnewline
\begin{minipage}[t]{0.97\columnwidth}\raggedright
My next example I wrote. I quoted from \textbf{Bedrooms have Windows} to
give some idea of what goes through a writer's head when he decides to
name real people in potentially scandalous situations-sexual in this
case, but scandalous only insofar as they deal with the body; also to
suggest the peculiar vanity of the writer-and then I'll give it a fairly
close reading.\strut
\end{minipage}\tabularnewline
\begin{minipage}[t]{0.97\columnwidth}\raggedright
One evening several years ago I was lying in bed, after some
unsatisfactory fumbles towards "safe sex" with a writer I once admired,
Dennis A. He turned his head-exactly as he'd turned mine, an hour
earlier---and get this, he said, "Why didn't you think to take home that
Tom Boll too and we could have had a threesome." "Why didn't I think," I
replied, an echo of disbelief. "Dennis, I did think; I thought and
thought. Had I thought any more, I wouldn't be having this safe sex.
Madness wouldn't have been safe from me." I felt attracted to him when
he spelled his name in plastic magnetized letters on Aaron's
refrigerator. Then he spelled mine with the same colorful letters.
Language fused. It was like William Carlos Williams. "There were plums
in that icebox," I said to him. "Forgive me. Forgive me. I couldn't help
it; they were so ripe and so purple and so cold."\strut
\end{minipage}\tabularnewline
\begin{minipage}[t]{0.97\columnwidth}\raggedright
I doubt if "memoir," the word "memoir," really disguises from the
reader, or listener, that something very close to a real event is being
described. And naturally many people will correctly identify "Dennis
A.," the writer "I used to admire," with the Australian historian and
sociologist Dennis Altman. I figured out that I call him "Dennis A." to
make the narrator's personality kind of coy and obnoxious---and it
works, doesn't it? Maybe I was miffed because the sex we had wasn't very
perfect---or maybe I thought, "He's forgotten about me, I'll employ this
sentence to haunt him." Writing as an act of revenge, and naming names a
superior way of taking it. Like a virus, the poison of this passage will
break down, over time, whatever goodwill and nice feeling that I---my
body---will have by then created---like some ghastly race between life
and death, immune system and viral infection. Bob's novel Jack the
Modernist begins with a paradigm of intention, deadly intention, "You're
not a lover till you blab about it," where the ugliness of the syllable
"blab" is meant to suggest a whole medicine cabinet's worth of emetic
antidotes to the possibly-too-pretty word "lover."\strut
\end{minipage}\tabularnewline
\begin{minipage}[t]{0.97\columnwidth}\raggedright
The same watchful watchdogs will naturally conclude that this pick-up
scene takes place not just in anyone's kitchen, in the kitchen not just
of any old Aaron, but in Aaron Shurin's kitchen. Maybe they already know
of my admiration for his writing and his influence. When I first met him
I had just read his book \textbf{The Graces}, and I was so struck by it
I could hardly connect its author with a living human being; it was,
maybe, a gift from another planet like something out of Chariots of the
Gods. And since then I've gotten to know Aaron better "as a person," but
still blitzed by this admiration which---I see now!---is another form of
objectification, turned outward instead of inward. Hence there was
little psychic difficulty in turning his name, like a totem, into an
amulet to adorn my prose. Hence the repulsive, shy-making casualness of
using that name, "Aaron," and the social-climbing note its use strikes
here, as though I were his intimate, or perhaps his boyfriend, and what
is it in actuality but an incursion on him, aimed squarely at him at his
most domestic and private, the kitchen setting, the letters on the
refrigerator, perhaps the implied sneer (leftover from the 60s) that the
letters are made of plastic. A plastic language. Doesn't sound like a
recommendation, does it? It's true that telling stories, "narrative,"
does involve a local, in the sense that this quote of mine does have a
certain atmosphere, a sophistication, but it isn't really mine, it's
borrowed or stolen, the way you or I might borrow someone's boyfriend or
wife, return it to them and destroy a relationship like breaking a milk
bottle. (This isn't to suggest I haven't thought of the pleasure it
brings to writer and reader both, but---maybe because I'm a
Catholic---it's a guilty pleasure; there's pleasure in guilt too, and
even if there isn't there's the dying fall you get when you string one
word after another, after another, onto another, like bugle beads. The
pleasure of accretion.)\strut
\end{minipage}\tabularnewline
\begin{minipage}[t]{0.97\columnwidth}\raggedright
Next I'll read something I wrote for the "Jack Spicer" issue of
\textbf{ACTS}. My intention here was to get at what I saw as the malice
characteristic of Spicer's poetics, and I couldn't think of another
analogy except to describe an occasion from my own life, an occasion
when I felt malice directed at me.\strut
\end{minipage}\tabularnewline
\begin{minipage}[t]{0.97\columnwidth}\raggedright
My aim was to develop a visceral writing, a writing that would as
closely as possible parallel the effects of the anonymous letter:
insult, horror, shock and embarrassment. But did my face turn red when I
received one myself:\strut
\end{minipage}\tabularnewline
\begin{minipage}[t]{0.97\columnwidth}\raggedright
Dear Kevin Killian: Your piece "Tom-Tom" in the latest issue of
\textbf{No Apologies} aroused my suspicion that you must be an asshole.
First of all what is a gay man doing writing about a psycho-killer of
women and getting off on it? I'm suspicious that it's easier for your
fantasy screw to get off when the victim is a (dehumanized) "Miss Thing"
than if your homicidal maniac lusted after boys like---just like-you . .
. So what's your story? If it isn't a good one, this dyke'll write you
off entirely.\strut
\end{minipage}\tabularnewline
\begin{minipage}[t]{0.97\columnwidth}\raggedright
Threats. Intimidation. A whole map of misreading of my adorable piece.
Again and again I read this letter, each time with increasing unease and
paranoia. I howled into the open air, "What have I done to deserve such
venom? Who wrote this tripe?" Despite the writer's declaration, I wasn't
fooled into believing him a woman. In my heart I knew this letter is
from one of my so-called friends-of the male species. My only questions
were, which one---and why? I'd been put into the abject position, one I
was to maintain for a long time. Writing produces deracination, I
decided, by own best example. So much for theory. My content upset
"her," I thought to myself. "Take away the psycho-killer storyline and
there'd be nothing even a child would object to." And "her" form upset
me. Isn't "dictation" Jack Spicer's word for "receiving the letter"? You
don't know where it comes from; so you react badly. Martians are writing
you, altering the furniture in your room. I wanted a drink so bad I
wound up in North Beach! Bats flying from wall to wall, the whole
schmear, and voices writing all my poems and signing them with Dennis
Cooper's name!\strut
\end{minipage}\tabularnewline
\begin{minipage}[t]{0.97\columnwidth}\raggedright
Here the dominant note of my writing becomes real plain: I'm talking
about hysteria. I mean that seriously enough, a line that runs a beat
too fast, that's capable of all sorts of unexpected connections (which
is good) but because it's hysterical is equally capable of making false
connections or ignoring the valid (which is bad). That is to say, in my
heart of hearts I don't really believe Dennis Cooper is signing all my
poems. Why, I don't even know what I meant by that. That's paranoid,
isn't it? But I had to find out, what are my rights as a narrator and
character?\strut
\end{minipage}\tabularnewline
\begin{minipage}[t]{0.97\columnwidth}\raggedright
When I put myself in this false position, I run the risk all divas take:
we might prove too much for even our greatest fans. And so I've noticed
the characteristic note when others write about me is comic, often
picking on my frailties as a real person and usually, by the way, noting
either that I love the stars, drink a lot of Tab, or also, you know,
drink a lot, of, you know, alcohol. The next passage is from a story by
Francesca Rosa called "Canidae" (the Latin word for "dogs"). In it I
thought I recognized myself as the writer, "K.":\strut
\end{minipage}\tabularnewline
\begin{minipage}[t]{0.97\columnwidth}\raggedright
As K. speaks, a worried-looking dog pokes its head through the blue
gauze curtains separating the reading area from whatever is behind it. A
black dog, except that its fur is so sparse the skin shows through, a
gray and black dog then with running eyes; the wattles of its throat a
livid pink. It brings the rest of its body through the curtain and walks
upstage of K. who is telling us a story about Long Island. The dog
explores, listlessly sniffs at the floor, the podium base and K.'s
ankles. It walks back and forth a few times, turns a slow circle while
digging its teeth into the root of its tail, and then stops to stare at
us, the listeners, again. Not at all shy, confident, as if resigned in
its desolation, like one of those Kafka characters that have survived
their own death. K. does not, or decides not to, notice his center stage
companion, and takes us from Long Island to New York and then back
again.\strut
\end{minipage}\tabularnewline
\begin{minipage}[t]{0.97\columnwidth}\raggedright
When I first heard this story I remember she was sitting on stage and I
was sitting in the audience, our narratological positions reversed, and
my ears got red, and a voice rose up in me with a strangled scream, "You
can dish it out, Kevin, but you sure can't take it!" In Bob's
talk,"Truth's Mirror is No Mirror," he says, "I wonder if we are at the
point of reversing Flaubert . . . by accepting an artificial self, with
its own scale, depth and continuity. Eastern religion responds to a
'made-up' world with compassion---but with a fatalism that is the flip
side of Flaubert's scorn. To the degree we 'see through' Flaubert's
scorn, we suffer from and enjoy a self-contempt that is close to
bragging . . . I wonder if it's possible to be aware of the artifacted
nature of the local and not be contemptuous of it?---to understand it as
a construct and be moved by its depth?"\strut
\end{minipage}\tabularnewline
\begin{minipage}[t]{0.97\columnwidth}\raggedright
I was telling my sister Maureen that I was giving a talk on the fiction
of personality, and she nodded and said she knew what that's about.
Sometimes, she said, when she's walking down the street she hears in her
head her theme music so she feels she's starring in her own TV show.
Elizabeth Bowen wrote, "Nothing gets on the page that you started with,
and nothing you started gets on the page. To write," she said, "is to
rave a little." To this formation I've added my own strategy, to rave a
little before I begin to write, to exploit the "fiction of personality"
and to see, if not blinded by the brushfire, what happens then on the
page. When I was seventeen and living in my parents' house on Long
Island I threw lots of parties, so did my friends Terry Black and Lance
Mallamo, we did so to enjoy ourselves and to write about them
afterwards. This is from a story one of them wrote about a party I gave,
a story that begins with me warning my parents to stay away from my
party and my fun and my personality and my fictions:\strut
\end{minipage}\tabularnewline
\begin{minipage}[t]{0.97\columnwidth}\raggedright
"All I know," pouted Kevin, "is that I care about this house as much as
anyone else and I think you're a hideous couple even to think I'd let
anything happen to it. Why I just can't explain the chills that go up my
spine every time I see it. It's like finding a lemonade stand in the
middle of a desert. When I think about the beauty and splendor of this
house tears come into my eyes. Its white walls, its grapefruit shingles,
its collapsible mailbox, the one-of-a-kind sodded lawn, and, to top it
all off like the cherry on a banana split, I think our Spot, our
pedigreed dog, out in the backyard rolling in mud and barking at all
passersby as if to say, 'I'm black and I'm proud!'"\strut
\end{minipage}\tabularnewline
\begin{minipage}[t]{0.97\columnwidth}\raggedright
Kevin's fit of love and tears was interrupted by the doorbell. "I just
want to say one more thing," he said in a hurt voice. "If I see either
of you downstairs from now until tomorrow morning at ten o'clock I'll
set both of you on fire. Don't think I don't mean it either because I'm
pretty sure that I do."\strut
\end{minipage}\tabularnewline
\begin{minipage}[t]{0.97\columnwidth}\raggedright
With that he stormed out of the room, slamming the door behind
him.\strut
\end{minipage}\tabularnewline
\begin{minipage}[t]{0.97\columnwidth}\raggedright
Later in the story Terry himself is in a car driving to my party.\strut
\end{minipage}\tabularnewline
\begin{minipage}[t]{0.97\columnwidth}\raggedright
"Terry," said Mary Phipps, a girl in the back seat. While drinking a
swig from a bottle of cheap apple wine she continued, "I'm glad you're
acting like yourself again instead of like Kevin Killian."\strut
\end{minipage}\tabularnewline
\begin{minipage}[t]{0.97\columnwidth}\raggedright
"Thank you," the courteous Terry replied through gritted teeth. Mary
Phipps smiled at him and he could not help smiling back. They were true
friends and he knew he should listen to what she had to say whether he
liked what she said or not. So he did.\strut
\end{minipage}\tabularnewline
\begin{minipage}[t]{0.97\columnwidth}\raggedright
"Not that we don't like Kevin," joined in another good friend of Terry's
named Sam Rye. "It's just that you're not him and we liked you better as
you."\strut
\end{minipage}\tabularnewline
\begin{minipage}[t]{0.97\columnwidth}\raggedright
Terry resigned himself to the friendly rainstorm of advice, opinions and
admonitions that followed. He could not help wondering though why it was
that Kevin always got almost all the attention and he, Terry, so
little.\strut
\end{minipage}\tabularnewline
\begin{minipage}[t]{0.97\columnwidth}\raggedright
But I've argued myself into a corner if I insist that New Narrative
makes room for the stupid, the overblown, and the nasty---these are
three different veins of my writing, each with its own jet flow. The
bizarre thing was that, after I delivered this talk on April 27, 1987,
the woman who had written me the anonymous poison-pen letter lifted her
hand from the audience and said, "Oh, I wrote that---before I knew you
better." Do we get to know people that much better that we can change
our minds so quickly about them? I was stunned. I remain stunned. I
think of life as a big empty desert place---cooler than Death
Valley---but just about as big---"I know! Let's call it Life
Valley!"\strut
\end{minipage}\tabularnewline
\begin{minipage}[t]{0.97\columnwidth}\raggedright
Where, as I pause for a second, clearing my head, trying not to write,
dragging myself from that particular abyss of memory and missed
opportunity, there a small hole appears before me, about fifteen feet
before me. And all the things I brought with me to this valley are in my
trailer, fifteen yards behind. And one by one I lose them down the
ever-expanding hole---cans, jars, movie magazines, photos, food and
books. The hole keeps caving in on itself. My little home on wheels is
silver, rounded like the new Volkswagen models, and drives like a dream.
Its doors shear off with a sudden crunch of metal, bright in the noon
air, they slither across the desert floor into the hole. I'm next
perhaps, hold on to me.\strut
\end{minipage}\tabularnewline
\begin{minipage}[t]{0.97\columnwidth}\raggedright
\# Hunger -- Technology -- Emotion\strut
\end{minipage}\tabularnewline
\begin{minipage}[t]{0.97\columnwidth}\raggedright
Chris Kraus\strut
\end{minipage}\tabularnewline
\begin{minipage}[t]{0.97\columnwidth}\raggedright
From my LA diary ---\strut
\end{minipage}\tabularnewline
\begin{minipage}[t]{0.97\columnwidth}\raggedright
Los Angeles, sometime in the late 90s ---\strut
\end{minipage}\tabularnewline
\begin{minipage}[t]{0.97\columnwidth}\raggedright
My heart and stomach flip while waiting in the endless gourmet take-out
line at Say Cheese on Hyperion. This is the third full day not eating
.\ldots{} I stare through thick plate-glass at tureens of baby peas in
mayonnaise. Ten bucks a quarter pound, they're canned. Little bits of
foreign cheese displayed on the top shelf like so many sad specimens.
English tilton, camembert. From the bodies of imprisoned animals to the
air-conditioned case, it's obvious this food was never touched with love
or understanding. The chubby woman up ahead of me seems to think this
food is good. She is luxuriating in the moment when she speaks her
choices to the shop girl, even though the girl is bored and hardly even
listening. I'd hoped to trick myself to eat by ordering the most
exquisite food but now this place offends me. Say Cheese, Say Choose.
She wraps the names of foods around her tongue, pleased with her
passable pronunciation. Why do I hate everything? The food here is so
vastly overpriced, it no longer smells like food, it smells like bills
and coins and plastic.\strut
\end{minipage}\tabularnewline
\begin{minipage}[t]{0.97\columnwidth}\raggedright
If I'm not touched it becomes impossible to eat. It's only after sex,
sometimes, that I can eat a little. When I'm not touched my skin feels
like the flip side of a magnet.\strut
\end{minipage}\tabularnewline
\begin{minipage}[t]{0.97\columnwidth}\raggedright
The Alien penetrated me very slowly as we sat together on the bed. (This
is Ulrike Meinhof speaking to the inhabitants of Earth .\ldots{}As the
rope was tightening around my neck I lost perception but regained all my
consciousness and discernment. An Alien made love with me ...)
Uncovering his body takes my breath away. The paleness of it underneath
the soft dark hair. The Alien was naked. I had several of my clothes on.
We're very still. Fibrational quivers between our bodies in the dark.
"This's exactly how I imagined it would be. So smooth." It now becomes
possible to say anything. Low voice. "Don't move." "I like to hear your
breathing."\strut
\end{minipage}\tabularnewline
\begin{minipage}[t]{0.97\columnwidth}\raggedright
Like me, the Alien is anorexic. Sometimes we talk about our
malabsorption problems. Everything turns to shit. Food's uncontrollable.
If only it were possible to circumvent the throat, the stomach and the
small intestine and digest food just by seeing. After several weeks the
Alien decides that he will no longer make love to me because I'm "not
the One." Aliens spend their lifetimes on this planet testing,
searching. They get dewy-eyed, nostalgic about hometown virgins.\strut
\end{minipage}\tabularnewline
\begin{minipage}[t]{0.97\columnwidth}\raggedright
I'm in my kitchen making chicken noodle soup for the Alien. It's his
fifth day of withdrawal from valium and heroin. He can't walk, can't
sleep. I want so much for him to eat. Even though he says he doesn't
love me, I can't believe it's true. Therefore, I want to help him. "How
about a nice piece of wholewheat toast?" I ask, ladling out his soup.
"Don't take offense by this," he says, "but there's something I have to
tell you. Your cunt smells bad. If you washed the way you should, I
would've done the things to you I do to all my other girlfriends." I
gasp. Soup spills. "Sorry," he says. "I guess I should've mentioned it
when we were dating."\strut
\end{minipage}\tabularnewline
\begin{minipage}[t]{0.97\columnwidth}\raggedright
Food stripped of all its color, nutrients and smells and then
reconstituted, like my expensive hair (he loves it), Ravissant Salon,
\$300, like suburban small town cunts drenched in Massengil.\strut
\end{minipage}\tabularnewline
\begin{minipage}[t]{0.97\columnwidth}\raggedright
If I could only eat, a little ---\strut
\end{minipage}\tabularnewline
\begin{minipage}[t]{0.97\columnwidth}\raggedright
Although no one, to my knowledge, has analyzed the work of Frederich
Nietzsche through the occurrence of his blinding headaches, the poet
Kenneth Rexroth reads Simone Weil's philosophy through her anorexia.
Both are "egregious nonsense ... unholy folly." Rexroth puts lays the
blame where it belongs; on the Catholic men, Gustave Thibon and Father
Perrin, who took her seriously. "If only," Rexroth speculated in
\textbf{The Nation} (1957), "she had sought out an unsophisticated
parish priest, who would have told her 'Come, come, my child, what you
need is to get baptized, obey the Ten Commandments, forget about
religion, put some meat on your bones and get a husband' ..."\strut
\end{minipage}\tabularnewline
\begin{minipage}[t]{0.97\columnwidth}\raggedright
What you need is a good fuck, he said to me.\strut
\end{minipage}\tabularnewline
\begin{minipage}[t]{0.97\columnwidth}\raggedright
In \textbf{Holy Anorexia}, the scholar Rudolph Bell wants to take the
magnificence of the medieval female saints and drag them down to his own
level. He does this by conflating them with contemporary teenage girls,
who he finds pathetic and ridiculous. St.~Catherine, St. Theresa, and
Hildegaard van Bingham are all essentially the same; they're solipsistic
brats. The collective trans-historic She, the holy anorexic, "emerges
from a frightened insecure psychic world to become a champion of
spiritual perfection .\ldots{} Her will is to do God's will, and she
alone claims to know God's will." The holy anorexic is a manipulative
vixen; she "commands the war against her body and therefore suffers
deeply at every defeat, whether it is a plate of food she gobbles down
or a disturbing flagellation by nude devils and wild beasts. Then with
varying degrees of success, the holy radical"---like the newly slender
teenage girl---"begins to feel victorious ..."\strut
\end{minipage}\tabularnewline
\begin{minipage}[t]{0.97\columnwidth}\raggedright
Like witches, or female writers, thinkers, artists, who use the names of
others when chronicling their lived experience, holy anorexics are not
merely people to be differed with; they must be despised.\strut
\end{minipage}\tabularnewline
\begin{minipage}[t]{0.97\columnwidth}\raggedright
Shouldn't it be possible to leave the body? Is it wrong to even try?
Hungry yet repelled by food, Weil wrote: "Our greatest affliction is
that looking and eating are two different operations. Eternal beatitude
is a state where to look is to eat."\strut
\end{minipage}\tabularnewline
\begin{minipage}[t]{0.97\columnwidth}\raggedright
"The Alien is in my eyes. He's flooding my eyes. He's completely
penetrating me, every bit of me in my eyes.\strut
\end{minipage}\tabularnewline
\begin{minipage}[t]{0.97\columnwidth}\raggedright
He's in my eyes, he's spreading into my brain. Oh God,\strut
\end{minipage}\tabularnewline
\begin{minipage}[t]{0.97\columnwidth}\raggedright
he's in my mind. He's making me feel things in my body that I don't
feel. He's making me feel feelings, sexual\strut
\end{minipage}\tabularnewline
\begin{minipage}[t]{0.97\columnwidth}\raggedright
feelings. And he's there. He's everywhere. My body's changing."\strut
\end{minipage}\tabularnewline
\begin{minipage}[t]{0.97\columnwidth}\raggedright
David Jacobs, 1988 interview with an Alien Abductee\strut
\end{minipage}\tabularnewline
\begin{minipage}[t]{0.97\columnwidth}\raggedright
- from \textbf{Aliens \& Anorexia}, Semiotexte/Smart Art Press
1999\strut
\end{minipage}\tabularnewline
\bottomrule
\end{longtable}

In \textbf{Aliens \& Anorexia} I am attempting to make contact with the
writer and philosopher Simone Weil. At the moment of her death in
August, 1943, Weil became an Alien, i.e., a legend who transformed
politics into tragic poetry. Not all Alien encounters are hostile and
dispassionate invasions. There are others who see the Aliens as their
friends. Aliens encounters, like narrative, happen essentially in
realtime. In order to make contact with the Aliens, it is necessary to
carve out little pieces of yourself to let the Aliens come in.

In Paris, the librarian Florence de Lussy is editing Weil's
\textbf{Collected Works} for Gallimard. Weil's been dead for more than
fifty years. The project's overdue. The edition will contain eleven
volumes, and yet hardly any of what Weil wrote was published in her own
lifetime. Because she was an amateur philosopher, teaching philosophy in
French girls' lycees until 1941 when Jews were banned from working for
the government, Weil's writing has a narrative quality mostly absent
from the philosophy of her own time. She wrote articles for the leftist
press, reports, position papers and communiques during her years as a
trade union activist. Concurrently, and in the years following her
disillusion with the trade union movement, she wrote notebooks and
voluminous letters to her friends and colleagues. In solidarity with the
dispossession of the workers who her labor colleagues claimed to
represent, Weil sought out the experience of dispossession in the person
that she knew the best, herself. "If the 'I' is the only thing we truly
own, we must destroy it," she wrote in \textbf{Gravity and Grace}. "Use
the 'I' to break down 'I'." She was despicable; according to Bataille,
"odious, immoral ... a dirty hook-nosed Jew." Had anyone taken her
seriously enough to prompt her to write professionally for publication,
her work would not have happened. She was writing to find out what she
thought.

"The body is a lever for salvation," Weil wrote in her notebook in New
York. "But in what way? What is the right way to use it?"

Simone Weil was a performative philosopher. Because her texts are really
notebook writings, there isn't ever any subject that's apart from her.
Which is not to say she's writing "memoir" or "autobiography."
Channeling her subjects through her person, Weil does what writers do.
She is constructing a narrative in realtime---arriving at a state of
openness, witnessed by her audience, the reader---in which thoughts fly
in and out according to who's listening. In Weil's philosophy, just like
in narrative or phone sex, it's not the story that we're really hearing,
it is the fact and act of telling it. Her thought approaches
narrative---an emotional transparency that occurs when someone else is
listening to you.

\hypertarget{on-narrativity}{%
\section{On Narrativity}\label{on-narrativity}}

Rachel Levitsky

\hypertarget{a.-alphabets}{%
\subsection{A. Alphabets}\label{a.-alphabets}}

WHY? Because I love to quote myself. Because I have a fantasy of saying
it all. Because of the dense foundation. Because it is that difficult to
imagine it's a girl. Because, I want to write a beautiful sentence.
(Violet Leduc) Because, A fine rain was filling him with hope and
despair. (ibid) Because there is so much to explain when you are not,
say, Stanley Kubrick, like why the back of the naked torso (female) is
relevant to the subject matter. Because I should have written of the
degradation of an aging pussy before Richard Hell did. Because the world
needs to be created and stated over and over again if one is a woman.
Because some of my favorite women writers feel they must write as men
because it is impossible for them to write action as a woman. (See
Christina Peri Rossi; Iris Murdoch told a college class in which sat
Marcella Durand, that writing as a man was easier.) Because I am afraid
a penis will be placed in my absences when what I need is a dildo, in
other words I need plastic, rubber, elasticity, strength, reliability,
trust, safety. In other words to my dismay, my absences, my blanks, my
ambiguities are most comfortable to men, meaning, they will define them,
but it is women that I want.

\hypertarget{b.-bully}{%
\subsection{B. Bully}\label{b.-bully}}

Because I'm a poet I will tell you this is about narrativity and I hope
you will believe me. Because it is narrative and I am a poet I expect
you to believe me. I got together with Judy the Pothead DumDum. She
didn't ask me about my writing or say that I looked good. Trashy but
good. She didn't say it, so I am saying it. I wasn't drawing a cat, dog,
house. I was scribbling. I didn't make sense to you or to myself. I was
being messy which was a clear expression. I want you to notice me but
don't expect you to. I didn't wear underpants and it wasn't a bad dream.
My shirt is both dirty and stained, just what you'd like to accuse me
of. My lips taste good and my pits stink. You like the smell of my hair
but its just Vidal Sassoon, bought cheap at the five and dime, no,
that'd be Rite Aid, everything that is anything these days is
everywhere. I am practically nowhere. I am trying to remember to stay
off the drugs most times I forget why. My ends are splitting and I can't
afford the expensive haircuts and don't trust the lady at the five and
dime. My hairs true color reforms from the root. You'd prefer to be
fooled. Call me cat, dog, house frau in a light blue frock. I can do it
for a moment and my thin lips rock and roll in almost any shade, though
the counter boy'd have me believe one works better than another. I
haven't got a date or room on my card for dinner. Martinis are
expensive. I make them at home like Laura Dern's mother in Wild at
Heart. I take other women's misfortune as a warning. She's a wizened cat
so waits, knowing Ill move first and fall on my reputedly pretty face. I
made the point that the pronouns don't matter but watch how they do, in
vitro, embryo, room with a view. All dogs are as faithful as me. I've a
new middle name, its hard to be ethical. Watch the one who emits so
much, so much love, s/he will betray you most stunningly or most poorly.
Jade Netanya looks up from the middle of the photo, grief stricken, the
dead guy wielding the eight-inch hammer divinely inscribed with the
ineffable name of god was her fiance, though she's a lesbian. The rabbi,
the unbalanced baal shuva, Jade, my sister, fellow poets. She was
waiting for Ms.~Perfection/Beatific Amor but No One likes to be alone.
Its like being diabetic which uninterceded leads to blindness and
amputations. Nowadays in more adults and poor fat children. He was a
menace to the children. The complaining neighbors came to feel guilty as
Jews will, they didn't invite him over enough to dinner. Instead, they
made a lovely funeral. About me. As a child I was often having others
take off their pants and squat. Expunge or include? A? or P?

C. Cut me ups I think it's Nicole Brossard who said, surely among
others, that in poetry the character is always far away, whereas in
prose there are characters more alive than herself. Let me look for it,
aha, here it is, page 27, in ``She Would Be The First Sentence of My
Novel.'' The characters, they will talk and argue among themselves, they
will disintegrate and reappear, they will separate and unite. They will
explain, as Yaya and Grace do constantly, how it is that they both care
to be lovely and don't give a damn about happiness or what you think of
them, how they understand that the monstrous is the more potent state,
and embrace it, but politically and for survival, they must cover it up
constantly. Grace, who primps like a lady, is as disturbing as Yaya, who
emits the foulness of hell on earth, when she (Grace) walks into a room.
The lookers on are confused by their own discomfort. The obvious
grotesque (Yaya) is easier to ignore. Men claim ugliness for themselves,
it is free, without cost, abandonment. It is theirs, they own it, they
saddle us with the cost of beauty. I saw a woman who'd been freed from
her cancerous chin, now an absence covered by scarred and taut skin. Of
course class, capitalism, if she had been rich . . . The next day there
was another, a s/he, a distortion. I fell in love. And today, there was
another, my cheating heart. A beautiful creature in a convex mirror,
reading \textbf{The Adventures of Buffy the Vampire Slayer}. In my
dreams I am another, idealized, the one with whom I am in love. On the
train, I am a monster, I hold onto the hope that you will not turn away.
I wait for the novel in which I am not a ghost. The theory gave so much
hope---those French (what is it about the French?), the Lesbian Body
(bloody), De Beauvoir's the potent female finger on the button
(machine). Feminine plural, a sex that isn't one, alter egos, as drag,
Dred, Queer Confab, Bound. I began to believe I was beautiful on the
inside. Another woman became a mother. Another lesbian a wife. At the
Lesbian bar a screech, a piercing cry on the heals of a bitter
complaint. Forty pounds heavier, those horrifying saggy tits, from the
mouths of babes, the new lover of a good friend's ex-lover. Political
moment? Nowhere to be found. At the mall, at the Body Shop you may send
postcards to senators. I escape to Coney Island, as yet mall-less. Here
I can feel this---we are in the interim, the momentless, bearing the
interminable wait for the next moment. When the next time comes, I will
still be writing the Lesbian body, forty pounds heavier, with sagging
tits.

~

~

\hypertarget{my-narrative}{%
\section{my narrative}\label{my-narrative}}

Pamela Lu

~

First there was this thing, this experience, that I felt myself having
at last. I had felt the experience coming on for some time, without
knowing quite what it was about to be, what it might resemble, or when
exactly it might arrive. Perhaps I had read about it once, glimpsed a
description of it catching a character, a familiar protagonist, a
stand-in, in its sure, capable hands. There was the possibility that it
might miss me entirely as I loped along the wrong side of the alley,
ordered the wrong label of scotch from the wrong west-end saloon. A
possibility that I might lift my eyes just in time to watch it happen to
my neighbor instead and, all too late, jump to my feet to exclaim, "Hey,
that's my horse! Well, maybe not my horse exactly, but absolutely its
blood relation!" To which the experience would reply, tenderly and oh so
Jamesian, "Not this time, not you. Maybe next year, to the soul you will
have become if only you had learned to abandon the regret of your
current, expiring ghost ..."

Then I was sure that it had happened, not merely to a surrogate, but to
me personally, that I had not only jumped into its hands but entered it
in total, inhabiting its limits with the whole of my being. The
unanswerable question of it flogged me with the rapture of body and
soul. I called myself into existence by my proper name, and watched my
former life flow past in parallel streams, burbling and breaking up into
little tearful books. I demanded unambiguous signs of my entry into the
actual world. And this all happened in the middle of time, while I or
some other writer was fast asleep, dreaming up masterpieces that were
bound to be forgotten. While I was too busy becoming the character who
would have become me in an ideal life, friends carried on in my absence.
I felt the need to track the movements of my friends, to check in
periodically and make sure no one had died or disappeared. Usually they
could all be accounted for, milling about in packs around the pool
table, as an unidentified narrator slipped through their midst like a
generous thief, strewing stolen, anonymous pears left and right.

In the slow crush of time, characters aged, my friends grew older, and
the unthinkable happened to people. I did not, however, in the moment in
which I was destined to live, understand any of this. I had been
impressed by the nature of fate and outcome in E.M. Forster novels -\/-
how, in the absence of divine intervention, the hand of fate was given
over to mortals who schemed compulsively, recklessly, and blindly to
bring about often unfortunate results, perhaps even a tragedy. There
were ambitious do-gooders whose ambitions caused more harm than good,
prejudiced Victorians who lashed out politely and brutally at the weak,
colonial subjects who pleased their masters by losing out again and
again. The characters lived; they brought life down upon themselves;
they wrestled with causality and causality won. Is that why I'm always
confusing causality with casualty?

A friend from school once observed that fiction students as a group
seemed bigger and physically stronger than their poet counterparts. How
much would this statement suggest a three-ring bout in the writers'
circle, with scrappy but effete sonneteers grappling at the well-muscled
backs of market scriveners? What kind of over-dense poetic whimsy could
tickle the plodding narrative to unclog its drain to the real story?
This story was so real it was almost untellable; one would have had to
be already dead, installed in some paradise or afterlife, in order to
understand it completely. The neutron star of its as yet undiscovered
language weighed the earth down in orbit, while sending out regular
pulsations of superhot, ecstatic light which caused even fatalists to
see. And yet the parameters of the story demanded the most outrageous
courage of those who would experience it: biblionauts who would sail
paper spaceships deep into the fiery outskirts of existence and return,
defenseless, with a strange, material testimony. Was this, after all,
the true knowledge and danger that awaited our fumbling lives?

~

\hypertarget{narrotics-new-narrative-and-the-prose-poem}{%
\section{Narrotics: New Narrative and the Prose
Poem}\label{narrotics-new-narrative-and-the-prose-poem}}

Nicole Markotic

I am interested in a dialogue about "new" narrative, which is perhaps
not so much \emph{new}, as newly theorized. Many prose writers do not
consider themselves fiction writers, yet at the same time are not really
part of on-going poetics discussions which, for the most part, do not
focus on narrative. Although I also write prose fiction, I consider my
prose poetry and other alternative, interdisciplinary, and innovative
sentences to be a neoteric prose that both challenges and expands
language boundaries.

For me, the prose poem is a poetic strategy embedded within the
structure of narrative, and a feminist response to patriarchal language
and forms. By embracing both prose syntax and poetic disruptions, the
prose poem defies conventional linear grammar and refuses to satisfy my
desires for either poetry or story. My desire is for so much more than
causal, linear, rational and persuasive normative sentences. In my novel
and in my poetry, I try to live between the promise of narrative and the
fulfillment of the habit of fiction.

By expropriating two distinct genres while presenting a form of writing
that is seen to consist of both, prose poetry presents itself as hybrid
writing; a hybrid that I explore as crossing between the desire to
exceed formal considerations, and the narrational insistence surrounding
such considerations. In my prose poems, I try to embrace the problematic
of poetry that "looks" like ordinary prose narrative, yet invites
disjunctive readings which may extend beyond traditional poetic forms
and conventional conceptions of narrative. My pieces "fail" as
poetry--yet continually celebrate the erotic contradiction such failure
narrates.

The prose poem, for me, offers itself as a "genre" that permits and
encourages me to move freely from sentence to line-break to full-stop to
repetition. Without entirely settling on its own definition, the prose
poem rattles my airtight nerves, upsets my crabapple go karts,
insistently and lasciviously rubs my discourse the wrong
way---inventing, always reinventing itself. Its very possibility denies
its own existence. For me, the prose poem remains liminal: not merely
transgressive, but indeterminite, caught in the phoenix act of
disappearing, reappearing as its own \emph{possibility}.

My interest in this form begins at the level of the sentence. What makes
a sentence a line of poetry rather than an excerpt from an essay or
novel? The answer, for me, is the surrounding words and sentences, the
position of those words on the page, the complicated ache of the forced
return at the end of each "line." Ear and eye games, the playing of
interruption and proceed. Not stop-and-go, but go-and-go-farther. As
soon as I take a step towards the horizon, the horizon
reconfigures---itself and me.

What interests me about the sentence as poetic form is that poets who
are \emph{least} interested in representational language are most often
the ones who turn to a grammatical structure which invests heavily in
sentential logic. The use and abuse of the sentence, then, seems to be
the focus of the poem. And because one sentence follows the next--which
offers a visual hypotasis but a semiotic parataxis---the poem insists on
a reading that is far more linear (for lack of a better word), than is a
line-break poem with disparate images stacked one above the other. The
one poetic device designated as prose or \emph{not} poetry, the
sentence, becomes the choice unit of composition for poets who intrigue
me, shock me, paratactically get under my skin.

Within each sentential piece can emerge a plasticity resistant to
notions of purity in either prose \emph{or} poetry. Gertrude Stein chose
the sentence as her basic unit of composition. Her sentences release
readers from the semantic baggage traditionally loaded into narrative
prose as they re-define, distort, conflate, skew, and otherwise render
the sentence pliable and visceral:

\begin{quote}
A kind in glass and a cousin, a spectacle and nothing strange in a
single hurt color and an arrangement in a system to pointing. All this
and not ordinary, not unordered in not resembling. The difference is
spreading. (\textbf{Tender Buttons} 9)
\end{quote}

Such prose rumbles inside the belly of poetry, or perhaps poetry shifts
awkwardly within the abdomen of prose. Readers, seduced by sentence
structure, discover themselves trespassing an erotics of prose,
transgressing away from familiar and known fiction offerings.

Writers engaging in what Stephen Fredman calls "poets prose," struggle
to create new ideas from \emph{inside} the conventions that constitute
the either/or structure of the prose poem. This structure is the chance
for prose writers to both follow narrative and to deviate away from
story. For example, Robert Kroetsch, in his poem "The Sad Phoenician,"
pursues an elusive and/but dialogue that generates a bifurcated and
pending narrative:

~and~ ~~~~~~the Phoenicians gave us the whole works\\
\hspace*{0.333em}but~ ~~~~~~what does that matter to a world that
ignores\\
\hspace*{0.333em}\hspace*{0.333em}\hspace*{0.333em}\hspace*{0.333em}\hspace*{0.333em}\hspace*{0.333em}\hspace*{0.333em}\hspace*{0.333em}\hspace*{0.333em}\hspace*{0.333em}\hspace*{0.333em}\hspace*{0.333em}\hspace*{0.333em}
them, the Greeks got all the credit of\\
\hspace*{0.333em}\hspace*{0.333em}\hspace*{0.333em}\hspace*{0.333em}\hspace*{0.333em}\hspace*{0.333em}\hspace*{0.333em}\hspace*{0.333em}\hspace*{0.333em}\hspace*{0.333em}\hspace*{0.333em}\hspace*{0.333em}\hspace*{0.333em}
course, because they stole the alphabet\\
\hspace*{0.333em}and~ ~~~~~~the girl from Swift Current, she more or
less\\
\hspace*{0.333em}\hspace*{0.333em}\hspace*{0.333em}\hspace*{0.333em}\hspace*{0.333em}\hspace*{0.333em}\hspace*{0.333em}\hspace*{0.333em}\hspace*{0.333em}\hspace*{0.333em}\hspace*{0.333em}\hspace*{0.333em}\hspace*{0.333em}
took everything\\
\hspace*{0.333em}but~ ~~~~~~the kitchen sink, claiming all my books,
my\\
\hspace*{0.333em}\hspace*{0.333em}\hspace*{0.333em}\hspace*{0.333em}\hspace*{0.333em}\hspace*{0.333em}\hspace*{0.333em}\hspace*{0.333em}\hspace*{0.333em}\hspace*{0.333em}\hspace*{0.333em}\hspace*{0.333em}\hspace*{0.333em}
my records, my prints; she moved in with that\\
\hspace*{0.333em}\hspace*{0.333em}\hspace*{0.333em}\hspace*{0.333em}\hspace*{0.333em}\hspace*{0.333em}\hspace*{0.333em}\hspace*{0.333em}\hspace*{0.333em}\hspace*{0.333em}\hspace*{0.333em}\hspace*{0.333em}\hspace*{0.333em}
photographer from Saskatoon, the one who\\
\hspace*{0.333em}\hspace*{0.333em}\hspace*{0.333em}\hspace*{0.333em}\hspace*{0.333em}\hspace*{0.333em}\hspace*{0.333em}\hspace*{0.333em}\hspace*{0.333em}\hspace*{0.333em}\hspace*{0.333em}\hspace*{0.333em}\hspace*{0.333em}
takes those sterling pictures of the wind\\
\hspace*{0.333em}and~ ~~~~~~I should sue\\
\hspace*{0.333em}but~ ~~~~~~she follows large flocks of birds, I hear,\\
\hspace*{0.333em}\hspace*{0.333em}\hspace*{0.333em}\hspace*{0.333em}\hspace*{0.333em}\hspace*{0.333em}\hspace*{0.333em}\hspace*{0.333em}\hspace*{0.333em}\hspace*{0.333em}\hspace*{0.333em}\hspace*{0.333em}\hspace*{0.333em}
calling my name\\
\hspace*{0.333em}and~ ~~~~~~pleading\\
\hspace*{0.333em}but~ ~~~~~~why she developed a thing for adverbs,
that's\\
\hspace*{0.333em}\hspace*{0.333em}\hspace*{0.333em}\hspace*{0.333em}\hspace*{0.333em}\hspace*{0.333em}\hspace*{0.333em}\hspace*{0.333em}\hspace*{0.333em}\hspace*{0.333em}\hspace*{0.333em}\hspace*{0.333em}\hspace*{0.333em}
too rich for my blood, I want to tell you\\
\hspace*{0.333em}and

Readers wander through an and/but tug-of-war that is also a narrative
wandering, from the Phoenicians and Greeks to Saskatchewan, to the very
grammar of the telling. It is this \emph{telling} that gets so many new
prose writers into trouble---isnt plot the very basis of novels? How can
one tell a story and avoid fiction? And why do we want to?

Feminist writers must resist and reinvent patriarchal language to make
it our own. We cross a border every time we pick up a pen or turn on the
computer or hum into a tape deck. The border is a visible line on the
page that the prose poem physically outlines. For many women, this is
not simply a gendered border, but one that has been criss-crossed by
social lines such as race or class. Harryette Mullen, in her prose
poetry book, \textbf{S*PeRM**K*T}, investigates the connective links
between the social and the linguistic:

Eat junk, don't shoot. Fast food leaves hunger off the hook.\\
Employees must wash\\
hands. Bleach your needles, cook the works. Stick it to the frying\\
pan, hyped again.\\
Another teflon prez. Caught in the fire around midnight, quick and\\
dirty biz. Smoked\\
in the self-cleaning oven.\\

Mullen's disconnected sentences can be read to produce meanings that
address subjects ranging from drugs to poverty in contemporary USA. But
the structure of the "narrative" does not create fictional characters
and circumstances to battle these issues; rather, Mullen's words entice
at the same time as they startle reading practices. Word association and
colloquial speech patterns reinvent the chance to re-enter the text from
another crack. This text does not close the book on the story, it
continues to remind that reading is a socially-coded act, and that a
reader's subjectivity is constructed through how she enters language.

Language enters me, and my sentences struggle against what I know to be
"just" a sentence. I hesitate in doorways and on fences, I plan my route
along edges and cracks in the sidewalk, I pencil in margins made up of
words only words, and then I cross over into blank space. Not because it
is "new" or "white" or "virginal" or any other dominant appropriative
term, but because the blank space is where the prose poem offers a
\emph{visible} margin, a territory (known or not) that hangs onto the
edges of writing, clings to the edges of edges. I get edgy when I try to
read or write a prose poem, anxious, excitable, downright perplexed. I
hang about on the inside of this perplexing form because it keeps me on
my language toes: unsure and insecure, tip-toeing across whatever lines
a poem insists upon.

I "plot" at sentences and subterfuge, dream of a poetry that withstands
ragged-right fashions, yet continues to address narrative within
disjunctive non-lineated prose. I look to the prose poem as mutation--
by definition unstable, punctuated by eternally reproducing desires.

I want to carry words to where theyre already going, but not necessarily
to where I think they're carrying me. I strive to enact an and/but
\emph{narrative erotic} that is neither fiction nor poetry, that invites
readers to see beyond seeing and write beyond reading. I veer away from
the end of that prose line, from the punct signifying the closing
ceremony, the final page next to the cover, the certain and the
certified. I question the metaphorical imperative that insists toes are
what we tip on, and grasp instead the reassurance of instability,
embrace the inability to embrace a stable subjectivity, and echo
cheerfully what Kathleen Fraser describes as "fragments of a wholeness
only guessed at" (\textbf{Each Next}). My prose poems want/desire/lust
after such guessing contortions.

\hypertarget{the-narrative-site-narrative-as-preliminary}{%
\section{The Narrative Site: narrative as
preliminary}\label{the-narrative-site-narrative-as-preliminary}}

investigation

Ashok Mathur

TEXT MISSING \# \emph{from} Nude Memoir

Laura Moriarty

\hypertarget{section-13}{%
\subsection{9}\label{section-13}}

Ten minus one

She counted but was

not believed. The syllables like a mantra. The answer. Counting in order
to rid herself of numbers, of time. The increments of the present. The
same count. A decimal minus one reminds one never to go over or to go
back. Not a sequence but a repetition like a song. A petition like a
question. Never again. Duchamp as a form of grief. A diary. A display of
the invisible. Of visible decay. The mode here is eliminative. It is the
only mouth she has left.

"Piece of Ass Lost." The necropolis pictured by a dead poet in Ellroy's
\textbf{Clandestine}. Poet cop. The mouthpiece of fate. Exploitation as
love. Women who don't get older. Women as men. It doesn't matter. The
U.S. number one in violent death just past Mexico and Brazil. First in
the First World and the Third. The guardian of the infested spirit.
Someone follows a woman. She is not a muse but a fate. She talks too
much.. They drink together.They eat. Is she one of many or a singular
masterpiece. Impossible to follow him down that street. The action is
complete to the degree that it is not legible. Or transferable to
another medium. It is not unfinished but undone. Not the crime but the
gun. The piece of string with which you hope to find yourself. Missing.
She was missing her head. What we have is a reconstruction.

Diana watches herself on TV. She wanders absently around the house. She
is not dead. She is an artist. The interview reveals everything she
hopes for. There is a queen and a corpse. She finds her cigarettes and
begins to smoke. She is seventy. Voluptuous as parchment. Thickly
written. She takes into herself a sense of death. She lets it out.

She wonders how to preserve access without giving into a deadened sense
of hierarchical exclusion. Now that authority has been shown to be the
shuck it is. To provide a method for reading, to alter the activity of
reading itself. In retrospect or in the sense that it had already taken
place. She read the scene before the crime. Or we wouldn't be having
this conversation. His ambition is naked, mechanical. He also wants to
read.

The vaginal scroll

She performs (Schneeman)

Memorably laid out

Source or origin

Like a physical note

Of itself sufficient

Opened like the book

She wrote

The reconstruction is sloppy. Bad. Not convincing. Not made to be so.
Black velvet lines the unseen back of the door. To soften the blow. Her
character pokes through her performance like bones through flesh. Judy.
Judith. She loves him but he loves her double. Her twin. Who is
convincing. A con, snare or fox. A pest. He likes the angularity. The
bending back. But he falls during the kiss. He is injured. "I hate this"
she says. He doesn't miss her then.

The scroll unwinding and the performer

Takes over removing

Her shroud her look

Distracted in the picture

She takes off

Her glasses her theatricality

Will be attacked in place of her person

He argues her down

Her desire is unknown

Impossible to predict its hold on him. "... a diagram of the cruel
geometries of desire." The reviews are ambivalent. The scriptedness of
their exchanges are painful to him. Her double jumps into the bay. The
actor after. Also his double. Later on a sound stage wet close-ups. The
real actors. His strained face dazed with obsession. The paleness of her
hair, dark dress spread out but clinging also. Her limpness in his arms.
He stares down at her, climbing imaginary steps out of the sea. Her high
heels in silhouette. The twist of her waist is displayed to him, to us.

\hypertarget{section-14}{%
\subsection{1}\label{section-14}}

Madeleine is weak but alive. Unconscious. He takes her to his apartment
on Russian Hill. He kicks open the bedroom door, placing her carefully
on his bed. He first takes off her shoes. She stirs but doesn't wake. He
holds her in a sitting position, unzipping her dress in the back. Pale
wide shoulders, wet slip, brassier. He pulls her dress and slip over her
head. He becomes aware that her breasts are against him. Nipples like
buttons. He works quickly against her waking. Unfastens garters. Peels
off stockings. He reaches around her waist and has some trouble with
clips. He begins to breathe hard but quietly. Her head falls back. Her
neck is long and white. His hands are dark against her skin. He is
careful with his hands like a surgeon. He draws the blanket over her.
Even in this drowned sleep there is a defensiveness to the set of her
features. He stares only briefly. His audience, himself, is aware of the
contrivance of his composure. But is unaware that her unconsciousness is
an act. He knows only in retrospect. Perhaps she knows that he is both
protector and killer. She seems to be in control. Also looking back. But
she is out of her depth.

The Bride does not refuse

this stripping by the

bachelors, even

accepts it since she

furnishes the love gasoline

and goes so far as to help

towards complete nudity

by developing in a

sparkling fashion

Her intense desire for orgasm clings to her like his red silk robe. When
she runs she knows he will go where she goes. She has tricked him but at
too great a cost. She sees this as a job. He sees it as rescue and sex.
But she can't be saved from the danger of being false. She can't be had.
The hopelessness skews their perception. Heightens their senses.

The artist likes to see the woman go too far. The woman likes it too.
She manages the twists with some virtuosity. She puts herself beyond her
skill. The character changes in desperation but is unable, each time, to
survive. Her eyes in multiple shots. His eye in the credits. Jimmie. A
simple expedient but effective. Death to itself. And Kim. The actor is
left.

To be disassembled

Exit stage right

And another rescue

Marooned on a penal colony in space

She remembers his memory

His voice a slurred machine

"My nakedness creates you," he says (dead)

She brings up the interface

Turns it on

A golem. Like herself. Skin like flesh only not in her mind. Inside the
memory, the wetware an obscene cream as if robots ran on semen. She
turns to him. In pieces. In a kind of mechanical pain. He carries in him
the mothers who didn't survive. It is evident in his soft address. He is
her confidant. Her history. She confides. They confer in a jacked-in
version of love. His love is wired in. They know things together that
otherwise only she knows. "It's better than sex," she says. He says "
No, it's not."

Kienholz by definition

The bionic man

"We can rebuild him"

Not a faceless door

But organ donor

Plugged in

What happened to him

If she is Death in Orphee

He is her driver

An assisted death. They would be lovers in another life. She prefers his
later more vicious fiction but he hasn't grown into it yet. Their times
are not synced.

He in fact She assisted\\
~\\
Was scattered It was speaking\\
\hspace*{0.333em}\hspace*{0.333em}\hspace*{0.333em}\hspace*{0.333em}\hspace*{0.333em}\hspace*{0.333em}\hspace*{0.333em}\hspace*{0.333em}\hspace*{0.333em}\hspace*{0.333em}\hspace*{0.333em}\hspace*{0.333em}\hspace*{0.333em}\hspace*{0.333em}\hspace*{0.333em}\hspace*{0.333em}\hspace*{0.333em}\hspace*{0.333em}\hspace*{0.333em}\hspace*{0.333em}\hspace*{0.333em}\hspace*{0.333em}\hspace*{0.333em}Not
belief\\
She longed for The expert\\
~\\
She was Not love

~

~

\hypertarget{for-soft-architects}{%
\section{For Soft Architects}\label{for-soft-architects}}

Lisa Robertson

~

Lucite

(a didactic)

(because the present is not articulate)

Sit us on Lucite gently and we will tell you how knowledge came to us.

First the dull mud softened resulting in putrefaction, lust and

intelligence, pearl globs, jeweled stuff like ferrets, little theatres
of

mica, a purse containing all the evil smells of daily life. Then just
the

one vowel, iterate and buttressed and expiring; leaning, embracing,
gazing.

It devised with our claw identity for the sake of food. Selves, it says,

feeding us, I adore you, you know. Like a boy blowing from a tree, we

decided, we were paid, we were free. We incessantly prepared for the

future. On the title page, two angels blowing on the trumpets of fame
held

up a globe decorated with three fleur-de-lys and topped with a crown. We

learned habits and tricks; we faked happiness and relief. We were a
single

grin with lips pasted back. We said we saw Europes of hallucination,
fatty

broths sprinkled with deer, stenciled eagles, serpents and lurid rags.
That

was a format of saying, a frayed ligature. We were fading into the
presence

or absence of food, of sleep, of doubt.

Enough of the least. Doubt wants no space. Sincerity takes too long in
an

aggressive emergency. The present is not articulate. Also we feel a
sense

of duality. We wear out the art. We start to modify our vocables-\/-
flick,

pour, dribble estrangement's sex. Since it is we who are one, and we who

are scattered. We're this pair or more that can't absorb one another in
a

meaning effect. We feel palpitated by daylight and its deliberate
plants. A

description makes us. Among all these ejaculations and judgements we
feel

its elsewhere sculpt our body.

We would be walking down the street in the poetrycity. Gauze would be

everywhere. The day would be big, halting, gracious, revocable, cheap.
We'd

be the she-dandies in incredibly voluptuous jackets ribboning back from
our

waists, totally lined in pink pure silk, also in pure humming, and we'd
be

heading through the buildings with knowledge---that is, ephemeral

knowledge, like leafage or sleeves or pigment. The streets are salons
that

receive abundantly our description. The soft buildings are charming. And

our manners are software. We feel sartorial joy. We'd be at the river

watching the fat water lap the blonde built part, loving temporal

improprieties, the bright trash floating in slow liberation. We'd be

applying our make-up at noon by the river, leaning on the balustrade,

thinking about a little shun, a little fight, a little sofa. We'd be

thinking about hinges. We'd feel for our pen. Something might seduce us.
A

likeness. Samesame pouring through it.

Knowledge comes to us.

~

~

\hypertarget{monstrous}{%
\section{Monstrous}\label{monstrous}}

Camille Roy

~

One of the forms of narrative I write is software. Its lucrative. About
four years ago I used stock options to buy a house right around the
corner plus one block from one of the worst housing projects in San
Francisco. A couple thousand people live there. It gives my neighborhood
the highest child hunger rate in the city. Our first night in the house
someone got murdered, just before midnight . It was a block a way but
the shot sounded like it was in our back yard. One shot, a pause, then
another. Purposeful. Somehow I knew it was intended to kill, and not
just a couple of kids shooting at the moon. Plus the neighbor told us
hed had his car stolen 3 times.

Impenetrable poverty plus dumb fuck rules, class and race segregation:
Id moved into the only San Francisco neighborhood that duplicated on a
smaller scale what I grew up with. It annoyed me.

Locality, forever. \emph{Skewed}. Something huge gets mutilated as it
slides through a stuffy tube. Were on the beach very far to the west,
watching what pops out. It contains all of American culture. I came here
so tightly wound. Born on 43rd street south side Chicago \& havent been
back since I left the hospital.

From my dining room window, at the rear of the house, the project looks
strangely vacant. There just never seem to be many people around. The
buildings proceed down the hill towards the old industrial port like
giant shabby steps, but there is never anyone on the racks of balconies.
Ive rarely driven through it. Structurally, its sort of a dead-end
place, the way its laid out, like a suburban subdivision: streets point
into it, then twist up like spaghetti. The few drive-through streets are
dotted with dealers scoping out the passing cars. Im just talking about
the roads.

When I first moved in I often found myself dreamily staring out the
dining room window. I wanted to check out one of those balconies. The
view would be amazing, they practically hang over the bay. Developers
have been salivating over that piece of land for years. Nowadaze they
are nibbling at the edges of the project, building expensive live work
lofts for software designers on adjoining vacant industrial land. Its
weird. Different economic classes get spliced together via crimes. The
mode of interaction being criminal. So one day I mentioned to a friend
of mine that I didnt get it, how did dealers get kids to work for
them---playing courier, or delivery boy. What would a dealer have on a
kid? Why get involved with some jacked-up, scary asshole? I felt like an
idiot as soon as the words left my mouth. Patiently, step by step, my
friend explained how it was done, until I could have done it myself, as
obviously he had. All I had to do was ask. Knowledge. The getting \&
taking and the tearing up. Did I want to go there?

Of course I did. One day I walked in, took a place on the balcony next
to all of my friends \& drank their salty water. I listened to the
radio. I watched as a crack lady ran down the street behind a white dog.
Then the dog was scratching at the door. When I woke up that sound was
the shade, bumping against the window frame. And I was thinking, as I am
always doing and my thinking told me this: \emph{This is what I want.
Its inside my system of attractions. Im penetrated by the present and
its always the same: chronic anger. Awful but refreshing}.

From up here, it is all visible. From down below, also. Radiant
contradiction. Eyeballs: the severely vivid mechanism. Finally what is
seen is not a target but just circumference, expanding. Highlights
scatter across the field.

I walked into the projects a couple of weeks ago. Its right around the
corner, why not just walk? It was a friends birthday. She told me where
she lived, but it wasnt easy to find. The apartments didnt have numbers
on them, you had to just know. I asked a bunch of people. Kids were
running everywhere. How come I hadnt seen them from my back window? I
look whiter than usual, I thought, looking at my hand. Up here and not
even shopping, that made me odd. People looked at me skeptically. I felt
skeptical about myself, but slick, as in greased. I wanted to fall off
my little ledge. Bored with what had gotten dished up as myself. The
backwash of swallowing it. That nausea.

The balcony was great. I hung with my friends and listened to the radio.
They played that song I like, the one about money. Later we went out to
eat birthday steaks.

California is shallow. Thats true. Though it thrills me that I can walk
across the city without getting beat up for crossing some invisible
dividing line of racial turf. Of course I could get beat up for
something else. \emph{Im so easy to please.} Its the instability at the
heart which is to say the heartlessness of just washing away faster.

Im supposed to write about narrativity but these problems of locality
are where I get started. For me writing grinds itself into whats
familiar yet unbearable. Add mobility to that and, voila, narrative.
Disjunction is the formal consequence of this ripping and tearing, and
its packed with information, almost to the point of being insensible.

The streets I walk measure me. They measure you too, through mechanisms
both criminal and friendly. Including that knowledge is a kind of
spectacular innocence---the moment of saturation feels dazzling, but
there is probably no point. Still I love it, formally and erotically and
intimately. Its all about nested structures. I entrust my twisted little
pieces to the warm nest of the sick social body, and I feel our bond. It
nourishes me.

To theorize my point of view, to pursue critical formalism as a ritual
and as a grasp for power, let me put it this way. Narrative provides
context so that the rupturing of identity is recognizable. I think we
are impossible beings. We ruthlessly evade scrutiny, yet recognition is
the beginning of transformative emotion. Its a feeding process. You dont
know if youre creating a monster.

As a narrative writer I improvise recognition. Its like a location from
which mutant beings emerge. This feels true, in life they never stop
emerging. Look---they even swarm through this text. I allow it because
Im terrified and seduced. To encounter them via narrative is to
formalize a moment of surrender.

(An earlier version of this piece was presented at the Portable Talk
Series, SUNY Buffalo.)

\hypertarget{narrating}{%
\section{Narrating}\label{narrating}}

Leslie Scalapino

I just write, without advance plan as to a plot or form the writing
driven by a wind of my being in some urgency at the time. Each work I've
done has been written in that way. Each is different. The next time I'll
write dependent on what's occurring then. Here I'll say something about
Defoe (Sun \& Moon, 1995) since of my prose it was the hardest on me (in
a state of extremity) and required the greatest extension. 'Narrating'
is for me:

Inextricably related not simply because events 'are related', but
because the relation intrinsically is oneself. The phenomenal world of
nature is inseparable from one's 'conception' (visual and tactile) and
also does not come from oneself. One is one's relation to the rose
disc-cloud on the desert as to social occurrence, such as people's
minute nuances or war:

"disc floating on the desert, with cattle that come to the edge of the
blue water and the white desert not coming from it.

One has to agree with them or is not there which in society can not push
out from themselves the red soft hanging disc. that is separate.

misery in having to be in agreement with them giving up being there
though one can't do that."

One can not be as social definition, yet that exists. That social
relation is inherent conflict as oneself in which one cannot exist or be
'causes' the syntax to be a 'place'. Surfaces collapse on each other,
existing together as plates in paragraphs, clarity only possible by
inclusion of, or being, the conflict. Conflict is the writing's shape
configured and is attention in space.

Sleeping dreams are 'compared' to the literal "day"-as-surface as if
both day and dreams were surfaces as thin discs and placed side-by-side
the ego not in either of these but there also:

"One sees oneself as simply a shallow behavior that is thin. and so one
is free. And then turned inside so that one's dreams are sent into one
from the day, as they are it. So there is no REM. People rest in the
thin area. One laughs frisky. love in behavior is there, in reverse.
That is the street, the way it is seen."

Also fictional events and real events (the Gulf War) are placed
side-by-side regardless of their size:

"An object connotes by a prior connection or culture.

Where the object loses, or begins, its connotative existence is to be
just then non-connotative. (To see its edge...) would place the piece at
the point of dissolving. forming as an object and socially. which is
nothing. An object's form isn't anything as that. So it's the point of
the form and the conceptual meaning separating. (or rather, being the
same)...Trying to get to or to see where actions separate from their
social existence which is what they are that is their connotative being,
so seeing an action without that."

Action is seen 'without' itself in the sense that action (that which is
enacted, as physical movement) is 'other than' writing, the one can not
be the other they (both) are also phenomena as conflict, thin discs that
impinge on each other:

"Action collapses on itself and is compressed, in that it is description
of itself. It is thus in the present-time, still and calm.

It does not diverge from itself, supposedly. That's a conception which
people already know. Therefore speaking enables one to see the diverging
or separation that is this present. These actions occur as if to make
the pupil of an eye (some other's) dilate, and be held open. The present
takes place as 'some other.'

Life is dilated.

It does not diverge because description cannot be separate. as it is of
itself."

Defoe, the writer, created fictional characters who were supposedly real
giving documentary accounts. They were to be imagined as entities, which
had no reality. Paragraphs are plates of actions which are not single
and do not begin or end there in order for there to be reflection (as
the text) throughout everywhere that is not mirroring of authority as
authority is not (is not in) any single one of those actions:

"To name it will be merely caught again in their authority as they will
recognize that as themselves and will be in their trap again.

If there is no area between his dreams and waking life, there is no
identifying with anyone.

Walk through stream of hot metal bumpers of cars that then move like
plates shifting. Mass of plates and she's veering in it. On tin tail of
motorbike of yellow teeth thrashing in neck. He's ahead then. On it
again thrashing neck. The bike skidding and swaying in narrow channel
amidst bumpers. It bucks forward. Neck bike veering and bucking out on
the vast tin sheet. the sunlight is reflecting off. He wriggles free,
lunging. The motorbike lunges forward. Crashes into car's side in the
mass. Slippery blood on his head neck slashing. She's slashed hard
slamming into car's side. Rider on foot then, veering and is way ahead
then out on stalled mass.

Seeing him out on the vast reflected expanse."

Prior sight we see things before they occur: whatever dream occurs,
that's waking life. Perception does occur in people (or 'in' something
happening) what is it?

"Perception occurs before the context it is not in a setting.

Without seeing it as it was going on while seeing it, really. The man in
being authority is the process of down fawning on the feet of the other
man while this is merely an image a negative that is imposed on it. Like
an angel.

Is it that seeing into the future as what's going to happen to people it
then just is that. It occurs. Isn't deflected. That isn't continual
change which is aware as we are in it.

We can see it in ourselves. that's why it's aware.

People see that later. Not when they're young. it's separate from what's
actually there which is to be seen directly. has to be.

There is no 'inner reality' to the destruction that this person is
creating for themselves from within, which is seen from the outside by
others.

So suffering is in oneself, and has no 'inner reality'."

The relation of creating a spatial shape as the writing to changing what
occurs (a friend dying from leukemia was having a bone marrow transplant
at the same time as the bombing of Iraq the illusion that one could
avert and change this reality while it's occurring in the mind relation
that's the writing yet the friend died and the bombing destroyed) is an
impossible terrain/conflict. The illusion itself took one apart. It's an
illusion that is bathos, like jump-starting a car. There is nothing
commensurate. Fictional jump-starts:

Real boys (and others) starving in the Sudanese desert. One is in or has
the dreams of one's fictional characters to make a particular surface.
"There is no resemblance between the boys and the boys in the desert,
that's forward in time supposedly. really, as it has reality in itself
and is separate from this. One non-reality against others in oneself, it
is seen between them. as oneself is simply that.

It isn't changed when it gets to this. Or by it later.

Dreams are excluded from this, which occur. That's convenient because I
can't remember them. dreamed one which met this red (actual) sunset that
just occurred.

This isn't it. Flattened to my inherent nature. I just woke up. The man
who was the driver deserted by the boys who don't resemble them on the
desert, has a dream:

Venus was hanging above the red smear, which I didn't see in it. The
neighbors with whom had formerly been at war for many years agreed to
smuggle oil to them.

It doesn't resemble them and can be reproduced and only seen in reading
as lovely isn't the market. Seen as lovely it therefore isn't the
market."

In writing: "The seeing is the event, rather than arising from its
occurrence." And: "People's experience lags behind the time they're in."
In other words, Defoe is trying to find the relation between seeing
(apprehension) and occurrence (events occur 'outside'). There is a
relation one makes:

"Make writing that forms before motions.

As when I was in a rage, it formed first I'd forgotten it so the anger
occurred, I was in misery. Walking was at base and therefore observation
occurs.

Not observation occurring about it after.

Fictions are so simple that no observation occurs after.

So the fiction occurs first.

Defoe occurs first and so has no validity. It's not bathos and is only
distorted to be first or as being that.

So there's no Garden of Eden. There's not authority at all, that's to be
at base.

Walking occurs first and observation is not derived from it.

I want to still observation first. I started to walk one time with no
writhing and didn't remember beginning walking. I had nothing to writhe
about. There isn't a start of that, that's separated from walking.

One can eliminate power in oneself by not eliciting placating. Memory of
what hasn't been seen before. So fiction (I mean all conceptualizing)
occurs where the real event isn't isn't seen, it's eliminated, but is
occurring or has. Openly doesn't resemble it.

There's present-time first. That's what fiction is, when it is entirely
stilled. It doesn't resemble the event."

Fictionalizing drives the mind-terrain into a state of separation. There
the dream abuts itself. A real event (such as here an anti-war
demonstration) also abuts itself only:

"The dream I had is closer to being because it abuts it then. before
it's occurred. This is in the present. The blossoms on the plum trees
opened. On some. Birds sit, and fly amongst branches of bare trees. It
has no reverse side, and is thin.

Birds skim low by me.

Wall of police swinging their clubs sweep forward. young punks with
tattoos emerge from the crowd, turning the cop car over and burning it.
They receive beatings.

...A memory as just a thin disc, as it is seen that way. There is no
event.

Warm, the birds stirring and flying. I had a memory of being on an empty
street in San Diego near-by the ocean; just that. that as a disc because
it became so stagnant it seemed it was going to collapse, and be still.
it flapped. my mind had become so tired its resistance was going. The
mind being weary as clarifying in itself.

... 'false joy' because they're lying to us. but inside it is not their
'false joy.' It could be one's own, that is a thought. which in itself
is joy which bangs against it. The dark cattle stood earlier with the
disc of moon on the other side of them why they were dark; when the sun
was on the side (of them) on which they were dark.

That is in memory as the actual event.

The chest in one is struggling in constriction of joke. Trash. The air
cannot get into the rib cage. The thought of a bicyclist comes up which
is a memory and therefore separated from one, who is small. That was him
and so flattened shooting ahead. His back, that makes one laugh
constricted inside so that it's him.

So the flapping swans here so they're the same as the sky simply flying
is that (in time), before.

The entire day completely dilated is flapping in one's recesses so it is
behind in the back of one. Stagnant in that sense, as it's flapping and
then narrowing it in front though it's still the day.

It's the day in a sense, narrowed to a thin disc in front in which the
entire dilated day behind in the recesses of oneself bangs shuttering on
it. The day collapses in that sense when that's slowed and interminable.

One rests.

One just sees from one's social group perspective say and then the huge
bright day is dilated flapping with that; so it's involuntarily let go
as it's large and while hitting the recesses in front of one in the
present, remains there shuttering."

Comic book style, the apprehension occurs via "the other" who has a hole
(or eye) in her side, everything seen alongside and flattened. The
visual slots or frames of the comic book (only conceptually implied) can
include all layers and perspectives at once, horizontal and vertical.
The Other is in love with James Dean who in death is turned into a deer.
She pursues the deer to rescue James Dean as the black rain falls. She's
traveling through refugees dying from blood blisters on the roads. Her
own blood blister is gently lanced by a soldier who's driving the jeep.

The incommensurable relation between outside-action (in the world) and
interior-action mind phenomena is overtly the subject. The shape the
writing takes is related to use of fictional and
autobiographical/historical event (such as the bombing of Iraq) at the
time it's occurring. The time it's occurring the paragraph-plates are
real-time moment of exterior event brought into compressed relation of
interior occurrence/horizon together and always separated. The intention
is that fictional illusion be at the moment of real-time to: implode it
as real-time action.

These are illusions in the practical sense of being 'only' writing
(writing has no relation to present or historical reality it has no
reality, is it as well, being mind phenomena. So the 'ordinary' small
action is {[}to be{]} as much 'reality' as events that are devastating).
I am trying to divest hierarchy-of-actions. 'Hierarchy-of-actions' voids
people's occurrences (that is, individuals' actions are relegated to
inconsequential or invisible). Such hierarchy substitutes 'overview' of
'history'/interpretation/doctrine therefore to divest
'hierarchy-of-actions' is certainly a political act. (In one's/reader's
/viewer's conceptualization then {[}is the intention{]}). What I'm
referring to as 'divesting hierarchy-of-actions' by definition has to be
in oneself.

Making illusions and noticing them makes them so overt that they
'become' bathos. (I don't regard that as "alienation," however.)
Fundamentally anarchism (viewing that as being observation itself) is
necessitated. Besides \textbf{Defoe}, my other prose works (crossed with
being poetry) are: \textbf{The Return of Painting}, \textbf{The Pearl,
and Orion/A Trilogy} (reprinted by Talisman, 1997) and \textbf{The Front
Matter, Dead Souls} (Wesleyan, 1996). Also \textbf{Orchid Jetsam} (in
manuscript) and \textbf{R-hu} (forthcoming Atelos Press).

~

\hypertarget{bottoms-up}{%
\section{Bottoms Up}\label{bottoms-up}}

Gail Scott

MONTREAL, QUEBEC: I return to the site of an old novel. This will not
always be a street of failure, even if still playing the former glamour
of the seedy. French-language covers of country music are playing on the
radio. Men (sit) alone at tables. One with haggard bearded face. His
lower jaw (is) toothless. Teeth in dreams = sex. The low façades
opposite parading peeling tin cornices. An air of stagnation possibly
verging on cliché. I take some verbs out; then put them in again.

It is the hottest summer ever. Meteorologists are, with divine
hindsight, admitting global warming. To escape the heat, I go to a dark
afternoon theatre. Where a sassy high school troupes performing Büchners
\textbf{Wozeck}. The adolescents sweet songs juxtaposed on the gloom and
violence under, somehow dissolving earlier-decade interpretations
foregrounding worn-out existential despair. Crossing a parking lot
behind a smoked meat shop, I am thinking a story, to avoid eternally
returning (to keep it moving), might be structured thus: an "older"
woman writer circulates at a "fringe" theatre festival of young artists.
She has lost her love. Feels discombobulated. Nothing. She skirts the
crowd asking several 20-year-olds: anything good? She will then string a
humiliated (but raunchy) love tale together, interlaying bits of
youthful sweetness and anger, representing to her in her present state
of emptiness, a more contemporaneously rapid conflation of beginnings
and endings. A word about narrative:

At another table, a man shakes his purple-shirted shoulders in the hot
bar air. To "Caught With Your Pants Down." Drinking straight tequila.
The same table where X and I sat. Discussing why our favorite writers of
new and experimental fiction are not rich and famous. "Too abject," we
say. In a bag at our feet, some cheaply extravagant dollhouse boudoir
furniture purchased for her little son. From the store with the saggy
awning opposite. Leaving, we forget it. Outside another man passing.
With several teeth missing. Teeth = action. But why the narrator (the
woman) endlessly remarking. Such toothless ghosts of subjects. Among
dozens of better-furbished faces. Is her "local" constructed out of some
kind of distancing or bias?* Or "mere" projection of loves losses? I am
thinking \textbf{not} abject the subject. Unless negatively framed. At
point of crossing space of (non)-reception between the work and the
world. When site (interior)s signed minor (inadequate). While
exterior=dominant. Yet, intrinsic and extrinsic syntax are not separate;
rapidly they mate. Should I be happy the lover having sex with no one.
If having only virtual sex with me? I take some verbs out. And put them
in again. A word about the waitress:

Dominatrix in clinging leopard-spotted polyester. Small breasts.
Ponytail. Fortyish. Saying (in québécois): UN verre de vin-maison ne te
ferait pas de mal, one glass of house wine will not ruin the stomach.
I.e. its terrible. For 75 cents more you can have a quarter liter.
Adding: you look en forme, terrific. WE tell each other-\/-in guise of
complimenting-\/-we haven't changed in years. I opting for the smallest
quantity of red vinegar. Now young woman passing by window. Huge breasts
encased in tight halter top. Some little cap over short hair.
And---upper incisors gone. Like the meteorologist, I admitting only to
coincidence of circumstances. Projecting accumulated lack on
unsuspecting bodies. (Waitress bringing unasked for quarter-liter).
Which bodies, in being thus sentenced, already frozen in past gesture or
alias. While environment drifting elsewhere. As in that restless white
river the lover (much later) wanting to attack with kayak. Drifting
eternally toward full-color horizons. The silver flashes indicating
minor notes of salmon struggling upstream. Sitting close beside her, I
encouragingly declaring such projection, battered by tergiversation of
context, likely best position. For contemporaneous narrator. Just as
lover putting hand under skirt cut on the bias to flare out attractively
at the bottom. And saying she still wanting something with "us." A story
is something---you can put your teeth into:

But what might happen here? With (soon) a closed sign across the window.
Two blocks down. A polished glass and marble multi-story, multi-venue,
multi-media repertory theatre. Having replaced the crushed blue façade
of the smoky cozy café-cinéma where we used to see old or experimental
European French movies. Built by Québécois Soft Image inventor Daniel
Langlois. Were like that here: always oscillating between the
ultra-modern and the crumbling. The sound systems so efficient, one
feels, listening to the Buena Vista Social Club, as if in some club in
crumbling Havana. Though the octogenarian musicians having mostly ceased
performing on their blockaded island. Because "nowhere" to go. Until
American Ry Cooder choreographing their (global) return. I am thinking
of when Québec liberationists claiming solidarity with Cuba. Which
solidarity currently updated to meaning "corporate opportunity." At
last, the Cubans playing Carnegie Hall. I am thinking how any narrative,
if fronted, marketed from economically preferential geographical site,
avoiding being troped as exotically remote, a remoteness that must
(paradoxically) scale global walls or languish. I am thinking of Cuban
sunsets, suffused towards evening with a little blood. Of the womans eye
growing humid, sentimental. When, in the movie, the elderly guitarist
exclaiming: "There is nothing like a night of love." To keep it moving,
I take some verbs out and put them in again. A word about identity:

Now the waitress in her leopardskin is leaning over the purple shirt.
Whos reading a sexy mens magazine. Pausing, he pulls a huge fluorescent
flashlight still wrapped in hard plastic cover from a bag. She points
out (still speaking {[}naturellement{]}, French): its no good without
batteries. Laughing. I can smell him. The conversation segueing to
European-French actress Catherine Deneuve. The waitress declaring in
drawn-out Québécois dipthongs the Europeans looking down on, Deneuve
"plate," boring: "Always the same tone." Outside a friend passing by
window. Looking for a date. Very strong shoulders. Her long red hair
streaming out behind. I know she has latex gloves in her bag. Walking
naturally, she "vogues." Causing her to get thrown out of womens
washrooms. For being "wrong" gender. I am thinking how certain minor or
unusual demeanors, if flaunted, getting read extrinsically as
unappealing or threatening. (In Canada, "Québécois" often = flamboyant
or risqué; coincidentally, same thing for "queer"). This can be
devastating. Or interesting. Friend passing again, very feminine in
wrap-around skirt, curly red hair. Yet walking as if faking it. I am
thinking how narrator (other). IF not knowing how friend
herself-self-perceiving, able to create a relationship with reader
permitting permutation of both obvious and unspeakable. The body being
capable of gestures contrary to understanding. I am thinking about
narrator as double. About evening out subject/object weight at each end
of sentence. A word about affect:

{[}It has never occurred to me to be a poet. The poet, with rare
dazzling exceptions, even when "absent," looms imbricated. In her total
façade of language. While reader happily at play. In her in sandbox of
spaces. Lacking spontaneity, I am drawn to the violence of animation.
Experienced by a subject. Drifting towards object. Across the placement
of the verb. Which nakedness, exposure, seeming somehow progressive,
egalitarian. Yet failing to make woman (narrator) cease obsession with
beginnings (causes) and endings (conclusions). So trauma still risking
terminating in single unbearable seduction scene. I am thinking of
certain literary feminists. Who using devices copped from poetry. To
construct porous or unbounded subject. Capable of merging "more"
ecologically with context. I am thinking of the gap of the unspeakable.
There may be no animal boundaryjust the stream and the pleasure that
lies in it, teasing the poet.** I am thinking about the portentousness
of sentencing. Alternately (defensively?), I am thinking that a sentence
in a community of sentences (paragraphs) somehow leaving impression of
consciously reaching out to other(s). Notwithstanding characteristic
narrative flick of head back over shoulder. At point of the period. For
purpose of getting . . . bearings. This I finding . . . touching. A word
about the geography of the bar{]}:

Many empty tables. Two beefs in dark corner. Chairs fading outward.
Towards sunwashed waitress, purple shirt and harsh light of street.
Waitress, flashlight and me. A sign above our heads: "Les crevettes de
MATANE arrivées!!! {Fraîches.} MATANE shrimp in!!!! Fresh. The drawn-out
Québécois fraîîîches savorously feminine and plural. The waitress and
purple shirt still complaining Deneuve lacking edge. Offering as
replacement, Québécoise country music star. Recently coming out with
"fabulous CD." Though "star" still having to work nightly in bars. "Very
tough. If youre pregnant." Outside, young nomad in artfully torn outfit
and \$50-dollar-haircut. Asking for change in anglicized French. They
come here from the suburbs. Not to feel guilty. In this huge dark room
with stuffed animals on walls, I am thinking of simulation in a
diminshing French-speaking city, where everyone considering themselves
minority; yet in some way also faking it. I am thinking of consequence
for narrative, so often straining toward "natural" on site where
continental popular culture in perpetual transmigration from Germanic to
Latinate. Of Mickey Mouse, Patsy Klein, Sylvester Stone, moving lips in
English. While saying something else entirely. I am thinking of living
on a Renaissance Ponte Vecchio: less bridge between than Babel of
echoes.** Whose interpretation requiring, precisely, resistance to
meaning. I am thinking of narration as introjection. Many friends have
dentures. A word about hybridity.

Two blocks up. Little stages. Blue-and-white fleur-de-lys flags for the
fête nationale. The Argentinean tango I missed. Somehow having erased
Québec "patron" Saint Jean Baptistes bleeding and miserable little lamb.
Featured in earlier decades national feast-day celebrations.* Fête
progressing north. Group doing samba. Québécois folk songs. Block
production of Molière farce. Free good food. Everyone speaking French
"with accent." Whole family dragging mahogany dining table out on
sidewalk. Now eating supper. Later, I, sleepless, listening to
thunderclaps on John Cage piece. To compensate for dryness. Leopardskin
waitress offering another glass of vinegar. Muscled guy in tight
sleeveless t-shirt and gleaming chain with cross. I am thinking of
narrative as opaque barrier. There is no limit between life and deathly
fascination. Viva Che! on wall. Now I must go before another cigarette.
Buy food. Take messages. If any. The disease is not under control.
Police car outside.

~

\hypertarget{bottom-notes}{%
\subsection{Bottom Notes}\label{bottom-notes}}

This piece is dedicated to Carla Harryman.

It is in dialogue with other writers and other texts, including the
writing of, and conversations with, Dianne Chisholm, Robert Glück
(notably re: the narrator as bottom), Carla Harryman, Camille Roy, and
Sarah Schulman.

** Textual references are to Barrett Wattens \textbf{Total Syntax}, a
cover citation from \textbf{Sight} by Lyn Hejinian and Leslie Scalapino,
and Sherry Simons essay "The Ponte Vecchio and the Comma of
Translation."

Saint Jean Baptiste Day, June 24, is a national feast day in Québec.

~

\hypertarget{narrative}{%
\section{Narrative}\label{narrative}}

Juliana Spahr

The water is blackish, green, and dark.\\
It gathers from its\\
separated state, gathers from rain, gathers into stream.\\
It gathers in the mountain.\\
It gathers then travels, collects to become brackish.\\
Water travels, falls over a cliff, is falling.\\
Water is falling.\\
Falling onto rocks.\\
Because water falls it is.\\
Because water streams it is.\\
Because water collects into a pool it is narrative.\\
The pool is cold.\\
The pool is sheltered from sun by a cliff.\\
The pool is filled with rocks.\\
Water gathers over rocks, on rocks.\\
Moss gathers over rocks, on rocks.\\
Three arrive at this scene.\\
This scene where water is cold.\\
Where rocks are mossy.\\
When the three enter the pool their feet slip on rocks that are\\
mossy.\\
As they slip, they slip deeper into water.\\
Slip on mossy coldness.\\
Slip into water so cold it makes their chest close.\\
Slip deeply in.\\
Water gathers them, gathers them by their slipping.\\
Water covers them.\\
Because water is brackish it is narrative.\\
Because water is brackish they don't open their mouths.\\
Because water is brackish they are closed in their immersion.\\
They do immerse.\\
But they don't open as they swim to falling water.\\
They stand beneath falling water.\\
Water beats down on them.\\
Beats on their shoulders.\\
Beats on their heads. Beats on them.\\
They stand beneath water that is a roar of a falling.\\
They stand in roar, in narrative.\\
But water is brackish.\\
So they do not kiss.\\
Do not open their mouth beneath the waterfall.\\
They let brackish water fall over them.\\
Over their heads.\\
Over their lips.\\
Over their eyes.\\
Over their ears.\\
Over their hands which they hold up to feel the roar landing into\\
their palms.\\
Down their body.\\
Brackish water.\\
Closed bodies.\\
Water gathers them to this place, this narrative place.\\
Water covers them and they are covered with brackish water.\\
They do immerse.\\
They don't open.\\
Immersion seals them off.\\
There is no open mouth.\\
No opening.\\
No exchanging.\\
Because it is brackish it is narrative falling over them.\\
It falls over.\\
It falls over them.\\
Sealed yet together.\\
They have come together with brackish.\\
They have let brackish wash over then even if they don't let it\\
inside.\\
Because water falls it is narrative.\\
Because they are immersed it is narrative.\\
Because they love each other while separated it is narrative.\\
Because the rocks are slippery with green moss it is narrative.\\
Because they slip on the rocks it is narrative.\\
Because they slip deeper into water.\\
Because they allow the slip, are prepared for the slip and love its\\
immersion it is narrative. Because it involves love it is narrative.\\
Because it leaves them alone it is narrative.\\
~\\

\hypertarget{why-write-narrative-when-you-could-be-the-next-guest-on-the-jerry}{%
\section{Why write narrative when you could be the next guest on the
Jerry}\label{why-write-narrative-when-you-could-be-the-next-guest-on-the-jerry}}

Springer show!

Anne Stone

I met someone whose tattoo had been sliced out of his arm, leaving a
jagged scar in place of the name of city he was born. I understood the
urge, to inject a place under your skin, only to have it removed like a
second stomach or a wart. Lately, I've been thinking a lot about place.
If you ask me where Im from, I can answer in a lot of different ways,
none of which would tell the whole story, or even manufacture the kind
of sense I would want to live with. I could show you the scar on my leg
and point to an intersection, or trace the scar angling up from the
first phalange on my index finger, pointing out the parking lot where
the Streetsville liquor store used to be. I could tell you about strip
malls, or stock yards, or franchised donut shops that remind me of
Dresden the morning after, draw you a map of the territory in the fine
sprinkling of ash collecting on a Formica counter. I could tell you
about the derelict cornfields that used to be where that donut shop is
now, or the first place I lived in St.-Henri, show you the abandoned
granary that burned to the ground seven years ago and has since been
replaced by industrial trailers. Or maybe the building I lived in
itself, which is sealed off now. I might point out the remains of a much
older structure, a series of barn-board sheds. I would go to one of
those sheds each afternoon to write. The pick-up-sticks anatomy of that
place inspired local kids into acts of arson. I dont know if it was my
own narrative tendencies or the steady supply of thirteen year olds with
zippos, but it was the first part of the building they chose to tear
down.

I should tell you, before we go any further, that the scar is really on
my pointer finger. You should know this story is not particularly
reliable and, as a friend told me once while I was staring absently at a
menu, the map is not the territory.

Somehow, during the months I spent in that shed, writing what would
become \textbf{jacks: a gothic gospel}, the smell of the packing plant
my father took me to as a girl infiltrated the barn-boards. The plant
was on his delivery route, and I used to stop off at the stock yards to
pet the shit-stained flanks of cows as they were herded into slaughter.
St-Henri reeks of my childhood and the Streetsville liquor store is as
organized as a months supply of birth control pills, the bins of
national and international wines formatted on the principle of a 28-day
dial, the kind that comes complete with seven sugary pills to remind you
of the color red. When I think of all the places that have come to
dereliction, I understand myself in terms of the word tenacious. I know
it is only my thinking that makes it so. I dont know if they hand out
birth control pills to fourteen year olds anymore, the kind that remind
me of a flattened representation of a monthly agenda or a rolodex with a
daily regimen of business cards. I do know that the smell of those
stockyard cows do not belong in this essay. Nor does the way they kill
cows in the Eastern Townships, which is much more and much less
personal.

I also think its important to tell you, before I go any further, that I
am not a vegetarian.

On my friends farm, the breeders have names like Sophia or, out of
earshot, are called after her daughters-in-law. The cows she will
slaughter and wrap herself are named after beige-toned bureaucratic
dociers. When I first met the woman who owns this farm, I asked about
the cows and was given the names of the breeders and the bull, the
others implied by a slurred ellipsis or existing as the space-off to her
story. I persisted, asking about one of the cows shed passed over.
"T-2306? Or do you mean that one?" she asked, pointing to Sophia,
"theyre sisters." A little later that day, she promised to save that
cows thigh-bone for my dog.

The cows exist in terms of place. The cows of the stockyard and the cows
of the townships farm are the same breed, indistinguishable, except for
their deformation in terms of place. Even their smells are different.
Sophia and T-2306 are twin sisters, born on the same day, from the same
uterus.

If I were to write about T-2306, I would want something of that tension
to come through. The format of the place infiltrating something which
might be appalled identity. Or perhaps by identity I mean body-as-place.
If I were a photographer, I might take a long-exposure photograph, the
kind that effaces her body if it implies movement, if her body-as-place
shifts in relation to geographical setting. The immobile objects and
landscape would exist in the picture as perfect solids. It would be an
unremarkable photo, except for the way you could see right through the
cow to everything behind it. Except for the way the picture would be
taken as daylight gave way to dark, and something of that lumen would
gather the hide together into the appearance of a bruise.

The length of exposure is an interesting way for me to begin to think
about length in narrative. As I am given to understand the technique, if
you lengthen the time that the shutter is open, the camera slowly tucks
whatever light there is into an objects surfaces. Ive seen night photos
taken with a long exposure, the light isnt reflected off the surfaces of
an object so much as collected there, staining the fabric. The light
appears to be a dirty liquid, seething up from under. The implications
fascinate me, the way that pulling sight long can deform what is there,
or ways of seeing it.

Thankfully, I am not bound to a single snapshot, because I want to
examine (and re-examine), lit (et relire), bed (and embed), these
surfaces, under-pinning the impoverished look of St-Henri, for instance,
with the reek of a west-end Toronto slaughter house, twenty years
distant in time, and 600 kilometers away. I want the space-off of the
shed I started writing \textbf{jacks} in to be that terrifying plant.
God: What is that? What is it? The dingy white towers next to the
stockyards are organized vertically, even though, as everybody knows,
cows are organized horizontally---they are grazers, right? So, between
the slaughterhouse and the packing plant some terrible deformation of
what it means to be a cow occurs, the axis of cowness undergoes a
grotesque skew. How do you even fit the word "cow" in a building that
tall? What could a building with elevators possibly have to do with a
cow? When I was a girl, I wasnt afraid of the men with electrical prods,
or the smell of shit and death. But I was terrified by those vertical
towers. I knew they embodied a cold and impersonal force. If I called
out to the men prodding the cows, for instance, I could get them to talk
to me, stop what they were doing for a while. I doubted my presence
could effect a pause when it came to anything in or about those
buildings. It wasnt so much that I didnt exist, as it was that in their
terms, I could not possibly exist. I lacked the relevance to insist my
own axis likewise undergo some miraculous Holstein shift from x to y.
The plant excluded all terms but those present in its formula. If I had
Goëdal for a companion on those walks through the slaughterhouse, they
might have made another kind of sense. As it is, the memory refuses my
attempts at sense-making, and continues to fascinate me twenty years
later. The shed I came to narrative in has been torn down, and the
vertical granary -\/- the one that existed at the place where Beaudoin
stutters to a stop at rue St-Ambroise, the one that once implied the
axial-shift undergone by a field of wheat -\/- burned to the ground one
winter night as I watched. These places have a grammar that I am only
now beginning to grapple with (or make-up).

I should probably confess that Ive apprehended a similar syntax in
unfamiliar restaurants. Once, a maitre-D unfolded my body like a table
cloth, the surface tension flattening and lengthening my body to the
skin of water, it delineated a lumen that precisely. Sensing some
terrifying grammar, I found that the only punctuation legible to me were
the round bottoms of oven-warmed plates pressing into my back like a
series of medieval cups. The kind 17th century doctors heat with candles
before placing in rows on either side of the spine, creating a series of
tiny vacuums to draw the suspected impurity up from under. I find
restaurants inscrutable. I mean, who would think to spell the word
"spent" on a plate in a diagonal line of cutlery? But god, the relief to
discover there was a silent means to signal an impending departure.

What happens in the meantime? As your liminal-linen body hosts a series
of plates? And yes, physically, the body is a place that can, with some
effort, be occupied by one's self, or by others.

I am not at a restaurant. If I close my eyes, I lose my boundaries, the
air is tepid against my skin, no rough frisson strokes this surface. I
am not sensitive enough to maintain a secure sense of where I end and
something other begins. That requires a lepers perpetual gaze to the
extremities. If I were to close my eyes and begin typing by rote, you
would have to remind me: Where are my arms just now? my fingers...

I am only sure of the tips. As they pad against the keys. One of my
fingers describes an unstable arch. Ive pulled a tendon with this
incessant tapping. It must be a very tiny tendon, the kind of thing you
might extract from the haunch of a bisected frog. Eyes closed, I am the
stoop of shoulders, the curve of a sore finger, and a series of
insensitive pads Ive developed as a result of my obsessive predilection
for narrative -\/- long narrative.

This is where I am today, and not at a restaurant packing plant. I am at
the place the scar on my index finger describes, an intersection between
place and the body. Writing narrative affords me a lateral way in which
to think about the body, about place, about the body in place, about the
body as place. In \textbf{Hush}, written very much like those
long-exposures I talked of earlier, the women exist in terms of
place---a narrative focus that, on the surface, would deny movement. But
writing against the vertical axis, implying it, even as I choose to
write on the horizontal plane, I can let a particular image accrete to
itself all sorts of other resonances---not just in the way a particular
image occurs in relation to her-body, Roses body, Loralies body, but how
it recurs, in relation to other events or objects or the place itself.

Plot is a term that I hear twice, I hear the English plot and the
français plotte, or cunt---so it is a gendering of narrative. I am not
about plot. But am about this other form of plotting in narrative. Of
telling a story that unfolds at the level of language, where the words I
use recur over the length of the narrative. The repetition becomes
skewed, and what it touches, likewise becomes communicable. It isnt
mood, or even sensibility Im aiming at, though thats what descriptive
words are employed for a lot of the time. No, what is here is as
convoluted as the vaginal tract of a snail, and though yes, snails
plottes are bundled relatively small, they do possess immense surfaces
to play on.

I should admit Ive forgotten the word for a word that is descriptive.

Where am I going next? I want you to imagine a picture taken with a very
long exposure. No.~I want you to imagine you think of the phrase
"glory-hole" every time you hear the word exposure, and then I want you
to imagine a picture taken with a very long exposure. A picture taken on
a night at the circus. The temporal trace of the antics of acrobats and
Dahlmer-clowns rendered in fly-wheels of color. Are you sufficiently
distracted? Good. Because until a moment ago I was holding a Polaroid in
my hand. It was a very old picture, or else, someone has gone to a lot
of trouble to make it look that way. The edges are black from fingering.
If I was going to show you the picture, evidently, I changed my mind,
because while you were looking at the circus-shot, I slipped the picture
into my pocket and lit a cigarette.

I think you should know that I am familiar with the practical
applications of luting agents.

\hypertarget{explain}{%
\section{Explain}\label{explain}}

Michelle Tea

~

I get asked sometimes to defend the way I write. To explain how it's um
writing I guess. Because I'm not making stuff up so why do I think that
anyone wants to hear about my crappy life. Well then don't read it,
punk! It's all I can think about, my own life and your life too. I mean,
what's everybody doing? I really want to know.

What if fiction writers had to answer dumb questions like that. Like,
What makes you think we believe this crap? How the hell do you know what
that geisha was thinking, white boy? Or, What are you afraid of? Are you
trying to hide something? Who are you, where can I find you skulking
about in these pages?

Oh I don't want to divide and conquer the writers. I like fiction, I
just read \textbf{Jesus Saves} by Darcey Steinke and it nearly put me in
a coma. My stomach hurt so bad I had to put it down, rest it on the dash
of a van that was rumbling through woodsy mountain landscape, manly
land, hunting land, in a state famous for serial killers, and in the
pages of this made-up story a little girl is tied up captured and doomed
and I know it's true, I mean, these things happen. I read Jenn Banbury,
\textbf{Like A Hole in the Head}, and it made me want to fuck up on some
spectacular way, as opposed to my normal, low-key fuck ups. Those
stories were really good.

But I just think it is so exciting that life can be so meaningful, that
you can smear it on a page and it's really important. It's as much of a
story as any fiction, but it's a document too. It happened, so it's
history. I'm reading \textbf{The Unsinkable Bambi Lake}, it's before me
on the table right now beside a rack of blueberry muffins and a pile of
bills that can't seem to pay themselves. When I read Bambi's book I get
to hear all about how cool my city, San Francisco, was in the 70s. All
the stuff I missed, victorians stuffed with glamorous and dramatic G.A.
queens, glitter and dustbunnies in the corner, Bambi backstage in Berlin
giving Bowie shit for selling out way before that Dancing in the Streets
fiasco. Dangerous boys and hustlers, getting on estrogen, really just
living a life, her own. This question about 'memoirs', I think: are they
asking about form or structure, or really, am I being asked if my life,
if Bambi's life and Eileen Myles' life, and Cookie Mueller's life, are
they asking if our lives matter? I think they are.

I'm lucky because I get to be friends with some of my favorite writers.
I sat with Eileen Myles at a coffee shop on Valencia and she told me
that Charles Bukowski gave her permission to write. He just claimed it,
his terrain. The horseraces and the bars, the night, the lamp that lit
up his lousy hotel. Why not him. A poet. Why not Eileen. I read her
book, \textbf{Chelsea Girls}, and it all clicked. ``Bread and Water,''
what is she doing in that story but being poor. Scrounging for money,
tallying debts to friends, having her period, drinking and not drinking.
It's a genius story and the whole book is full of them, electric. I was
zapped. Why not me. My poverty and the girls that don't love me and how
drunk I got the other night. How I was a prostitute. It seems to be
literature when guys write about it, its practically become a genre, men
writing about their transcendental trips to the cathouse, their orgasms
and revelations. Or men writing about women's lives in general. Straight
people writing about queers and white people writing about every other
race on the planet. The writing that I love, it's the Other telling the
part that got left out, the truth. Not only a writer and a historian,
but a spy. There's this awful copy shop near my house, I go there all
the time because I'm too lazy to walk up to Kinko's. The guys at this
place are such jerks. I had a bunch of my books and he said, Are Those
Your Books? Yeah. You wrote them? Yeah. He makes this suspicious little
scrunched-up face. Are You Sure? he asks. He means it. Looking at my
dirty fucked up hair and tattoos scrawled up my arms and whatever else
he saw. You Just Don't Look Like You Would Be A Writer. Yeah well keep
an eye out for yourself in my next novel, asshole.

I'm at work and I get a phone call and it's this girl. It's 2:30 in the
afternoon and she just woke up. I can hear birds chirping in the
background like she's in some fabulous jungle and I imagine light
slanting through fat glossy leaves and her all rumpled from sleep in a
slippery slip that sticks to her bones. Do You Have A Bird, I asked. I
Do. What Kind? A Cockatiel. I wonder what its name is but don't ask
because I want to keep it wild the way she is. She got arrested by the
cops for tagging Freedom Is Dying in a phone booth, they hauled her off
to jail. I feel this teeny obsession with her forming, and it feels a
lot like needing to write. A sort of zinging that grips you from the
inside. There's a place for her in my life right now and so I need to
write her down. The other night we were both high in that club that's so
red, a deep low red that made everything red, my dress and torn
fishnets, her shiny legs she kept twisting around, unfurling like a
mermaid. The red beer label curling on the bottle that she begged off
the bartender for free. I want to tell you all about her fantastic
strangeness, the sexy awkward way she walks like she's going to wipe
out, how she's paranoid, the excellent purity she sort of radiates out
from her eyes all black and smudgy, her dirty hair. I could tell you
about her jagged poems and drawings and the tattoos on her fingers and
the sad Johnny Thunders song on her answering machine. I could tell you
her whole story as I know it, and I want to, but it's not mine to tell.

And then she calls me the next day kind of wiggy, says I Got Tripped Out
Thinking I Was Going To End Up In One Of Your Stories. All the time
someone gets mad at me or else I worry they will. Where does my life
stop, where does it slide into someone else's story and then it's their
job to tell it. My ex-girlfriend is going to kill me when she finds out
I wrote about that time she flipped out and was banging her head against
my bedroom wall and I had to kick her out and she told me to fuck off. I
mean, that's personal. And can I mention that thing about her parents?
It seems so crucial to understanding who she is as a character. But
she's not a character, she's a person. And she had this affect on me.
Can I talk about it? When I went to Ali's house and she was cutting her
arms and there was blood on her yellow sheets and there was the razor,
kind of crusty on her dresser. I said Ali, I Feel Like That's Part Of
The Story, Can I Write That? She said, Tell The Truth, Michelle. But
then, Ali's a writer too.

Sometimes, maybe even right now, I don't want to write. It seems so
boring. There are so many rules and I made them all myself: not here in
the house, in a cafe or a bar. You have to take a walk, and think about
it while your walking, try to go back there, feel all hurt or fucked up
and kind of lost, feel really in love, you were just with the person,
you smell like her, remember how you ate a burrito afterwards, what it
felt like to crave the steak. You have to drink something while you
write, either coffee or beer, or sangria. If you're getting a late start
and the sun is down already then it should be a bar. You don't want the
place closing up around you when you're just hitting your stride, you're
going to stay up late immersed in all this life. Also you have to smoke
continuously, so pick a place that will let you do that. Maybe you
should get on a Greyhound and go somewhere else a place with romantic
terrain where you can be with your cigarettes. I think, maybe I should
just be in a band. Play some drums. Maybe I should take pictures, that's
as true as any story I've got to tell. But it's so expensive. Once I
didn't have any money, and I hardly knew anybody and I was so scared I'd
have to live out on the street. And I thought, it's ok I can still
write. Paper and pens are easy, they're cheap or easily stolen, you
could probably get people to give them to you. And if something really
bad like that ever happened, and whenever something does happen that
really sucks, it's hard, I think well ok I'll write about it. It's so
consoling and so redeeming. When I have a friend who doesn't write and I
see her having a hard time, like she's lonely or on too many drugs or
something fucked has happened I think oh I wish she could write it.
Write it all out. She could be an enormous angel. Probably she is
anyway, but then she could see it all lit up on the page, swimming
there, shining back up at her, good lighting for the movie that is her
life. She could write the soundtrack. She could bask in herself, it's
the greatest. Writing is the greatest, and writing the truth in
particular. Your little slice of it. You've got that, and you've got
your body, and I think that's it.

~
